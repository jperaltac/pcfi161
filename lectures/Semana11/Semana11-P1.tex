\documentclass[10pt]{beamer}

% ------------------------------------------------------------------------
% Carga de tu preámbulo personalizado (preamble.tex).
% Asegúrate de tenerlo en la misma carpeta para que \input funcione.
% ------------------------------------------------------------------------
\usetheme[progressbar=frametitle]{metropolis}
\usepackage{appendixnumberbeamer}
\usepackage{fancyvrb}
\usepackage{booktabs}
\usepackage[scale=2]{ccicons}
\usepackage{pgfplots}
\usepgfplotslibrary{dateplot}
\usepackage{type1cm}
\usepackage{lettrine}
\usepackage{ragged2e}
\usepackage{xspace}
\newcommand{\themename}{\textbf{\textsc{metropolis}}\xspace}
\usepackage{graphicx} % Allows including images
\usepackage{booktabs} % Allows the use of \toprule, \midrule and \bottomrule in tables
\usepackage[utf8]{inputenc} %solucion del problema de los acentos.
\usepackage{xcolor}
\usepackage{verbatim}

\definecolor{LightGray}{gray}{0.9}
% Paleta de azules estilo Python/matplotlib
\definecolor{PythonBlue}{RGB}{31, 119, 180}      % Azul principal matplotlib
\definecolor{LightPythonBlue}{RGB}{174, 199, 232} % Azul claro para fondos
\definecolor{DarkPythonBlue}{RGB}{23, 90, 135}    % Azul oscuro para líneas

% Colores para tema oscuro tipo VS Code/Colab
\definecolor{DarkBackground}{RGB}{40, 44, 52}     % Fondo oscuro similar a VS Code
\definecolor{CodeText}{RGB}{171, 178, 191}        % Texto gris claro para código
\definecolor{DarkFrame}{RGB}{60, 63, 65}          % Color del marco/borde

% Colores para tema claro
\definecolor{LightBackground}{RGB}{248, 248, 242} % Fondo claro tipo GitHub
\definecolor{LightCodeText}{RGB}{51, 51, 51}      % Texto oscuro para tema claro
\definecolor{LightFrame}{RGB}{220, 220, 220}      % Marco gris claro

\usepackage{minted}

%%%%%%%%%%%%%%%%%%%%%%%%%%%%%%%%%%%%%%%%%%%%%%%%%%%%%%%%%%%%%%%%%%%%%%%%%%%%%%%%%%%%%%
% CONFIGURACIÓN DE TEMAS - CAMBIA AQUÍ PARA ALTERNAR
% ========================================================================
% Descomenta UNA de las siguientes líneas para elegir el tema:

% TEMA OSCURO (VS Code style)
%\newcommand{\mytheme}{oscuro}
%\usemintedstyle{monokai}
%\newcommand{\mybgcolor}{DarkBackground}

% TEMA CLARO (GitHub style) - comenta las 3 líneas de arriba y descomenta estas:
 \newcommand{\mytheme}{claro}
 \usemintedstyle{default}
 \newcommand{\mybgcolor}{LightBackground}

% ========================================================================
\newcommand{\mypyfile}[1]{\inputminted[linenos=true, fontsize=\footnotesize, frame=lines, framesep=5\fboxrule,framerule=1pt]{python}{#1}}

% Comando personalizado para código Python inline
\newcommand{\pycode}[1]{\begin{minted}{python}#1\end{minted}}

% ========================================================================
% COMANDOS ESPECÍFICOS PARA CADA TEMA
% ========================================================================
% Comando para tema oscuro específico
\newcommand{\pyoscuro}[1]{%
\begin{minted}[
  bgcolor=DarkBackground,
  style=monokai,
  breaklines,
  frame=lines,
  framesep=2mm,
  baselinestretch=1.2,
  linenos,
  fontsize=\footnotesize
]{python}
#1
\end{minted}
}

% Comando para tema claro específico
\newcommand{\pyclaro}[1]{%
\begin{minted}[
  bgcolor=LightBackground,
  style=default,
  breaklines,
  frame=lines,
  framesep=2mm,
  baselinestretch=1.2,
  linenos,
  fontsize=\footnotesize
]{python}
#1
\end{minted}
}
%%%%%%%%%%%%%%%%%%%%%%%%%%%%%%%%%%%%%%%%%%%%%%%%%%%%%%%%%%%%%%%%%%%%%%%%%%%%%%%%%%%%%%



%%%%%%%%%%%%%%%%%%%%%%%%%%%%%%%%%%%%%%%%%%%%%%%%%%%%%%%%%%%%%%%%%%%%%%%%%%%%%%%%%%%%%%
% Configuración global para todos los bloques minted de Python
% Usa automáticamente el tema seleccionado arriba
\setminted[python]{
  breaklines,
  frame=lines,
  framesep=2mm,
  baselinestretch=1.2,
  bgcolor=\mybgcolor,
  linenos, 
  fontsize=\footnotesize
} % Configuración dinámica según tema elegido
%%%%%%%%%%%%%%%%%%%%%%%%%%%%%%%%%%%%%%%%%%%%%%%%%%%%%%%%%%%%%%%%%%%%%%%%%%%%%%%%%%%%%%



%%%%%%%%%%%%%%%%%%%%%%%%%%%%%%%%%%%%%%%%%%%%%%%%%%%%%%%%%%%%%%%%%%%%%%%%%%%%%%%%%%%%%%
% Configuración de colores del tema con paleta azul Python
\setbeamercolor{progress bar}{fg=DarkPythonBlue,bg=LightPythonBlue}
\setbeamercolor{title separator}{fg=DarkPythonBlue,bg=white!50!black}
\setbeamercolor{frametitle}{fg=white,bg=PythonBlue}
\title[PCFI161]{Programaci\'on para F\'isica y Astronom\'ia}
\subtitle{Departamento de Física.}

\newcommand{\myfront}{
\author[PCFI161]{Corodinadora: C Loyola \\ Profesores C Femenías / F Bugini / D Basantes}
\institute[UNAB]{Universidad Andrés Bello \\ Departamento de Física y Astronomía}
\date{Primer Semestre 2025}
}

\titlegraphic{
  \includegraphics[width=.08\textwidth]{1) Logo/logo-tux.png}\hfill
  \includegraphics[width=.3\textwidth]{1) Logo/logo-unab.png}\hfill
  \includegraphics[width=.08\textwidth]{1) Logo/logo-python.png}
}

\makeatletter
\setbeamertemplate{title page}{
  \begin{minipage}[b][\paperheight]{\textwidth}
    \vfill%
    \ifx\inserttitle\@empty\else\usebeamertemplate*{title}\fi
    \ifx\insertsubtitle\@empty\else\usebeamertemplate*{subtitle}\fi
    \usebeamertemplate*{title separator}
    \ifx\beamer@shortauthor\@empty\else\usebeamertemplate*{author}\fi
    \ifx\insertdate\@empty\else\usebeamertemplate*{date}\fi
    \ifx\insertinstitute\@empty\else\usebeamertemplate*{institute}\fi
    \vfill
    \ifx\inserttitlegraphic\@empty\else\inserttitlegraphic\fi
    \vspace*{1cm}
  \end{minipage}
}
\makeatother

% Configuración personalizada del frametitle para incluir la sección
\makeatletter
\setbeamertemplate{frametitle}{%
  \nointerlineskip%
  \begin{beamercolorbox}[%
      wd=\paperwidth,%
      sep=0pt,%
      leftskip=\@ifundefined{metropolis@frametitle@padding}{12pt}{\metropolis@frametitle@padding},%
      rightskip=\@ifundefined{metropolis@frametitle@padding}{12pt}{\metropolis@frametitle@padding},%
    ]{frametitle}%
  \@ifundefined{metropolis@frametitlestrut@start}{}{\metropolis@frametitlestrut@start}%
  {\textcolor{LightPythonBlue}{\insertsectionhead}\ifx\insertsectionhead\@empty\else\ $\ni$  \fi}\insertframetitle%
  \nolinebreak%
  \@ifundefined{metropolis@frametitlestrut@end}{}{\metropolis@frametitlestrut@end}%
  \end{beamercolorbox}%
}
\makeatother

\makeatletter
% Configuración segura de longitudes del tema Metropolis
\@ifundefined{metropolis@titleseparator@linewidth}{}{\setlength{\metropolis@titleseparator@linewidth}{2pt}}
\@ifundefined{metropolis@progressonsectionpage@linewidth}{}{\setlength{\metropolis@progressonsectionpage@linewidth}{2pt}}
\@ifundefined{metropolis@progressinheadfoot@linewidth}{}{\setlength{\metropolis@progressinheadfoot@linewidth}{2pt}}
\makeatother



\begin{document}

% ------------------------------------------------------------------------
% Portada de la Presentación
% ------------------------------------------------------------------------
\myfront{}

% ------------------------------------------------------------------------
% Slide 1: Título de la Sesión
% ------------------------------------------------------------------------
\begin{frame}
  \titlepage
  % Por ejemplo:
  % \title{Semana 11 - Sesión 1 (Sesión 21): Continuación de POO y Análisis de Datos}
\end{frame}

% ------------------------------------------------------------------------
% Slide 2: Índice / Tabla de Contenidos
% ------------------------------------------------------------------------
\begin{frame}
  \frametitle{Resumen - Semana 11, Sesión 1 (Sesión 21)}
  \tableofcontents
\end{frame}

% ------------------------------------------------------------------------
% Configuración de bloques
% ------------------------------------------------------------------------
\metroset{block=fill}

% ----------------------------------------------------------------------------------------
% SECCIÓN 1: Introducción y Conexión con la Sesión Anterior
% ----------------------------------------------------------------------------------------
\section{Introducción y Repaso}

% ------------------------------------------------------------------------
% Slide 3: Post-Solemne II Contexto
% ------------------------------------------------------------------------
\begin{frame}{Después de la Solemne II}
  \begin{itemize}
    \item \textbf{Semana 10, Sesión 2}: Estudiantes rindieron la Solemne II, abarcando los contenidos de:
      \begin{itemize}
        \item POO (clases, herencia).
        \item NumPy avanzado (matrices, linalg, random).
        \item Matplotlib (gráficas 2D/3D, subplots, histogramas).
        \item pandas (lectura de CSV, análisis básico).
      \end{itemize}
    \item \textbf{Objetivo de la Semana 11}: Avanzar con temas complementarios, retroalimentar la Solemne II y consolidar nuevas competencias. 
      \begin{itemize}
        \item Podría ser POO más avanzada (polimorfismo) o robustez en manejo de datos (filtrados, merges, etc.), de acuerdo con el Syllabus.
        \item Revisar la \textbf{retroalimentación} general de la Solemne II.
      \end{itemize}
  \end{itemize}
\end{frame}

% ------------------------------------------------------------------------
% Slide 4: Objetivos de la Sesión 21
% ------------------------------------------------------------------------
\begin{frame}{Objetivos de la Sesión 21}
  \begin{itemize}
    \item \textbf{Entregar} feedback global de la Solemne II, identificando logros y áreas de mejora.
    \item \textbf{Refrescar} la POO y ver algún aspecto adicional (polimorfismo, composición, etc.), según tiempo y Syllabus.
    \item \textbf{Explorar} un ejemplo práctico o un caso de datos más complejo (opcional).
    \item \textbf{Planificar} los siguientes pasos en el curso (si hay proyectos finales o nueva Solemne).
  \end{itemize}
\end{frame}

% ----------------------------------------------------------------------------------------
% SECCIÓN 2: Retroalimentación de la Solemne II
% ----------------------------------------------------------------------------------------
\section{Retroalimentación Solemne II}

% ------------------------------------------------------------------------
% Slide 5: Estadísticas Generales
% ------------------------------------------------------------------------
\begin{frame}{Estadísticas de la Solemne II}
  \begin{itemize}
    \item \textbf{Distribución de notas} (si aplica):
      \begin{itemize}
        \item Rango de notas y promedio general (sin datos reales, se ejemplifica).
        \item Observación de que \textbf{la mayoría manejó bien} la sintaxis POO, pero falló en subplots 3D, por ejemplo.
      \end{itemize}
    \item \textbf{Tiempo de examen} fue suficiente para la mayoría.
    \item \textbf{Entrega en CANVAS} con archivos .ipynb (como planeado).
  \end{itemize}
\end{frame}

% ------------------------------------------------------------------------
% Slide 6: Áreas Destacadas y Áreas a Mejorar
% ------------------------------------------------------------------------
\begin{frame}{Feedback General}
  \begin{itemize}
    \item \textbf{Lo positivo}:
      \begin{itemize}
        \item Claridad en la definición de clases, herencia y atributos.
        \item Uso de \textbf{NumPy} para operaciones vectorizadas o linalg.
        \item Integración con pandas para leer datos y mostrarlos en gráficas simples.
      \end{itemize}
    \item \textbf{Lo que se puede mejorar}:
      \begin{itemize}
        \item Estructura de \textbf{subplots} y personalización (leyendas, ejes).
        \item Falta de comentarios en secciones clave o poca explicación en Markdown.
        \item Manejo de \textbf{excepciones} o validaciones (opcional, según nivel).
        \item Unificación de código (evitar repetir lógica).
      \end{itemize}
  \end{itemize}
\end{frame}

% ------------------------------------------------------------------------
% Slide 7: Consejos y Sugerencias Post-Solemne
% ------------------------------------------------------------------------
\begin{frame}{Consejos Post-Solemne}
  \begin{itemize}
    \item \textbf{Probar y comentar} el código en cada paso, especialmente en exámenes o tareas.
    \item Organizar \textbf{funciones auxiliares} en vez de repetir (DRY principle).
    \item Para gráficas, usar \textbf{subplots} con nombres descriptivos (\(\texttt{ax1, ax2}\) en lugar de \(\texttt{axs[0], axs[1]}\)).
    \item Revisar \textbf{ejemplos oficiales} de Matplotlib/pandas para inspirar estilos y layouts.
  \end{itemize}
\end{frame}

% ----------------------------------------------------------------------------------------
% SECCIÓN 3: Continuación de Programación Orientada a Objetos
% ----------------------------------------------------------------------------------------
\section{POO Adicional}

% ------------------------------------------------------------------------
% Slide 8: Polimorfismo
% ------------------------------------------------------------------------
\begin{frame}{Polimorfismo (Visión Rápida)}
  \begin{itemize}
    \item \textbf{Polimorfismo} = la capacidad de usar la misma interfaz (métodos) en distintos tipos (clases derivadas).
    \item En Python, se basa en la \textbf{duck typing}: “si camina como pato y suena como pato...”.
    \item Ejemplo:
      \begin{itemize}
        \item \(\texttt{Particle}\) y \(\texttt{Star}\) pueden compartir un método \(\texttt{show\_info()}\) pero cada uno implementarlo distinto.
      \end{itemize}
    \item \textbf{Beneficio}: simplifica la lógica si tenemos una lista de objetos distintos pero un método común.
  \end{itemize}
\end{frame}

% ------------------------------------------------------------------------
% Slide 9: Composición (Has-A)
% ------------------------------------------------------------------------
\begin{frame}{Composición (Has-A)}
  \begin{itemize}
    \item \textbf{Composición} = un objeto está conformado por otros objetos (relación \emph{has-a}).
    \item Ejemplo:
      \begin{itemize}
        \item Clase \texttt{SolarSystem} que contiene una lista de \texttt{Body} (planetas, estrellas).
        \item Métodos para agregar, remover, iterar cuerpos.
      \end{itemize}
    \item \textbf{Ventaja}: modularidad y organización del código, sin heredar (no es un \textbf{is-a}).
  \end{itemize}
\end{frame}

% ------------------------------------------------------------------------
% Slide 10: Ejemplo: Sistema con Composición
% ------------------------------------------------------------------------
\begin{frame}[fragile]{Ejemplo: Sistema SolarSimplified}
\begin{minted}[
frame=lines,
framesep=2mm,
fontsize=\footnotesize,
bgcolor=LightGray
]{python}
class Body:
    def __init__(self, name, mass, x, y):
        self.name = name
        self.mass = mass
        self.x = x
        self.y = y

class SolarSystem:
    def __init__(self):
        self.bodies = []

    def add_body(self, body):
        self.bodies.append(body)

    def total_mass(self):
        return sum(b.mass for b in self.bodies)

    def show_bodies(self):
        for b in self.bodies:
            print(f"{b.name}: mass={b.mass}, pos=({b.x},{b.y})")

ss = SolarSystem()
ss.add_body(Body("Sun", 1.989e30, 0, 0))
ss.add_body(Body("Earth", 5.972e24, 1.496e11, 0))
ss.show_bodies()
print("Total mass:", ss.total_mass())
\end{minted}
\end{frame}

% ----------------------------------------------------------------------------------------
% SECCIÓN 4: Ejemplo Práctico o Actividad
% ----------------------------------------------------------------------------------------
\section{Ejemplo Práctico}

% ------------------------------------------------------------------------
% Slide 11: Ejemplo - Visualizar Distancias en un Sistema
% ------------------------------------------------------------------------
\begin{frame}{Ejemplo: Composición + Visualización}
  \begin{block}{Enunciado Breve}
    \begin{itemize}
      \item Usar la clase \texttt{SolarSystem}, agregar 3-4 \texttt{Body}.
      \item Crear una \textbf{matriz NxN} con las distancias entre cada par, usando \textbf{NumPy}.
      \item Graficar en un \textbf{heatmap} (\texttt{plt.imshow}) con \textbf{colorbar}, etiquetar ejes con \texttt{Body.name}.
    \end{itemize}
  \end{block}
  \textbf{Objetivo}: ilustrar cómo la \textbf{composición} (múltiples \texttt{Body} en \texttt{SolarSystem}) + \textbf{NumPy} y \textbf{Matplotlib} generan un análisis visual.
\end{frame}

% ------------------------------------------------------------------------
% Slide 12: Actividad en Grupos
% ------------------------------------------------------------------------
\begin{frame}{Actividad en Grupos}
  \begin{itemize}
    \item Formar \textbf{parejas/tríos}, retomar la idea anterior u otro ejemplo creativo.
    \item Reforzar:
      \begin{itemize}
        \item \textbf{Composición} (relación \emph{has-a}), polimorfismo si lo desean.
        \item \textbf{Matrices} de distancias o valores con \textbf{NumPy}.
        \item \textbf{Visualización} en matplotlib (heatmap, scatter, etc.).
      \end{itemize}
    \item \textbf{Comparar} resultados al final y plantear dudas.
  \end{itemize}
\end{frame}

% ------------------------------------------------------------------------
% Slide 13: Sugerencias Generales
% ------------------------------------------------------------------------
\begin{frame}{Sugerencias Generales}
  \begin{itemize}
    \item \textbf{Mantener} el código ordenado, con \texttt{def main()} o celdas separadas en Colab.
    \item \textbf{Documentar} la composición y polimorfismo si los aplican.
    \item \textbf{numpy.linalg.norm} para distancias, o una función manual \((x1-x2)^2 + (y1-y2)^2\).
    \item \textbf{plt.imshow} y \texttt{plt.colorbar} + \textbf{tick\_labels} si quieren personalizar ejes.
  \end{itemize}
\end{frame}

% ----------------------------------------------------------------------------------------
% SECCIÓN 5: Conclusiones y Próximos Pasos
% ----------------------------------------------------------------------------------------
\section{Conclusiones y Próximos Pasos}

% ------------------------------------------------------------------------
% Slide 14: Discusión de Soluciones
% ------------------------------------------------------------------------
\begin{frame}{Discusión de Soluciones}
  \begin{itemize}
    \item Compartir \textbf{cómo integraron} la clase \texttt{SolarSystem} (u otra) con NumPy y gráficas.
    \item Dificultades encontradas, \textbf{beneficios} de la composición vs. herencia.
    \item ¿Algún polimorfismo en acción (ej. diferentes cuerpos con un \texttt{show\_info()})?
  \end{itemize}
\end{frame}

% ------------------------------------------------------------------------
% Slide 15: Conclusiones de la Sesión 21
% ------------------------------------------------------------------------
\begin{frame}{Conclusiones de la Sesión 21}
  \begin{itemize}
    \item Revisamos \textbf{retroalimentación} de la Solemne II, con énfasis en fortalezas y debilidades.
    \item Profundizamos POO con \textbf{polimorfismo} y \textbf{composición}, aplicadas a ejemplos físicos/astronómicos.
    \item Continuamos integrando \textbf{NumPy} y \textbf{Matplotlib} para análisis y visualización en proyectos de clase.
    \item Sentamos base para proyectos finales o simulaciones más completas (si así lo indica el Syllabus).
  \end{itemize}
\end{frame}

% ------------------------------------------------------------------------
% Slide 16: Próxima Sesión (Semana 11, Sesión 2)
% ------------------------------------------------------------------------
\begin{frame}{Próximos Temas}
  \begin{itemize}
    \item Posible \textbf{desarrollo de un proyecto integrador} (si corresponde al plan).
    \item Expandir \textbf{POO} (más métodos especiales, polimorfismo avanzado) o \textbf{data handling} (estadística, merges, time-series).
    \item Revisar la \textbf{retroalimentación individual} de la Solemne II en detalle (notas en CANVAS).
  \end{itemize}
\end{frame}

% ------------------------------------------------------------------------
% Slide 17: Recursos Adicionales
% ------------------------------------------------------------------------
\begin{frame}{Recursos Adicionales}
  \begin{itemize}
    \item \href{https://docs.python.org/3/tutorial/classes.html}{\textbf{Python Docs - Classes}} - Sección herencia y polimorfismo.
    \item \href{https://realpython.com/python3-object-oriented-programming/}{\textbf{Real Python - OOP}} - polimorfismo y ejemplos.
    \item \href{https://matplotlib.org/stable/}{\textbf{Matplotlib Docs}} - sección \texttt{imshow}, heatmap, colorbar.
    \item \textbf{Foros y Comunidad}: Stack Overflow, Reddit \texttt{/r/learnpython}.
  \end{itemize}
\end{frame}

% ------------------------------------------------------------------------
% Slide 18: Cierre de la Sesión
% ------------------------------------------------------------------------
\begin{frame}
  \Huge{\centerline{¡Buen trabajo y hasta la próxima sesión!}}
  \vspace{0.3cm}
  \normalsize
  \begin{itemize}
    \item Revisen \textbf{Canvas} para feedback detallado de la Solemne II.
    \item Practiquen con \textbf{POO avanzada} (composición, polimorfismo) en ejemplos propios.
  \end{itemize}
\end{frame}

\end{document}

