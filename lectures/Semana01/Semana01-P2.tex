\documentclass[10pt]{beamer}

% ------------------------------------------------------------------------
% Carga del preámbulo personalizado
% (Asegúrate de contar con preamble.tex en la misma carpeta,
%  donde defines temas, colores, macros como \myfront, etc.)
% ------------------------------------------------------------------------
\usetheme[progressbar=frametitle]{metropolis}
\usepackage{appendixnumberbeamer}
\usepackage{fancyvrb}
\usepackage{booktabs}
\usepackage[scale=2]{ccicons}
\usepackage{pgfplots}
\usepgfplotslibrary{dateplot}
\usepackage{type1cm}
\usepackage{lettrine}
\usepackage{ragged2e}
\usepackage{xspace}
\newcommand{\themename}{\textbf{\textsc{metropolis}}\xspace}
\usepackage{graphicx} % Allows including images
\usepackage{booktabs} % Allows the use of \toprule, \midrule and \bottomrule in tables
\usepackage[utf8]{inputenc} %solucion del problema de los acentos.
\usepackage{xcolor}
\usepackage{verbatim}

\definecolor{LightGray}{gray}{0.9}
% Paleta de azules estilo Python/matplotlib
\definecolor{PythonBlue}{RGB}{31, 119, 180}      % Azul principal matplotlib
\definecolor{LightPythonBlue}{RGB}{174, 199, 232} % Azul claro para fondos
\definecolor{DarkPythonBlue}{RGB}{23, 90, 135}    % Azul oscuro para líneas

% Colores para tema oscuro tipo VS Code/Colab
\definecolor{DarkBackground}{RGB}{40, 44, 52}     % Fondo oscuro similar a VS Code
\definecolor{CodeText}{RGB}{171, 178, 191}        % Texto gris claro para código
\definecolor{DarkFrame}{RGB}{60, 63, 65}          % Color del marco/borde

% Colores para tema claro
\definecolor{LightBackground}{RGB}{248, 248, 242} % Fondo claro tipo GitHub
\definecolor{LightCodeText}{RGB}{51, 51, 51}      % Texto oscuro para tema claro
\definecolor{LightFrame}{RGB}{220, 220, 220}      % Marco gris claro

\usepackage{minted}

%%%%%%%%%%%%%%%%%%%%%%%%%%%%%%%%%%%%%%%%%%%%%%%%%%%%%%%%%%%%%%%%%%%%%%%%%%%%%%%%%%%%%%
% CONFIGURACIÓN DE TEMAS - CAMBIA AQUÍ PARA ALTERNAR
% ========================================================================
% Descomenta UNA de las siguientes líneas para elegir el tema:

% TEMA OSCURO (VS Code style)
%\newcommand{\mytheme}{oscuro}
%\usemintedstyle{monokai}
%\newcommand{\mybgcolor}{DarkBackground}

% TEMA CLARO (GitHub style) - comenta las 3 líneas de arriba y descomenta estas:
 \newcommand{\mytheme}{claro}
 \usemintedstyle{default}
 \newcommand{\mybgcolor}{LightBackground}

% ========================================================================
\newcommand{\mypyfile}[1]{\inputminted[linenos=true, fontsize=\footnotesize, frame=lines, framesep=5\fboxrule,framerule=1pt]{python}{#1}}

% Comando personalizado para código Python inline
\newcommand{\pycode}[1]{\begin{minted}{python}#1\end{minted}}

% ========================================================================
% COMANDOS ESPECÍFICOS PARA CADA TEMA
% ========================================================================
% Comando para tema oscuro específico
\newcommand{\pyoscuro}[1]{%
\begin{minted}[
  bgcolor=DarkBackground,
  style=monokai,
  breaklines,
  frame=lines,
  framesep=2mm,
  baselinestretch=1.2,
  linenos,
  fontsize=\footnotesize
]{python}
#1
\end{minted}
}

% Comando para tema claro específico
\newcommand{\pyclaro}[1]{%
\begin{minted}[
  bgcolor=LightBackground,
  style=default,
  breaklines,
  frame=lines,
  framesep=2mm,
  baselinestretch=1.2,
  linenos,
  fontsize=\footnotesize
]{python}
#1
\end{minted}
}
%%%%%%%%%%%%%%%%%%%%%%%%%%%%%%%%%%%%%%%%%%%%%%%%%%%%%%%%%%%%%%%%%%%%%%%%%%%%%%%%%%%%%%



%%%%%%%%%%%%%%%%%%%%%%%%%%%%%%%%%%%%%%%%%%%%%%%%%%%%%%%%%%%%%%%%%%%%%%%%%%%%%%%%%%%%%%
% Configuración global para todos los bloques minted de Python
% Usa automáticamente el tema seleccionado arriba
\setminted[python]{
  breaklines,
  frame=lines,
  framesep=2mm,
  baselinestretch=1.2,
  bgcolor=\mybgcolor,
  linenos, 
  fontsize=\footnotesize
} % Configuración dinámica según tema elegido
%%%%%%%%%%%%%%%%%%%%%%%%%%%%%%%%%%%%%%%%%%%%%%%%%%%%%%%%%%%%%%%%%%%%%%%%%%%%%%%%%%%%%%



%%%%%%%%%%%%%%%%%%%%%%%%%%%%%%%%%%%%%%%%%%%%%%%%%%%%%%%%%%%%%%%%%%%%%%%%%%%%%%%%%%%%%%
% Configuración de colores del tema con paleta azul Python
\setbeamercolor{progress bar}{fg=DarkPythonBlue,bg=LightPythonBlue}
\setbeamercolor{title separator}{fg=DarkPythonBlue,bg=white!50!black}
\setbeamercolor{frametitle}{fg=white,bg=PythonBlue}
\title[PCFI161]{Programaci\'on para F\'isica y Astronom\'ia}
\subtitle{Departamento de Física.}

\newcommand{\myfront}{
\author[PCFI161]{Corodinadora: C Loyola \\ Profesores C Femenías / F Bugini / D Basantes}
\institute[UNAB]{Universidad Andrés Bello \\ Departamento de Física y Astronomía}
\date{Primer Semestre 2025}
}

\titlegraphic{
  \includegraphics[width=.08\textwidth]{1) Logo/logo-tux.png}\hfill
  \includegraphics[width=.3\textwidth]{1) Logo/logo-unab.png}\hfill
  \includegraphics[width=.08\textwidth]{1) Logo/logo-python.png}
}

\makeatletter
\setbeamertemplate{title page}{
  \begin{minipage}[b][\paperheight]{\textwidth}
    \vfill%
    \ifx\inserttitle\@empty\else\usebeamertemplate*{title}\fi
    \ifx\insertsubtitle\@empty\else\usebeamertemplate*{subtitle}\fi
    \usebeamertemplate*{title separator}
    \ifx\beamer@shortauthor\@empty\else\usebeamertemplate*{author}\fi
    \ifx\insertdate\@empty\else\usebeamertemplate*{date}\fi
    \ifx\insertinstitute\@empty\else\usebeamertemplate*{institute}\fi
    \vfill
    \ifx\inserttitlegraphic\@empty\else\inserttitlegraphic\fi
    \vspace*{1cm}
  \end{minipage}
}
\makeatother

% Configuración personalizada del frametitle para incluir la sección
\makeatletter
\setbeamertemplate{frametitle}{%
  \nointerlineskip%
  \begin{beamercolorbox}[%
      wd=\paperwidth,%
      sep=0pt,%
      leftskip=\@ifundefined{metropolis@frametitle@padding}{12pt}{\metropolis@frametitle@padding},%
      rightskip=\@ifundefined{metropolis@frametitle@padding}{12pt}{\metropolis@frametitle@padding},%
    ]{frametitle}%
  \@ifundefined{metropolis@frametitlestrut@start}{}{\metropolis@frametitlestrut@start}%
  {\textcolor{LightPythonBlue}{\insertsectionhead}\ifx\insertsectionhead\@empty\else\ $\ni$  \fi}\insertframetitle%
  \nolinebreak%
  \@ifundefined{metropolis@frametitlestrut@end}{}{\metropolis@frametitlestrut@end}%
  \end{beamercolorbox}%
}
\makeatother

\makeatletter
% Configuración segura de longitudes del tema Metropolis
\@ifundefined{metropolis@titleseparator@linewidth}{}{\setlength{\metropolis@titleseparator@linewidth}{2pt}}
\@ifundefined{metropolis@progressonsectionpage@linewidth}{}{\setlength{\metropolis@progressonsectionpage@linewidth}{2pt}}
\@ifundefined{metropolis@progressinheadfoot@linewidth}{}{\setlength{\metropolis@progressinheadfoot@linewidth}{2pt}}
\makeatother



\begin{document}

% ------------------------------------------------------------------------
% Portada personalizada. Por ejemplo, usando \myfront si lo tienes en el preámbulo.
% ------------------------------------------------------------------------
\myfront{}

% ------------------------------------------------------------------------
% SLIDE 1: Título de la segunda sesión
% ------------------------------------------------------------------------
\begin{frame}
  \titlepage
  % Puedes ajustar el título/subtítulo para esta sesión específica
  % por ejemplo: \title{Sesión 2 - Semana 1: Ejercicios Iniciales en Colab}
\end{frame}

% ------------------------------------------------------------------------
% SLIDE 2: Índice / Tabla de contenidos
% ------------------------------------------------------------------------
\begin{frame}
  \frametitle{Resumen - Sesión 2 (Semana 1)}
  \tableofcontents
\end{frame}

% ------------------------------------------------------------------------
% Configuración de bloques (en caso de usar metrópolis u otro tema)
% ------------------------------------------------------------------------
\metroset{block=fill}

% ----------------------------------------------------------------------------------------
% SECCIÓN 1: Recapitulación de la Sesión 1
% ----------------------------------------------------------------------------------------
\section{Recapitulación Sesión 1}

% ------------------------------------------------------------------------
% Slide 3: Repaso Rápido
% ------------------------------------------------------------------------
\begin{frame}{Repaso de la Clase Anterior}
  \begin{itemize}
    \item Contexto general del curso y relevancia de la programación en Física/Astronomía.
    \item Familiarización inicial con Google Colab:
      \begin{itemize}
        \item Creación de notebooks.
        \item Ejecución de código básico.
      \end{itemize}
    \item Introducción a los tipos de datos y operaciones simples (asignaciones, suma, resta, etc.).
    \item Primeros ejemplos de entrada y salida (\texttt{input()}, \texttt{print()}).
  \end{itemize}
\end{frame}

% ------------------------------------------------------------------------
% Slide 4: Objetivos de Esta Sesión
% ------------------------------------------------------------------------
\begin{frame}{Objetivos de la Sesión 2}
  \begin{itemize}
    \item \textbf{Practicar} asignaciones simples y operaciones aritméticas en Colab.
    \item \textbf{Explorar} más ejemplos de entrada/salida y la ejecución inmediata de código.
    \item \textbf{Fomentar} la colaboración e intercambio de estrategias entre estudiantes.
    \item \textbf{Resolver} ejercicios que integren los conceptos vistos en la sesión anterior.
  \end{itemize}
\end{frame}

% ----------------------------------------------------------------------------------------
% SECCIÓN 2: Ajustes y Entorno en Colab
% ----------------------------------------------------------------------------------------
\section{Ajustes y Entorno en Colab}

% ------------------------------------------------------------------------
% Slide 5: Organización de Archivos
% ------------------------------------------------------------------------
\begin{frame}{Organización de Archivos y Notebooks}
  \begin{itemize}
    \item Es recomendable mantener una carpeta específica para la asignatura en Google Drive.
    \item Crearemos un notebook llamado \texttt{Sesion2\_Semana1.ipynb} para guardar nuestro trabajo.
    \item \textbf{Tip:} Usa nombres descriptivos para notebooks y subcarpetas (e.g., \emph{ejercicios}, \emph{notas}, \emph{pruebas}).
  \end{itemize}
\end{frame}

% ------------------------------------------------------------------------
% Slide 6: Ejecución Inmediata de Código
% ------------------------------------------------------------------------
\begin{frame}[fragile]{Ejecución Inmediata de Código en Colab}
  \begin{itemize}
    \item Cada celda de un notebook se ejecuta de forma independiente.
    \item Es posible realizar pruebas rápidas sin afectar las demás celdas.
    \item \textbf{Ejemplo de celda interactiva}:
    \begin{minted}[
      frame=lines,
      framesep=2mm,
      baselinestretch=1.2,
      bgcolor=LightGray,
      fontsize=\footnotesize
    ]{python}
# Celda 1
x = 10
print(x)
    \end{minted}

    \begin{minted}[
      frame=lines,
      framesep=2mm,
      baselinestretch=1.2,
      bgcolor=LightGray,
      fontsize=\footnotesize
    ]{python}
# Celda 2
x = x + 5
print(x)  # Mostrará 15
    \end{minted}
  \end{itemize}
\end{frame}

% ----------------------------------------------------------------------------------------
% SECCIÓN 3: Ejercicios de Asignaciones y Operaciones
% ----------------------------------------------------------------------------------------
\section{Ejercicios de Asignaciones y Operaciones}

% ------------------------------------------------------------------------
% Slide 7: Ejercicio 1 - Conversión de Unidades
% ------------------------------------------------------------------------
\begin{frame}{Ejercicio 1: Conversión de Unidades}
  \begin{block}{Enunciado}
    \begin{itemize}
      \item Pide al usuario que introduzca una longitud en \textbf{metros}.
      \item Convierte ese valor a centímetros, milímetros y kilómetros.
      \item Imprime los resultados.
    \end{itemize}
  \end{block}
  \textbf{Objetivo:} Practicar asignaciones simples, multiplicaciones y/o divisiones.
\end{frame}

% ------------------------------------------------------------------------
% Slide 8: Ejercicio 2 - Suma de Dos Variables
% ------------------------------------------------------------------------
\begin{frame}{Ejercicio 2: Suma de Dos Variables}
  \begin{block}{Enunciado}
    \begin{itemize}
      \item Pide al usuario dos números (pueden ser enteros o decimales).
      \item Asigna cada número a una variable distinta (\texttt{a, b}).
      \item Realiza la suma y muestra el resultado.
    \end{itemize}
  \end{block}
  \textbf{Extensión:} Imprime también la resta, el producto y el cociente.
\end{frame}

% ------------------------------------------------------------------------
% Slide 9: Ejercicio 3 - Promedio de Tres Notas
% ------------------------------------------------------------------------
\begin{frame}{Ejercicio 3: Promedio de Tres Notas}
  \begin{block}{Enunciado}
    \begin{itemize}
      \item Solicita tres notas (numeros en \([1.0 - 7.0]\) típicamente).
      \item Calcula el promedio aritmético.
      \item Muestra el resultado con un mensaje apropiado.
    \end{itemize}
  \end{block}
  \textbf{Discusión:}
  \begin{itemize}
    \item ¿Qué pasa si ingresan valores fuera del rango?
    \item El tipo de dato a usar: \texttt{float}.
  \end{itemize}
\end{frame}

% ----------------------------------------------------------------------------------------
% SECCIÓN 4: Resolución Colaborativa
% ----------------------------------------------------------------------------------------
\section{Resolución Colaborativa}

% ------------------------------------------------------------------------
% Slide 10: Grupos de Trabajo
% ------------------------------------------------------------------------
\begin{frame}{Trabajo en Grupos}
  \begin{itemize}
    \item Dividir la clase en \textbf{equipos de 2-3 integrantes}.
    \item Cada equipo crea o comparte un notebook en Colab con sus compañeros.
    \item Se recomienda comentar el código para anotar:
      \begin{itemize}
        \item Qué hace cada línea.
        \item Si surge algún error, cómo se corrigió.
      \end{itemize}
    \item Comparar sus resultados y conclusiones.
  \end{itemize}
\end{frame}

% ------------------------------------------------------------------------
% Slide 11: Puesta en Común
% ------------------------------------------------------------------------
\begin{frame}{Puesta en Común de Dudas y Experiencias}
  \begin{itemize}
    \item Cada equipo expondrá brevemente:
      \begin{itemize}
        \item ¿Qué ejercicio les costó más y por qué?
        \item ¿Cómo resolvieron los problemas encontrados?
        \item ¿Algún atajo o truco que consideren útil?
      \end{itemize}
    \item Se fomenta la retroalimentación colectiva.
    \item \textbf{Tip:} Documentar buenas prácticas que surjan de la discusión.
  \end{itemize}
\end{frame}

% ------------------------------------------------------------------------
% Slide 12: Ejemplo de Solución (Conversión de Unidades)
% ------------------------------------------------------------------------
\begin{frame}[fragile]{Ejemplo de Solución: Conversión de Unidades}
\begin{minted}[
frame=lines,
framesep=2mm,
baselinestretch=1.1,
bgcolor=LightGray,
fontsize=\footnotesize
]{python}
long_m = float(input("Introduce una longitud en metros: "))
cm = long_m * 100
mm = long_m * 1000
km = long_m / 1000

print("En centímetros:", cm, "cm")
print("En milímetros:", mm, "mm")
print("En kilómetros:", km, "km")
\end{minted}
\end{frame}

% ------------------------------------------------------------------------
% Slide 13: Ejemplo de Solución (Suma de Dos Variables)
% ------------------------------------------------------------------------
\begin{frame}[fragile]{Ejemplo de Solución: Suma de Dos Variables}
\begin{minted}[
frame=lines,
framesep=2mm,
baselinestretch=1.1,
bgcolor=LightGray,
fontsize=\footnotesize
]{python}
a_str = input("Ingresa el primer número: ")
b_str = input("Ingresa el segundo número: ")

a = float(a_str)
b = float(b_str)

suma = a + b
resta = a - b
producto = a * b
cociente = a / b  # Cuidar la división por cero

print("Suma =", suma)
print("Resta =", resta)
print("Producto =", producto)
print("Cociente =", cociente)
\end{minted}
\end{frame}

% ------------------------------------------------------------------------
% Slide 14: Ejemplo de Solución (Promedio de Tres Notas)
% ------------------------------------------------------------------------
\begin{frame}[fragile]{Ejemplo de Solución: Promedio de Tres Notas}
\begin{minted}[
frame=lines,
framesep=2mm,
baselinestretch=1.1,
bgcolor=LightGray,
fontsize=\footnotesize
]{python}
n1 = float(input("Nota 1: "))
n2 = float(input("Nota 2: "))
n3 = float(input("Nota 3: "))

promedio = (n1 + n2 + n3) / 3
print("El promedio de las tres notas es:", promedio)
\end{minted}
\textbf{Discusión:} Manejo de rangos y validaciones (opcional).
\end{frame}

% ------------------------------------------------------------------------
% Slide 15: Pequeña Discusión sobre Errores
% ------------------------------------------------------------------------
\begin{frame}{Errores Frecuentes en Python}
  \begin{itemize}
    \item \textbf{ValueError}: ocurre al convertir strings inválidos en \texttt{float} o \texttt{int}.
    \item \textbf{ZeroDivisionError}: cuando \texttt{b = 0} y se hace \texttt{a/b}.
    \item \textbf{NameError}: uso de variables no definidas o mal escritas.
  \end{itemize}
  \textbf{Tip:} Leer atentamente el mensaje de error para identificar la causa y línea afectada.
\end{frame}

% ----------------------------------------------------------------------------------------
% SECCIÓN 5: Actividad Adicional y Retroalimentación
% ----------------------------------------------------------------------------------------
\section{Actividad Adicional}

% ------------------------------------------------------------------------
% Slide 16: Actividad Extra - Escalas Físicas
% ------------------------------------------------------------------------
\begin{frame}{Actividad Extra: Escalas Físicas}
  \begin{block}{Enunciado}
    \begin{itemize}
      \item Pide la temperatura en \(^\circ C\) y conviértela a \(^\circ F\) y K.
      \item Pide la masa en kg y conviértela a libras.
      \item Muestra un pequeño resumen en pantalla con los resultados.
    \end{itemize}
  \end{block}
  \textbf{Objetivo:} Reforzar uso de variables, operaciones aritméticas y \texttt{print}.
\end{frame}

% ------------------------------------------------------------------------
% Slide 17: Mini-Reto - Operaciones con Complejos
% ------------------------------------------------------------------------
\begin{frame}{Mini-Reto: Números Complejos (Opcional)}
  \begin{itemize}
    \item Python maneja complejos con la letra \texttt{j} (\emph{ej}: \texttt{3+2j}).
    \item Investiga cómo sumar, restar y multiplicar números complejos.
    \item \textbf{Ejemplo}:
      \[
        z1 = 3 + 4j, \quad z2 = 2 - 1j
      \]
      \[
        z3 = z1 * z2
      \]
      \(\dots\) Imprime \(\operatorname{Re}(z3)\) y \(\operatorname{Im}(z3)\).
    \item \textbf{Tip}: Usa \texttt{z.real} y \texttt{z.imag} para acceder a sus partes real e imaginaria.
  \end{itemize}
\end{frame}

% ------------------------------------------------------------------------
% Slide 18: Momento de Discusión
% ------------------------------------------------------------------------
\begin{frame}{Discusión Grupales y Dudas}
  \begin{itemize}
    \item ¿Qué soluciones o trucos surgieron durante la actividad extra?
    \item ¿Se presentaron nuevas dudas o errores inesperados?
    \item ¿Qué parte de Python se está volviendo más clara y qué sigue siendo confuso?
  \end{itemize}
\end{frame}

% ------------------------------------------------------------------------
% Slide 19: Retroalimentación y Comentarios
% ------------------------------------------------------------------------
\begin{frame}{Retroalimentación}
  \begin{itemize}
    \item Comparte tu experiencia de aprendizaje con tus compañeros.
    \item \textbf{Ventajas} de Colab: ejecución inmediata, trabajo colaborativo, fácil despliegue de resultados.
    \item \textbf{Desafíos} detectados: conexión a internet, diferencia de versiones, etc.
  \end{itemize}
\end{frame}

% ----------------------------------------------------------------------------------------
% SECCIÓN 6: Conclusiones y Próximos Pasos
% ----------------------------------------------------------------------------------------
\section{Conclusiones}

% ------------------------------------------------------------------------
% Slide 20: Resumen de la Sesión 2
% ------------------------------------------------------------------------
\begin{frame}{Resumen de la Sesión 2}
  \begin{itemize}
    \item Reforzamos las operaciones básicas y la asignación de variables.
    \item Practicamos \textbf{entrada/salida} con varios ejemplos.
    \item Exploramos la \textit{ejecución inmediata} de celdas en Colab y la importancia del orden.
    \item Fomentamos la resolución colaborativa para intercambiar estrategias.
  \end{itemize}
\end{frame}

% ------------------------------------------------------------------------
% Slide 21: Enfoque para la Próxima Sesión
% ------------------------------------------------------------------------
\begin{frame}{Próximos Pasos}
  \begin{itemize}
    \item \textbf{Sesión 3 (Semana 2)}: Introducción a estructuras de control (\texttt{if}, \texttt{while}).
    \item \textbf{Explotaremos} ejemplos físicos básicos (análisis de condiciones, pequeños bucles, etc.).
    \item \textbf{Revisión previa}: Asegúrate de dominar los \textbf{tipos de datos} y la conversión de \texttt{input()} a \texttt{float}.
  \end{itemize}
\end{frame}

% ------------------------------------------------------------------------
% Slide 22: Referencias y Recursos
% ------------------------------------------------------------------------
\begin{frame}{Recursos Recomendados}
  \begin{itemize}
    \item \textbf{Documentación Python:} \href{https://docs.python.org/3/}{docs.python.org/3}
    \item \textbf{Tutoriales en línea:} W3Schools, Real Python.
    \item \textbf{Comunidades:} Stack Overflow, Reddit \texttt{/r/learnpython}.
    \item \textbf{GitHub:} Busca \emph{“intro to python for physics”} para ejemplos.
  \end{itemize}
\end{frame}

% ------------------------------------------------------------------------
% Slide 23: Comentarios Finales
% ------------------------------------------------------------------------
\begin{frame}{Comentarios Finales}
  \begin{itemize}
    \item \textbf{Practicar} es fundamental: domina bien asignaciones y operaciones antes de pasar a estructuras más complejas.
    \item \textbf{Comparte} dudas en foros o con tus compañeros.
    \item \textbf{Recuerda}: Python es sensible a mayúsculas y espacios en la indentación (veremos más en bucles).
  \end{itemize}
\end{frame}

% ------------------------------------------------------------------------
% Slide 24: Invitación a Explorar
% ------------------------------------------------------------------------
\begin{frame}{Invitación a Explorar por tu Cuenta}
  \begin{itemize}
    \item Juega con \textbf{sentencias de asignación} para ver cómo cambiar valores.
    \item Crea \textbf{pequeños scripts} con 2-3 entradas distintas.
    \item Prueba \textbf{operaciones} con números \emph{muy grandes} y \emph{muy pequeños}.
    \item Observa cómo Python maneja la \textit{precisión} numérica.
  \end{itemize}
\end{frame}

% ------------------------------------------------------------------------
% Slide 25: Cierre de la Sesión
% ------------------------------------------------------------------------
\begin{frame}
  \huge{\centerline{¡Gracias y hasta la próxima sesión!}}
  \vspace{0.4cm}
  \normalsize
  \begin{itemize}
    \item Recuerda guardar tus notebooks en Drive.
    \item Si te sobró tiempo, continúa con los mini-retos.
    \item ¡Nos vemos en la Semana 2 con más Python!
  \end{itemize}
\end{frame}

\end{document}

