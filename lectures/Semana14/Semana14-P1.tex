\documentclass[10pt]{beamer}
\usetheme[progressbar=frametitle]{metropolis}
\usepackage{appendixnumberbeamer}
\usepackage{fancyvrb}
\usepackage{booktabs}
\usepackage[scale=2]{ccicons}
\usepackage{pgfplots}
\usepgfplotslibrary{dateplot}
\usepackage{type1cm}
\usepackage{lettrine}
\usepackage{ragged2e}
\usepackage{xspace}
\newcommand{\themename}{\textbf{\textsc{metropolis}}\xspace}
\usepackage{graphicx} % Allows including images
\usepackage{booktabs} % Allows the use of \toprule, \midrule and \bottomrule in tables
\usepackage[utf8]{inputenc} %solucion del problema de los acentos.
\usepackage{xcolor}
\usepackage{verbatim}

\definecolor{LightGray}{gray}{0.9}
% Paleta de azules estilo Python/matplotlib
\definecolor{PythonBlue}{RGB}{31, 119, 180}      % Azul principal matplotlib
\definecolor{LightPythonBlue}{RGB}{174, 199, 232} % Azul claro para fondos
\definecolor{DarkPythonBlue}{RGB}{23, 90, 135}    % Azul oscuro para líneas

% Colores para tema oscuro tipo VS Code/Colab
\definecolor{DarkBackground}{RGB}{40, 44, 52}     % Fondo oscuro similar a VS Code
\definecolor{CodeText}{RGB}{171, 178, 191}        % Texto gris claro para código
\definecolor{DarkFrame}{RGB}{60, 63, 65}          % Color del marco/borde

% Colores para tema claro
\definecolor{LightBackground}{RGB}{248, 248, 242} % Fondo claro tipo GitHub
\definecolor{LightCodeText}{RGB}{51, 51, 51}      % Texto oscuro para tema claro
\definecolor{LightFrame}{RGB}{220, 220, 220}      % Marco gris claro

\usepackage{minted}

%%%%%%%%%%%%%%%%%%%%%%%%%%%%%%%%%%%%%%%%%%%%%%%%%%%%%%%%%%%%%%%%%%%%%%%%%%%%%%%%%%%%%%
% CONFIGURACIÓN DE TEMAS - CAMBIA AQUÍ PARA ALTERNAR
% ========================================================================
% Descomenta UNA de las siguientes líneas para elegir el tema:

% TEMA OSCURO (VS Code style)
%\newcommand{\mytheme}{oscuro}
%\usemintedstyle{monokai}
%\newcommand{\mybgcolor}{DarkBackground}

% TEMA CLARO (GitHub style) - comenta las 3 líneas de arriba y descomenta estas:
 \newcommand{\mytheme}{claro}
 \usemintedstyle{default}
 \newcommand{\mybgcolor}{LightBackground}

% ========================================================================
\newcommand{\mypyfile}[1]{\inputminted[linenos=true, fontsize=\footnotesize, frame=lines, framesep=5\fboxrule,framerule=1pt]{python}{#1}}

% Comando personalizado para código Python inline
\newcommand{\pycode}[1]{\begin{minted}{python}#1\end{minted}}

% ========================================================================
% COMANDOS ESPECÍFICOS PARA CADA TEMA
% ========================================================================
% Comando para tema oscuro específico
\newcommand{\pyoscuro}[1]{%
\begin{minted}[
  bgcolor=DarkBackground,
  style=monokai,
  breaklines,
  frame=lines,
  framesep=2mm,
  baselinestretch=1.2,
  linenos,
  fontsize=\footnotesize
]{python}
#1
\end{minted}
}

% Comando para tema claro específico
\newcommand{\pyclaro}[1]{%
\begin{minted}[
  bgcolor=LightBackground,
  style=default,
  breaklines,
  frame=lines,
  framesep=2mm,
  baselinestretch=1.2,
  linenos,
  fontsize=\footnotesize
]{python}
#1
\end{minted}
}
%%%%%%%%%%%%%%%%%%%%%%%%%%%%%%%%%%%%%%%%%%%%%%%%%%%%%%%%%%%%%%%%%%%%%%%%%%%%%%%%%%%%%%



%%%%%%%%%%%%%%%%%%%%%%%%%%%%%%%%%%%%%%%%%%%%%%%%%%%%%%%%%%%%%%%%%%%%%%%%%%%%%%%%%%%%%%
% Configuración global para todos los bloques minted de Python
% Usa automáticamente el tema seleccionado arriba
\setminted[python]{
  breaklines,
  frame=lines,
  framesep=2mm,
  baselinestretch=1.2,
  bgcolor=\mybgcolor,
  linenos, 
  fontsize=\footnotesize
} % Configuración dinámica según tema elegido
%%%%%%%%%%%%%%%%%%%%%%%%%%%%%%%%%%%%%%%%%%%%%%%%%%%%%%%%%%%%%%%%%%%%%%%%%%%%%%%%%%%%%%



%%%%%%%%%%%%%%%%%%%%%%%%%%%%%%%%%%%%%%%%%%%%%%%%%%%%%%%%%%%%%%%%%%%%%%%%%%%%%%%%%%%%%%
% Configuración de colores del tema con paleta azul Python
\setbeamercolor{progress bar}{fg=DarkPythonBlue,bg=LightPythonBlue}
\setbeamercolor{title separator}{fg=DarkPythonBlue,bg=white!50!black}
\setbeamercolor{frametitle}{fg=white,bg=PythonBlue}
\title[PCFI161]{Programaci\'on para F\'isica y Astronom\'ia}
\subtitle{Departamento de Física.}

\newcommand{\myfront}{
\author[PCFI161]{Corodinadora: C Loyola \\ Profesores C Femenías / F Bugini / D Basantes}
\institute[UNAB]{Universidad Andrés Bello \\ Departamento de Física y Astronomía}
\date{Primer Semestre 2025}
}

\titlegraphic{
  \includegraphics[width=.08\textwidth]{1) Logo/logo-tux.png}\hfill
  \includegraphics[width=.3\textwidth]{1) Logo/logo-unab.png}\hfill
  \includegraphics[width=.08\textwidth]{1) Logo/logo-python.png}
}

\makeatletter
\setbeamertemplate{title page}{
  \begin{minipage}[b][\paperheight]{\textwidth}
    \vfill%
    \ifx\inserttitle\@empty\else\usebeamertemplate*{title}\fi
    \ifx\insertsubtitle\@empty\else\usebeamertemplate*{subtitle}\fi
    \usebeamertemplate*{title separator}
    \ifx\beamer@shortauthor\@empty\else\usebeamertemplate*{author}\fi
    \ifx\insertdate\@empty\else\usebeamertemplate*{date}\fi
    \ifx\insertinstitute\@empty\else\usebeamertemplate*{institute}\fi
    \vfill
    \ifx\inserttitlegraphic\@empty\else\inserttitlegraphic\fi
    \vspace*{1cm}
  \end{minipage}
}
\makeatother

% Configuración personalizada del frametitle para incluir la sección
\makeatletter
\setbeamertemplate{frametitle}{%
  \nointerlineskip%
  \begin{beamercolorbox}[%
      wd=\paperwidth,%
      sep=0pt,%
      leftskip=\@ifundefined{metropolis@frametitle@padding}{12pt}{\metropolis@frametitle@padding},%
      rightskip=\@ifundefined{metropolis@frametitle@padding}{12pt}{\metropolis@frametitle@padding},%
    ]{frametitle}%
  \@ifundefined{metropolis@frametitlestrut@start}{}{\metropolis@frametitlestrut@start}%
  {\textcolor{LightPythonBlue}{\insertsectionhead}\ifx\insertsectionhead\@empty\else\ $\ni$  \fi}\insertframetitle%
  \nolinebreak%
  \@ifundefined{metropolis@frametitlestrut@end}{}{\metropolis@frametitlestrut@end}%
  \end{beamercolorbox}%
}
\makeatother

\makeatletter
% Configuración segura de longitudes del tema Metropolis
\@ifundefined{metropolis@titleseparator@linewidth}{}{\setlength{\metropolis@titleseparator@linewidth}{2pt}}
\@ifundefined{metropolis@progressonsectionpage@linewidth}{}{\setlength{\metropolis@progressonsectionpage@linewidth}{2pt}}
\@ifundefined{metropolis@progressinheadfoot@linewidth}{}{\setlength{\metropolis@progressinheadfoot@linewidth}{2pt}}
\makeatother



\title{Semana 14 – Sesión 1 (Sesión 27):\
Introducción a Ordenamiento y Búsqueda}
\author{PCFI161 – Programación para Física y Astronomía}
\date{02 jun 2025}

\begin{document}
\myfront{}
\begin{frame}\titlepage\end{frame}

\begin{frame}
  \frametitle{Resumen -- Semana 14, Sesión 1}
  \tableofcontents
\end{frame}

\metroset{block=fill}

% -----------------------------------------------------------------
\section{Motivación}
% -----------------------------------------------------------------
\begin{frame}{¿Por qué ordenar y buscar?}
\begin{itemize}
  \item \textbf{Astronomía}: ordenar estrellas por brillo para ver cuál es la más tenue o la más brillante.
  \item \textbf{Física}: organizar mediciones de laboratorio (tiempos de caída, voltajes) antes de graficar.
  \item \textbf{Computación}: muchos algoritmos necesitan datos \emph{ordenados} (p.ej.\ \texttt{binary search}). 
\end{itemize}
\end{frame}

% -----------------------------------------------------------------
\section{Ordenamiento básico}
% -----------------------------------------------------------------
\begin{frame}[fragile]{Bubble Sort paso a paso}
\begin{minted}{python}
def bubble_sort(lista):
    n = len(lista)
    for i in range(n-1):            # pasadas
        for j in range(n-1-i):      # pares a comparar
            if lista[j] > lista[j+1]:
                lista[j], lista[j+1] = lista[j+1], lista[j]
    return lista
\end{minted}
\begin{itemize}
  \item Compara \textbf{pares adyacentes} y “burbujea” el mayor al final.  
  \item Complejidad: \(\mathcal{O}(n^{2})\).  
  \item Fácil de codificar, ideal para entender intercambios.  \footnote{Ver animaciones interactivas: HackerEarth :contentReference[oaicite:0]{index=0}}
\end{itemize}
\end{frame}

\begin{frame}[fragile]{Ejemplo -- Ordenar distancias planetarias}
\begin{minted}{python}
dist_au = [1.52, 0.39, 5.20, 1.00, 30.07]   # Marte, Mercurio, Júpiter, Tierra, Neptuno
print("Original :", dist_au)
print("Ordenado  :", bubble_sort(dist_au))
\end{minted}
\pause
Resultado esperado: \([0.39, 1.00, 1.52, 5.20, 30.07]\)
\end{frame}

% -----------------------------------------------------------------
\section{Búsqueda}
% -----------------------------------------------------------------
\begin{frame}[fragile]{Búsqueda lineal}
\begin{minted}{python}
def linear_search(lista, objetivo):
    for i, x in enumerate(lista):
        if x == objetivo:
            return i
    return -1   # no está
\end{minted}
\end{frame}

\begin{frame}[fragile]{Búsqueda binaria (en lista ordenada)}
\begin{minted}{python}
def binary_search(lista_ordenada, objetivo):
    low, high = 0, len(lista_ordenada)-1
    while low <= high:
        mid = (low + high) // 2
        if lista_ordenada[mid] == objetivo:
            return mid
        elif objetivo < lista_ordenada[mid]:
            high = mid - 1
        else:
            low = mid + 1
    return -1
\end{minted}
\begin{itemize}
  \item Reduce el rango a la mitad cada vez $\longrightarrow$ \(\mathcal{O}(\log n)\).  
  \item Sólo funciona si la lista está \textbf{ordenada}.  
  \item Explicación amigable: Khan Academy.
\end{itemize}
\end{frame}

% -----------------------------------------------------------------
\section{Ejercicios en clase}
% -----------------------------------------------------------------

% ---------- Slide A: Generar / cargar temperaturas ----------------
\begin{frame}[fragile]{Ejercicio 1 – Preparar los datos de temperatura}
  \begin{block}{Paso 1 – Obtener la lista \texttt{T\_C}}
  \begin{minted}{python}
import pandas as pd
import numpy as np

np.random.seed(42)
temps = np.random.normal(loc=22, scale=2, size=5_000).round(2).tolist()
  \end{minted}
  \vspace{-0.4em}
  \small
   Se crean 5000 \textit{mediciones} con media 22$^\circ$C.
  \end{block}
\end{frame}

% ---------- Slide B: Implementar Bubble Sort ----------------------
\begin{frame}[fragile]{Ejercicio 1 -- Ordenar y medir rendimiento}
\vspace{0.3cm}
\begin{columns}
\column{0.48\textwidth}
\vspace{-0.2cm}
{\scriptsize\textbf{Tu burbuja}}
\begin{minted}{python}
def bubble_sort(lst):
    n = len(lst)
    ....
\end{minted}

\column{0.48\textwidth}
\vspace{-0.2cm}
{\scriptsize\textbf{Versión NumPy}}
\begin{minted}{python}
import numpy as np
arr = np.array(temps)
arr_sorted = np.sort(arr) 
\end{minted}
\end{columns}

\vspace{0.3em}
{\footnotesize
\begin{block}{Paso 2 – Comparar tiempos con \%timeit (Jupyter/Colab)}
\begin{minted}{python}
%%timeit -n 3 -r 3
bubble_sort(temps.copy())    # tu implementación

%%timeit -n 3 -r 3
np.sort(arr)                 # NumPy optimizado
\end{minted}
\end{block}
}

{\scriptsize
Las opciones \texttt{-n} y \texttt{-r} hacen que se ejecute el código 3 veces (n), luego repite eso 3 veces (r) para tener un total de 9 ejecuciones. Luego calcula la media y desviación estándar de esas 3 repeticiones.
Después de ordenar:  
\texttt{minT = sorted\_list[0]}, \texttt{maxT = sorted\_list[-1]},  
\texttt{mediana = (sorted\_list[n//2] + sorted\_list[(n-1)//2]) / 2}
}
\end{frame}

% ---------- Slide C1: Preparar datos simulados ----------------------
\begin{frame}[fragile]{Ejercicio 2 - Búsqueda Binaria}
  \begin{block}{Simulamos 300 magnitudes con distribución normal}
  \begin{minted}{python}
import numpy as np
mags = np.random.normal(8, 1.5, size=300).round(2)  # Simulación
  \end{minted}
  \end{block}

  \begin{block}{Ordenamos la lista (requisito para búsqueda binaria)}
  \begin{minted}{python}
mags_ordenadas = sorted(mags)
  \end{minted}
  \end{block}

  \begin{itemize}
    \item `np.random.normal` genera datos aleatorios tipo Gaussianos.
    \item `round(2)` acorta los decimales.
    \item `sorted()` es obligatorio: la búsqueda binaria sólo funciona en listas ordenadas.
  \end{itemize}
\end{frame}


% ---------- Slide C2: Definición de la función binary_search ----------------------
\begin{frame}[fragile]{Paso 2 – Definición de la función de búsqueda binaria}
  \begin{block}{Función con contador de comparaciones}
  \begin{minted}{python}
def binary_search(lst, target):
    low, high, checks = 0, len(lst)-1, 0
    while low <= high:
        mid = (low + high) // 2
        checks += 1
        if lst[mid] == target:
            return mid, checks
        elif target < lst[mid]:
            high = mid - 1
        else:
            low = mid + 1
    return -1, checks
  \end{minted}
  \end{block}

  \begin{itemize}
    \item Divide la lista a la mitad en cada paso.
    \item `checks` cuenta cuántas comparaciones hicimos.
    \item Retorna la posición si se encuentra, y `-1` si no.
  \end{itemize}
\end{frame}


% ---------- Slide C3: Ejecución y análisis ----------------------
\begin{frame}[fragile]{Paso 3 – Ejecución y comparación}
  \begin{block}{Buscar valor ingresado por el usuario}
  \begin{minted}{python}
val = float(input("Ingresa magnitud a buscar: "))
pos, nchecks = binary_search(mags_ordenadas, val)
if pos != -1:
    print(f"Encontrado en índice {pos} tras {nchecks} comparaciones.")
else:
    print(f"No aparece (se hicieron {nchecks} comparaciones).")
  \end{minted}
  \end{block}

  \begin{itemize}
    \item La búsqueda binaria es mucho más rápida que la búsqueda lineal.
    \item Búsqueda lineal: hasta 300 comparaciones.
    \item Búsqueda binaria: a lo más \( \log_2(300) \approx 9 \) comparaciones.
  \end{itemize}
\end{frame}





% ---------- Slide D: Guía para comparar ---------------------------
\begin{frame}[fragile]{Cómo evaluar tu búsqueda}
\begin{itemize}
  \item Añade un \textbf{contador global} en la función para registrar comparaciones.  
  \item Realiza la misma búsqueda con un \emph{recorrido lineal} y compara ambos contadores.  
  \item Usa \texttt{\%timeit} para medir el tiempo de la búsqueda lineal vs.\ binaria.  
\end{itemize}

\vspace{0.3cm}
\begin{minted}{python}
# búsqueda lineal (baseline)
def linear_search(lst, target):
    for checks, x in enumerate(lst, 1):
        if x == target:
            return checks
    return checks  # no encontrado

%timeit binary_search(mags_ordenadas, val)
%timeit linear_search(mags_ordenadas, val)
\end{minted}
\end{frame}

% ---------- Slide E: Cierre --------------------------------------
\begin{frame}{Cierre}
\begin{itemize}
  \item Resumen: \textbf{bubble sort} (sencillo, lento) vs.\ \texttt{np.sort} (rápido).  
  \item \textbf{Binary search} ahorra comparaciones en listas ordenadas.  
  \item \textbf{Practica} \texttt{\%timeit} con distintas longitudes de lista para ver la
        diferencia de escala.  
\end{itemize}
\end{frame}


\end{document}
