\documentclass[10pt]{beamer}
\usetheme[progressbar=frametitle]{metropolis}
\usepackage{appendixnumberbeamer}
\usepackage{fancyvrb}
\usepackage{booktabs}
\usepackage[scale=2]{ccicons}
\usepackage{pgfplots}
\usepgfplotslibrary{dateplot}
\usepackage{type1cm}
\usepackage{lettrine}
\usepackage{ragged2e}
\usepackage{xspace}
\newcommand{\themename}{\textbf{\textsc{metropolis}}\xspace}
\usepackage{graphicx} % Allows including images
\usepackage{booktabs} % Allows the use of \toprule, \midrule and \bottomrule in tables
\usepackage[utf8]{inputenc} %solucion del problema de los acentos.
\usepackage{xcolor}
\usepackage{verbatim}

\definecolor{LightGray}{gray}{0.9}
% Paleta de azules estilo Python/matplotlib
\definecolor{PythonBlue}{RGB}{31, 119, 180}      % Azul principal matplotlib
\definecolor{LightPythonBlue}{RGB}{174, 199, 232} % Azul claro para fondos
\definecolor{DarkPythonBlue}{RGB}{23, 90, 135}    % Azul oscuro para líneas

% Colores para tema oscuro tipo VS Code/Colab
\definecolor{DarkBackground}{RGB}{40, 44, 52}     % Fondo oscuro similar a VS Code
\definecolor{CodeText}{RGB}{171, 178, 191}        % Texto gris claro para código
\definecolor{DarkFrame}{RGB}{60, 63, 65}          % Color del marco/borde

% Colores para tema claro
\definecolor{LightBackground}{RGB}{248, 248, 242} % Fondo claro tipo GitHub
\definecolor{LightCodeText}{RGB}{51, 51, 51}      % Texto oscuro para tema claro
\definecolor{LightFrame}{RGB}{220, 220, 220}      % Marco gris claro

\usepackage{minted}

%%%%%%%%%%%%%%%%%%%%%%%%%%%%%%%%%%%%%%%%%%%%%%%%%%%%%%%%%%%%%%%%%%%%%%%%%%%%%%%%%%%%%%
% CONFIGURACIÓN DE TEMAS - CAMBIA AQUÍ PARA ALTERNAR
% ========================================================================
% Descomenta UNA de las siguientes líneas para elegir el tema:

% TEMA OSCURO (VS Code style)
%\newcommand{\mytheme}{oscuro}
%\usemintedstyle{monokai}
%\newcommand{\mybgcolor}{DarkBackground}

% TEMA CLARO (GitHub style) - comenta las 3 líneas de arriba y descomenta estas:
 \newcommand{\mytheme}{claro}
 \usemintedstyle{default}
 \newcommand{\mybgcolor}{LightBackground}

% ========================================================================
\newcommand{\mypyfile}[1]{\inputminted[linenos=true, fontsize=\footnotesize, frame=lines, framesep=5\fboxrule,framerule=1pt]{python}{#1}}

% Comando personalizado para código Python inline
\newcommand{\pycode}[1]{\begin{minted}{python}#1\end{minted}}

% ========================================================================
% COMANDOS ESPECÍFICOS PARA CADA TEMA
% ========================================================================
% Comando para tema oscuro específico
\newcommand{\pyoscuro}[1]{%
\begin{minted}[
  bgcolor=DarkBackground,
  style=monokai,
  breaklines,
  frame=lines,
  framesep=2mm,
  baselinestretch=1.2,
  linenos,
  fontsize=\footnotesize
]{python}
#1
\end{minted}
}

% Comando para tema claro específico
\newcommand{\pyclaro}[1]{%
\begin{minted}[
  bgcolor=LightBackground,
  style=default,
  breaklines,
  frame=lines,
  framesep=2mm,
  baselinestretch=1.2,
  linenos,
  fontsize=\footnotesize
]{python}
#1
\end{minted}
}
%%%%%%%%%%%%%%%%%%%%%%%%%%%%%%%%%%%%%%%%%%%%%%%%%%%%%%%%%%%%%%%%%%%%%%%%%%%%%%%%%%%%%%



%%%%%%%%%%%%%%%%%%%%%%%%%%%%%%%%%%%%%%%%%%%%%%%%%%%%%%%%%%%%%%%%%%%%%%%%%%%%%%%%%%%%%%
% Configuración global para todos los bloques minted de Python
% Usa automáticamente el tema seleccionado arriba
\setminted[python]{
  breaklines,
  frame=lines,
  framesep=2mm,
  baselinestretch=1.2,
  bgcolor=\mybgcolor,
  linenos, 
  fontsize=\footnotesize
} % Configuración dinámica según tema elegido
%%%%%%%%%%%%%%%%%%%%%%%%%%%%%%%%%%%%%%%%%%%%%%%%%%%%%%%%%%%%%%%%%%%%%%%%%%%%%%%%%%%%%%



%%%%%%%%%%%%%%%%%%%%%%%%%%%%%%%%%%%%%%%%%%%%%%%%%%%%%%%%%%%%%%%%%%%%%%%%%%%%%%%%%%%%%%
% Configuración de colores del tema con paleta azul Python
\setbeamercolor{progress bar}{fg=DarkPythonBlue,bg=LightPythonBlue}
\setbeamercolor{title separator}{fg=DarkPythonBlue,bg=white!50!black}
\setbeamercolor{frametitle}{fg=white,bg=PythonBlue}
\title[PCFI161]{Programaci\'on para F\'isica y Astronom\'ia}
\subtitle{Departamento de Física.}

\newcommand{\myfront}{
\author[PCFI161]{Corodinadora: C Loyola \\ Profesores C Femenías / F Bugini / D Basantes}
\institute[UNAB]{Universidad Andrés Bello \\ Departamento de Física y Astronomía}
\date{Primer Semestre 2025}
}

\titlegraphic{
  \includegraphics[width=.08\textwidth]{1) Logo/logo-tux.png}\hfill
  \includegraphics[width=.3\textwidth]{1) Logo/logo-unab.png}\hfill
  \includegraphics[width=.08\textwidth]{1) Logo/logo-python.png}
}

\makeatletter
\setbeamertemplate{title page}{
  \begin{minipage}[b][\paperheight]{\textwidth}
    \vfill%
    \ifx\inserttitle\@empty\else\usebeamertemplate*{title}\fi
    \ifx\insertsubtitle\@empty\else\usebeamertemplate*{subtitle}\fi
    \usebeamertemplate*{title separator}
    \ifx\beamer@shortauthor\@empty\else\usebeamertemplate*{author}\fi
    \ifx\insertdate\@empty\else\usebeamertemplate*{date}\fi
    \ifx\insertinstitute\@empty\else\usebeamertemplate*{institute}\fi
    \vfill
    \ifx\inserttitlegraphic\@empty\else\inserttitlegraphic\fi
    \vspace*{1cm}
  \end{minipage}
}
\makeatother

% Configuración personalizada del frametitle para incluir la sección
\makeatletter
\setbeamertemplate{frametitle}{%
  \nointerlineskip%
  \begin{beamercolorbox}[%
      wd=\paperwidth,%
      sep=0pt,%
      leftskip=\@ifundefined{metropolis@frametitle@padding}{12pt}{\metropolis@frametitle@padding},%
      rightskip=\@ifundefined{metropolis@frametitle@padding}{12pt}{\metropolis@frametitle@padding},%
    ]{frametitle}%
  \@ifundefined{metropolis@frametitlestrut@start}{}{\metropolis@frametitlestrut@start}%
  {\textcolor{LightPythonBlue}{\insertsectionhead}\ifx\insertsectionhead\@empty\else\ $\ni$  \fi}\insertframetitle%
  \nolinebreak%
  \@ifundefined{metropolis@frametitlestrut@end}{}{\metropolis@frametitlestrut@end}%
  \end{beamercolorbox}%
}
\makeatother

\makeatletter
% Configuración segura de longitudes del tema Metropolis
\@ifundefined{metropolis@titleseparator@linewidth}{}{\setlength{\metropolis@titleseparator@linewidth}{2pt}}
\@ifundefined{metropolis@progressonsectionpage@linewidth}{}{\setlength{\metropolis@progressonsectionpage@linewidth}{2pt}}
\@ifundefined{metropolis@progressinheadfoot@linewidth}{}{\setlength{\metropolis@progressinheadfoot@linewidth}{2pt}}
\makeatother



\title{Semana 14 – Sesión 2 (Sesión 28):\\
De la teoría a la práctica – Merge Sort, Benchmarks y Aplicaciones}
\author{PCFI161 – Programación para Física y Astronomía}
\date{02 jun 2025}

\begin{document}
\myfront{}
\begin{frame}\titlepage\end{frame}

\begin{frame}
  \frametitle{Resumen -- Semana 14, Sesión 2}
  \tableofcontents
\end{frame}

\metroset{block=fill}

% -----------------------------------------------------------------
\section{Recap rápido}
% -----------------------------------------------------------------
\begin{frame}{Lo aprendido en la Sesión 1}
\begin{itemize}
  \item \textbf{Bubble Sort}: claro pero ineficiente \(\mathcal{O}(n^{2})\).
  \item \textbf{Binary Search}: súper-rápida, pero sólo en listas ordenadas.
  \item \texttt{\%\%timeit}: herramienta para medir micro-rendimiento en Colab.
\end{itemize}
\alert{Pregunta relámpago}: ¿cuál es la complejidad de buscar linealmente en una lista?
\end{frame}

% -----------------------------------------------------------------
\section{Recursividad \& Merge Sort}
% -----------------------------------------------------------------
\begin{frame}[fragile]{¿Qué es recursividad?}
\begin{itemize}
  \item Una función que se llama a sí misma para resolver un problema más pequeño.
  \item Necesita: \textit{caso base} + \textit{llamada recursiva}.
  \item Ejemplo clásico: factorial, Fibonacci… ¡y algoritmos de ordenamiento!
\end{itemize}
\end{frame}

\begin{frame}[fragile]{Merge Sort en Python (\(\mathcal{O}(n\log n)\))}
\begin{minted}[fontsize=\scriptsize]{python}
def merge_sort(lst):
    if len(lst) <= 1:
        return lst                    # caso base
    mid = len(lst) // 2
    left  = merge_sort(lst[:mid])     # divide
    right = merge_sort(lst[mid:])
    return merge(left, right)         # conquista

def merge(left, right):
    out, i, j = [], 0, 0
    while i < len(left) and j < len(right):
        if left[i] < right[j]:
            out.append(left[i]); i += 1
        else:
            out.append(right[j]); j += 1
    return out + left[i:] + right[j:]
\end{minted}
\begin{itemize}
  \item Dos fases: división recursiva y mezcla ordenada.
  \item Estable (\alert{preserva orden relativas de iguales}) y predecible.
\end{itemize}
\end{frame}

% -----------------------------------------------------------------
\section{Benchmark y visualización}
% -----------------------------------------------------------------
\begin{frame}[fragile]{Midiendo rendimientos}
\begin{columns}
\column{0.65\textwidth}
\begin{minted}[fontsize=\tiny]{python}
import numpy as np, timeit, pandas as pd

# --- función de benchmark ------------------------------------------
def timing(alg, n, rep=3):
    base = np.random.randint(0, 10_000, n).tolist()
    return timeit.timeit(lambda: alg(base.copy()), number=rep) / rep

# --- medir ---------------------------------------------------------
Ns   = [2**k for k in range(8, 15)]           # 256 … 16384
algs = {"bubble": bubble_sort,
        "merge":  merge_sort,
        "numpy":  lambda x: np.sort(x).tolist()}

df = pd.DataFrame({name: [timing(f, n) for n in Ns]
                   for name, f in algs.items()},
                  index=Ns)

print(df)
\end{minted}
\column{0.35\textwidth}
\small
\begin{itemize}
  \item Función \texttt{timing} promedia \texttt{rep} ejecuciones.
  \item Guardamos resultados en un \texttt{DataFrame} $\longrightarrow$ ideal para graficar.
\end{itemize}
\end{columns}
\end{frame}

\begin{frame}[fragile]{Gráfica \texttt{bubble} vs \texttt{merge} vs \texttt{np.sort}}
\begin{minted}{python}
import matplotlib.pyplot as plt

df.plot(marker="o")
plt.loglog()                       # ejes log-log
plt.xlabel("Tamaño de la lista (n)")
plt.ylabel("Tiempo [s]")
plt.title("Escalamiento temporal de algoritmos de ordenamiento")
plt.legend(title="Algoritmo")
plt.show()
\end{minted}
\vspace{-0.8em}
\begin{itemize}
  \item La pendiente ~2 de bubble confirma su \(\mathcal{O}(n^{2})\).
  \item Merge y \texttt{np.sort} (Timsort) se alinean con \(n\log n\).
\end{itemize}
\end{frame}

% -----------------------------------------------------------------
\section{Caso de aplicación con \texttt{pandas}}
% -----------------------------------------------------------------
\begin{frame}[fragile]{Mini-dataset astronómico}
\begin{minted}{python}
import pandas as pd, numpy as np
np.random.seed(0)
df = pd.DataFrame({
    "nombre":  [f"Star-{i}" for i in range(5000)],
    "mag":     np.random.normal(8, 1.2, 5000).round(2),
    "dist_pc": np.random.exponential(20, 5000).round(1)
})
\end{minted}
\begin{itemize}
  \item Cada grupo puede descargar un catálogo real (\texttt{exoplanet.eu}) o usar este simulado.
  \item Tareas guiadas:
    \begin{enumerate}
      \item Ordenar por magnitud aparente y hallar los 10 más brillantes.
      \item Usar \texttt{binary\_search} para ubicar rápidamente una estrella con magnitud ≈ 9.3 en la lista ordenada.
    \end{enumerate}
\end{itemize}
\end{frame}

% -----------------------------------------------------------------
\section{Trabajo en sala}
% -----------------------------------------------------------------
\begin{frame}[fragile]{Actividad práctica grupal}
\begin{block}{Objetivo}
Comparar empíricamente \(\mathcal{O}(n^{2})\) y \(\mathcal{O}(n\log n)\) 
y reflexionar sobre cuándo vale la pena optimizar.
\end{block}

\textbf{Pasos sugeridos}
\begin{enumerate}
  \item Implementar \texttt{merge\_sort} (o \texttt{quick\_sort}) desde cero.
  \item Medir tiempos con \texttt{\%\%timeit} para $n=2^{10},2^{12},2^{14}$.
  \item Graficar tiempo vs.~$n$ y anotar pendiente aproximada.
  \item Responder en el notebook:  
  ¿Para qué tamaños de $n$ bubble sort sigue siendo aceptable?
\end{enumerate}

\textbf{Aprevechemos de ejercitar y aclarar dudas. Hay Tarea la próxima semana.}
\end{frame}

% -----------------------------------------------------------------
\section{Cierre}
% -----------------------------------------------------------------
\begin{frame}[fragile]{Conclusiones}
\begin{itemize}
  \item Recursividad introduce \emph{divide \& conquer}: potencia
        evidente en merge sort.  
  \item Visualizar tiempos ayuda a internalizar las escalas de
        complejidad.  
  \item La próxima semana veremos cómo \emph{perfilar} código y
        optimizar “cuellos de botella” (\textbf{Semana 15}).  
\end{itemize}
\centering
\Large ¡Nos vemos en el próximo laboratorio!
\end{frame}

\end{document}
