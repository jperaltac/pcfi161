\documentclass[10pt]{beamer}

% Copia local de preamble.tex:
\usetheme[progressbar=frametitle]{metropolis}
\usepackage{appendixnumberbeamer}
\usepackage{fancyvrb}
\usepackage{booktabs}
\usepackage[scale=2]{ccicons}
\usepackage{pgfplots}
\usepgfplotslibrary{dateplot}
\usepackage{type1cm}
\usepackage{lettrine}
\usepackage{ragged2e}
\usepackage{xspace}
\newcommand{\themename}{\textbf{\textsc{metropolis}}\xspace}
\usepackage{graphicx} % Allows including images
\usepackage{booktabs} % Allows the use of \toprule, \midrule and \bottomrule in tables
\usepackage[utf8]{inputenc} %solucion del problema de los acentos.
\usepackage{xcolor}
\usepackage{verbatim}

\definecolor{LightGray}{gray}{0.9}
% Paleta de azules estilo Python/matplotlib
\definecolor{PythonBlue}{RGB}{31, 119, 180}      % Azul principal matplotlib
\definecolor{LightPythonBlue}{RGB}{174, 199, 232} % Azul claro para fondos
\definecolor{DarkPythonBlue}{RGB}{23, 90, 135}    % Azul oscuro para líneas

% Colores para tema oscuro tipo VS Code/Colab
\definecolor{DarkBackground}{RGB}{40, 44, 52}     % Fondo oscuro similar a VS Code
\definecolor{CodeText}{RGB}{171, 178, 191}        % Texto gris claro para código
\definecolor{DarkFrame}{RGB}{60, 63, 65}          % Color del marco/borde

% Colores para tema claro
\definecolor{LightBackground}{RGB}{248, 248, 242} % Fondo claro tipo GitHub
\definecolor{LightCodeText}{RGB}{51, 51, 51}      % Texto oscuro para tema claro
\definecolor{LightFrame}{RGB}{220, 220, 220}      % Marco gris claro

\usepackage{minted}

%%%%%%%%%%%%%%%%%%%%%%%%%%%%%%%%%%%%%%%%%%%%%%%%%%%%%%%%%%%%%%%%%%%%%%%%%%%%%%%%%%%%%%
% CONFIGURACIÓN DE TEMAS - CAMBIA AQUÍ PARA ALTERNAR
% ========================================================================
% Descomenta UNA de las siguientes líneas para elegir el tema:

% TEMA OSCURO (VS Code style)
%\newcommand{\mytheme}{oscuro}
%\usemintedstyle{monokai}
%\newcommand{\mybgcolor}{DarkBackground}

% TEMA CLARO (GitHub style) - comenta las 3 líneas de arriba y descomenta estas:
 \newcommand{\mytheme}{claro}
 \usemintedstyle{default}
 \newcommand{\mybgcolor}{LightBackground}

% ========================================================================
\newcommand{\mypyfile}[1]{\inputminted[linenos=true, fontsize=\footnotesize, frame=lines, framesep=5\fboxrule,framerule=1pt]{python}{#1}}

% Comando personalizado para código Python inline
\newcommand{\pycode}[1]{\begin{minted}{python}#1\end{minted}}

% ========================================================================
% COMANDOS ESPECÍFICOS PARA CADA TEMA
% ========================================================================
% Comando para tema oscuro específico
\newcommand{\pyoscuro}[1]{%
\begin{minted}[
  bgcolor=DarkBackground,
  style=monokai,
  breaklines,
  frame=lines,
  framesep=2mm,
  baselinestretch=1.2,
  linenos,
  fontsize=\footnotesize
]{python}
#1
\end{minted}
}

% Comando para tema claro específico
\newcommand{\pyclaro}[1]{%
\begin{minted}[
  bgcolor=LightBackground,
  style=default,
  breaklines,
  frame=lines,
  framesep=2mm,
  baselinestretch=1.2,
  linenos,
  fontsize=\footnotesize
]{python}
#1
\end{minted}
}
%%%%%%%%%%%%%%%%%%%%%%%%%%%%%%%%%%%%%%%%%%%%%%%%%%%%%%%%%%%%%%%%%%%%%%%%%%%%%%%%%%%%%%



%%%%%%%%%%%%%%%%%%%%%%%%%%%%%%%%%%%%%%%%%%%%%%%%%%%%%%%%%%%%%%%%%%%%%%%%%%%%%%%%%%%%%%
% Configuración global para todos los bloques minted de Python
% Usa automáticamente el tema seleccionado arriba
\setminted[python]{
  breaklines,
  frame=lines,
  framesep=2mm,
  baselinestretch=1.2,
  bgcolor=\mybgcolor,
  linenos, 
  fontsize=\footnotesize
} % Configuración dinámica según tema elegido
%%%%%%%%%%%%%%%%%%%%%%%%%%%%%%%%%%%%%%%%%%%%%%%%%%%%%%%%%%%%%%%%%%%%%%%%%%%%%%%%%%%%%%



%%%%%%%%%%%%%%%%%%%%%%%%%%%%%%%%%%%%%%%%%%%%%%%%%%%%%%%%%%%%%%%%%%%%%%%%%%%%%%%%%%%%%%
% Configuración de colores del tema con paleta azul Python
\setbeamercolor{progress bar}{fg=DarkPythonBlue,bg=LightPythonBlue}
\setbeamercolor{title separator}{fg=DarkPythonBlue,bg=white!50!black}
\setbeamercolor{frametitle}{fg=white,bg=PythonBlue}
\title[PCFI161]{Programaci\'on para F\'isica y Astronom\'ia}
\subtitle{Departamento de Física.}

\newcommand{\myfront}{
\author[PCFI161]{Corodinadora: C Loyola \\ Profesores C Femenías / F Bugini / D Basantes}
\institute[UNAB]{Universidad Andrés Bello \\ Departamento de Física y Astronomía}
\date{Primer Semestre 2025}
}

\titlegraphic{
  \includegraphics[width=.08\textwidth]{1) Logo/logo-tux.png}\hfill
  \includegraphics[width=.3\textwidth]{1) Logo/logo-unab.png}\hfill
  \includegraphics[width=.08\textwidth]{1) Logo/logo-python.png}
}

\makeatletter
\setbeamertemplate{title page}{
  \begin{minipage}[b][\paperheight]{\textwidth}
    \vfill%
    \ifx\inserttitle\@empty\else\usebeamertemplate*{title}\fi
    \ifx\insertsubtitle\@empty\else\usebeamertemplate*{subtitle}\fi
    \usebeamertemplate*{title separator}
    \ifx\beamer@shortauthor\@empty\else\usebeamertemplate*{author}\fi
    \ifx\insertdate\@empty\else\usebeamertemplate*{date}\fi
    \ifx\insertinstitute\@empty\else\usebeamertemplate*{institute}\fi
    \vfill
    \ifx\inserttitlegraphic\@empty\else\inserttitlegraphic\fi
    \vspace*{1cm}
  \end{minipage}
}
\makeatother

% Configuración personalizada del frametitle para incluir la sección
\makeatletter
\setbeamertemplate{frametitle}{%
  \nointerlineskip%
  \begin{beamercolorbox}[%
      wd=\paperwidth,%
      sep=0pt,%
      leftskip=\@ifundefined{metropolis@frametitle@padding}{12pt}{\metropolis@frametitle@padding},%
      rightskip=\@ifundefined{metropolis@frametitle@padding}{12pt}{\metropolis@frametitle@padding},%
    ]{frametitle}%
  \@ifundefined{metropolis@frametitlestrut@start}{}{\metropolis@frametitlestrut@start}%
  {\textcolor{LightPythonBlue}{\insertsectionhead}\ifx\insertsectionhead\@empty\else\ $\ni$  \fi}\insertframetitle%
  \nolinebreak%
  \@ifundefined{metropolis@frametitlestrut@end}{}{\metropolis@frametitlestrut@end}%
  \end{beamercolorbox}%
}
\makeatother

\makeatletter
% Configuración segura de longitudes del tema Metropolis
\@ifundefined{metropolis@titleseparator@linewidth}{}{\setlength{\metropolis@titleseparator@linewidth}{2pt}}
\@ifundefined{metropolis@progressonsectionpage@linewidth}{}{\setlength{\metropolis@progressonsectionpage@linewidth}{2pt}}
\@ifundefined{metropolis@progressinheadfoot@linewidth}{}{\setlength{\metropolis@progressinheadfoot@linewidth}{2pt}}
\makeatother



\begin{document}

\myfront{}

\begin{frame}
  \titlepage
\end{frame}

\begin{frame}
  \frametitle{Resumen - Parte 2 (Semana 03)}
  \tableofcontents
\end{frame}

%----------------------------------------------------------------------------------------
%	PRESENTATION SLIDES
%----------------------------------------------------------------------------------------
\metroset{block=fill}

%------------------------------------------------
\section{Estructuras de control}

\begin{frame}{Estructuras de control}
Hasta ahora, hemos visto programas \textbf{lineales}: se ejecutan de arriba abajo de forma secuencial.  
Para hacerlos más flexibles y tomar decisiones (o repetir bloques de código), Python ofrece \textit{estructuras de control}:
\begin{itemize}
    \item \textbf{if, else, elif} (condicionales)
    \item \textbf{while} (bucles)
    \item Más adelante: \texttt{for}, \texttt{try-except}, etc.
\end{itemize}
\end{frame}

\subsection{Declaración IF}
\begin{frame}[fragile]{La declaración \texttt{IF}}
\begin{minted}{python}
x = int(input("Ingrese un entero no mayor a diez:"))
if x > 10:
  print("Usted ingreso un numero mayor que diez.")
  print("Permítame corregirle.")
  x = 10
print("Su número es", x)
\end{minted}

\textbf{Nota}: Observa la indentación. Todo lo que esté “sangrado” debajo de \texttt{if} será el bloque que se ejecuta si la condición es verdadera.
\end{frame}

\begin{frame}[fragile]{Comparaciones}
\footnotesize
\begin{itemize}
    \item \texttt{x == 1}  compara si \(x\) es igual a 1.
    \item \texttt{x != 1}  compara si \(x\) es distinto de 1.
    \item \texttt{x > 1}, \texttt{x >= 1}, \texttt{x < 1}, \texttt{x <= 1}.
    \item Se pueden combinar condiciones con \texttt{and} / \texttt{or}.
\end{itemize}

\begin{minted}{python}
if x > 10 or x < 1:
  # hacer algo
if x >= 1 and x <= 10:
  # hacer otra cosa
\end{minted}
\end{frame}

\begin{frame}[fragile]{\texttt{else} y \texttt{elif}}
\begin{minted}{python}
if x>10:
  print("Su numero es mayor que diez.")
else:
  print("Su numero esta correcto. Nada por hacer.")
\end{minted}

\begin{minted}{python}
if x>10:
  print("Su numero es mayor que diez.")
elif x>9:
  print("Su numero esta bien, pero cercano a 10.")
else:
  print("Su numero esta correcto. Nada por hacer.")
\end{minted}
\end{frame}

\subsection{La declaración WHILE}
\begin{frame}[fragile]{La declaración \texttt{WHILE}}
Útil para repetir un bloque mientras se cumpla alguna condición:
\begin{minted}{python}
x = int(input("Ingrese numero <= 10: "))
while x > 10:
  print("Es mayor que diez. Reintente.")
  x = int(input("Ingrese numero <= 10: "))
print("Su numero es", x)
\end{minted}

\textbf{Combinando}:  
\begin{minted}{python}
while x>10 or x<1:
  # ...
\end{minted}
\end{frame}

\section{La secuencia de Fibonacci}
\begin{frame}{La secuencia de Fibonacci}
Los números de Fibonacci se definen como:
\[
F_n = F_{n-1} + F_{n-2}, \quad F_1 = 1, F_2 = 1
\]

Los primeros términos: 1, 1, 2, 3, 5, 8, 13, 21, ...

\end{frame}

\begin{frame}[fragile]{Ejemplo: Fibonacci hasta 1000}
\begin{minted}{python}
f1 = 1
f2 = 1
next_ = f1 + f2

while f1 <= 1000:
  print(f1)
  f1 = f2
  f2 = next_
  next_ = f1 + f2
\end{minted}
\end{frame}

\section{Funciones definidas por el usuario}
\begin{frame}[fragile]{Funciones definidas por el usuario}
A menudo necesitamos funciones específicas. Ej: calcular factorial \(n!\):
\begin{minted}{python}
def factorial(n):
  f = 1
  k = 1
  while (k <= n):
    f *= k
    k += 1
  return f

# Programa principal
val = int(input("Ingrese un entero positivo: "))
print(val, "! =", factorial(val))
\end{minted}
\end{frame}

\begin{frame}{Funciones definidas por el usuario}
\begin{itemize}
    \item \texttt{def nombre\_funcion(argumento):}  
      abre la definición de la función.  
    \item Bloque indentado = cuerpo de la función.
    \item \texttt{return} finaliza la función y devuelve un valor.
\end{itemize}
\end{frame}

\section{Recursividad}
\begin{frame}[fragile]{Recursividad}
La recursividad consiste en que la función se llama a sí misma (para subproblemas más pequeños):
\begin{minted}{python}
def factorial(n):
  if n == 1:
    return 1
  else:
    return n * factorial(n-1)
\end{minted}
\end{frame}

\section{Actividades}
\begin{frame}{Actividades para Hoy}
\begin{itemize}
	\item Implemente un programa que pida un número y verifique si es par o impar.  
	      Use un condicional \texttt{if} o \texttt{while} hasta que sea válido.  
	\item Extienda el ejemplo de Fibonacci para que el usuario indique hasta qué número máximo desea imprimir.  
	\item Defina una función que calcule la potencia \(a^b\) mediante multiplicaciones sucesivas (sin usar el operador \texttt{**}). Compare con la función incorporada.
\end{itemize}
\end{frame}

\begin{frame}
\Huge{\centerline{Fin de la Parte 2}}
\end{frame}

\end{document}

