\documentclass[10pt]{beamer}

% ------------------------------------------------------------------------
% Carga del preámbulo personalizado (preamble.tex)
% Asegúrate de tenerlo en la misma carpeta para que \input funcione.
% ------------------------------------------------------------------------
\usetheme[progressbar=frametitle]{metropolis}
\usepackage{appendixnumberbeamer}
\usepackage{fancyvrb}
\usepackage{booktabs}
\usepackage[scale=2]{ccicons}
\usepackage{pgfplots}
\usepgfplotslibrary{dateplot}
\usepackage{type1cm}
\usepackage{lettrine}
\usepackage{ragged2e}
\usepackage{xspace}
\newcommand{\themename}{\textbf{\textsc{metropolis}}\xspace}
\usepackage{graphicx} % Allows including images
\usepackage{booktabs} % Allows the use of \toprule, \midrule and \bottomrule in tables
\usepackage[utf8]{inputenc} %solucion del problema de los acentos.
\usepackage{xcolor}
\usepackage{verbatim}

\definecolor{LightGray}{gray}{0.9}
% Paleta de azules estilo Python/matplotlib
\definecolor{PythonBlue}{RGB}{31, 119, 180}      % Azul principal matplotlib
\definecolor{LightPythonBlue}{RGB}{174, 199, 232} % Azul claro para fondos
\definecolor{DarkPythonBlue}{RGB}{23, 90, 135}    % Azul oscuro para líneas

% Colores para tema oscuro tipo VS Code/Colab
\definecolor{DarkBackground}{RGB}{40, 44, 52}     % Fondo oscuro similar a VS Code
\definecolor{CodeText}{RGB}{171, 178, 191}        % Texto gris claro para código
\definecolor{DarkFrame}{RGB}{60, 63, 65}          % Color del marco/borde

% Colores para tema claro
\definecolor{LightBackground}{RGB}{248, 248, 242} % Fondo claro tipo GitHub
\definecolor{LightCodeText}{RGB}{51, 51, 51}      % Texto oscuro para tema claro
\definecolor{LightFrame}{RGB}{220, 220, 220}      % Marco gris claro

\usepackage{minted}

%%%%%%%%%%%%%%%%%%%%%%%%%%%%%%%%%%%%%%%%%%%%%%%%%%%%%%%%%%%%%%%%%%%%%%%%%%%%%%%%%%%%%%
% CONFIGURACIÓN DE TEMAS - CAMBIA AQUÍ PARA ALTERNAR
% ========================================================================
% Descomenta UNA de las siguientes líneas para elegir el tema:

% TEMA OSCURO (VS Code style)
%\newcommand{\mytheme}{oscuro}
%\usemintedstyle{monokai}
%\newcommand{\mybgcolor}{DarkBackground}

% TEMA CLARO (GitHub style) - comenta las 3 líneas de arriba y descomenta estas:
 \newcommand{\mytheme}{claro}
 \usemintedstyle{default}
 \newcommand{\mybgcolor}{LightBackground}

% ========================================================================
\newcommand{\mypyfile}[1]{\inputminted[linenos=true, fontsize=\footnotesize, frame=lines, framesep=5\fboxrule,framerule=1pt]{python}{#1}}

% Comando personalizado para código Python inline
\newcommand{\pycode}[1]{\begin{minted}{python}#1\end{minted}}

% ========================================================================
% COMANDOS ESPECÍFICOS PARA CADA TEMA
% ========================================================================
% Comando para tema oscuro específico
\newcommand{\pyoscuro}[1]{%
\begin{minted}[
  bgcolor=DarkBackground,
  style=monokai,
  breaklines,
  frame=lines,
  framesep=2mm,
  baselinestretch=1.2,
  linenos,
  fontsize=\footnotesize
]{python}
#1
\end{minted}
}

% Comando para tema claro específico
\newcommand{\pyclaro}[1]{%
\begin{minted}[
  bgcolor=LightBackground,
  style=default,
  breaklines,
  frame=lines,
  framesep=2mm,
  baselinestretch=1.2,
  linenos,
  fontsize=\footnotesize
]{python}
#1
\end{minted}
}
%%%%%%%%%%%%%%%%%%%%%%%%%%%%%%%%%%%%%%%%%%%%%%%%%%%%%%%%%%%%%%%%%%%%%%%%%%%%%%%%%%%%%%



%%%%%%%%%%%%%%%%%%%%%%%%%%%%%%%%%%%%%%%%%%%%%%%%%%%%%%%%%%%%%%%%%%%%%%%%%%%%%%%%%%%%%%
% Configuración global para todos los bloques minted de Python
% Usa automáticamente el tema seleccionado arriba
\setminted[python]{
  breaklines,
  frame=lines,
  framesep=2mm,
  baselinestretch=1.2,
  bgcolor=\mybgcolor,
  linenos, 
  fontsize=\footnotesize
} % Configuración dinámica según tema elegido
%%%%%%%%%%%%%%%%%%%%%%%%%%%%%%%%%%%%%%%%%%%%%%%%%%%%%%%%%%%%%%%%%%%%%%%%%%%%%%%%%%%%%%



%%%%%%%%%%%%%%%%%%%%%%%%%%%%%%%%%%%%%%%%%%%%%%%%%%%%%%%%%%%%%%%%%%%%%%%%%%%%%%%%%%%%%%
% Configuración de colores del tema con paleta azul Python
\setbeamercolor{progress bar}{fg=DarkPythonBlue,bg=LightPythonBlue}
\setbeamercolor{title separator}{fg=DarkPythonBlue,bg=white!50!black}
\setbeamercolor{frametitle}{fg=white,bg=PythonBlue}
\title[PCFI161]{Programaci\'on para F\'isica y Astronom\'ia}
\subtitle{Departamento de Física.}

\newcommand{\myfront}{
\author[PCFI161]{Corodinadora: C Loyola \\ Profesores C Femenías / F Bugini / D Basantes}
\institute[UNAB]{Universidad Andrés Bello \\ Departamento de Física y Astronomía}
\date{Primer Semestre 2025}
}

\titlegraphic{
  \includegraphics[width=.08\textwidth]{1) Logo/logo-tux.png}\hfill
  \includegraphics[width=.3\textwidth]{1) Logo/logo-unab.png}\hfill
  \includegraphics[width=.08\textwidth]{1) Logo/logo-python.png}
}

\makeatletter
\setbeamertemplate{title page}{
  \begin{minipage}[b][\paperheight]{\textwidth}
    \vfill%
    \ifx\inserttitle\@empty\else\usebeamertemplate*{title}\fi
    \ifx\insertsubtitle\@empty\else\usebeamertemplate*{subtitle}\fi
    \usebeamertemplate*{title separator}
    \ifx\beamer@shortauthor\@empty\else\usebeamertemplate*{author}\fi
    \ifx\insertdate\@empty\else\usebeamertemplate*{date}\fi
    \ifx\insertinstitute\@empty\else\usebeamertemplate*{institute}\fi
    \vfill
    \ifx\inserttitlegraphic\@empty\else\inserttitlegraphic\fi
    \vspace*{1cm}
  \end{minipage}
}
\makeatother

% Configuración personalizada del frametitle para incluir la sección
\makeatletter
\setbeamertemplate{frametitle}{%
  \nointerlineskip%
  \begin{beamercolorbox}[%
      wd=\paperwidth,%
      sep=0pt,%
      leftskip=\@ifundefined{metropolis@frametitle@padding}{12pt}{\metropolis@frametitle@padding},%
      rightskip=\@ifundefined{metropolis@frametitle@padding}{12pt}{\metropolis@frametitle@padding},%
    ]{frametitle}%
  \@ifundefined{metropolis@frametitlestrut@start}{}{\metropolis@frametitlestrut@start}%
  {\textcolor{LightPythonBlue}{\insertsectionhead}\ifx\insertsectionhead\@empty\else\ $\ni$  \fi}\insertframetitle%
  \nolinebreak%
  \@ifundefined{metropolis@frametitlestrut@end}{}{\metropolis@frametitlestrut@end}%
  \end{beamercolorbox}%
}
\makeatother

\makeatletter
% Configuración segura de longitudes del tema Metropolis
\@ifundefined{metropolis@titleseparator@linewidth}{}{\setlength{\metropolis@titleseparator@linewidth}{2pt}}
\@ifundefined{metropolis@progressonsectionpage@linewidth}{}{\setlength{\metropolis@progressonsectionpage@linewidth}{2pt}}
\@ifundefined{metropolis@progressinheadfoot@linewidth}{}{\setlength{\metropolis@progressinheadfoot@linewidth}{2pt}}
\makeatother



\begin{document}

% ------------------------------------------------------------------------
% Portada personalizada (puedes usar \myfront si está definido en tu preamble.tex)
% ------------------------------------------------------------------------
\myfront{}

% ------------------------------------------------------------------------
% Slide 1: Título de la Sesión
% ------------------------------------------------------------------------
\begin{frame}
  \titlepage
  % Ejemplo de configuración del título:
  % \title{Semana 3 - Sesión 2 (Sesión 6): Laboratorio Práctico de Estructuras de Control}
\end{frame}

% ------------------------------------------------------------------------
% Slide 2: Índice / Tabla de contenidos
% ------------------------------------------------------------------------
\begin{frame}
  \frametitle{Resumen - Semana 3, Sesión 2 (Sesión 6)}
  \tableofcontents
\end{frame}

% ------------------------------------------------------------------------
% Configuración de bloques
% ------------------------------------------------------------------------
\metroset{block=fill}

% ----------------------------------------------------------------------------------------
% SECCIÓN 1: Conexión con la Sesión Anterior
% ----------------------------------------------------------------------------------------
\section{Introducción y Repaso}

% ------------------------------------------------------------------------
% Slide 3: Recapitulación de la Sesión 5
% ------------------------------------------------------------------------
\begin{frame}{Recapitulación de la Sesión Previa (Sesión 5)}
  \begin{itemize}
    \item \textbf{Semana 3, Sesión 1 (Sesión 5)} se enfocó en:
      \begin{itemize}
        \item Introducir las estructuras de control fundamentales en Python:
          \begin{itemize}
            \item \texttt{if}, \texttt{elif}, \texttt{else} para condicionales.
            \item \texttt{while} para bucles basados en condiciones lógicas.
          \end{itemize}
        \item Ejemplos sencillos de decisiones y repeticiones.
        \item Manejo de la indentación y bloques de código.
      \end{itemize}
    \item \textbf{Objetivo de hoy}: Realizar un \textbf{laboratorio práctico} con ejercicios enfocados en la aplicación de condicionales y bucles.
  \end{itemize}
\end{frame}

% ------------------------------------------------------------------------
% Slide 4: Objetivos de la Sesión 6
% ------------------------------------------------------------------------
\begin{frame}{Objetivos de la Sesión 6}
  \begin{itemize}
    \item \textbf{Aplicar} las estructuras de control en la resolución de problemas concretos.
    \item \textbf{Fomentar} el trabajo en parejas o grupos para compartir estrategias.
    \item \textbf{Profundizar} en el uso de \texttt{if-elif-else} y \texttt{while}.
    \item \textbf{Fortalecer} la comprensión de la lógica condicional y bucles a través de experimentación en Colab.
  \end{itemize}
\end{frame}

% ----------------------------------------------------------------------------------------
% SECCIÓN 2: Recordatorio de Estructuras de Control
% ----------------------------------------------------------------------------------------
\section{Recordatorio: Estructuras de Control}

% ------------------------------------------------------------------------
% Slide 5: Condicionales (If, Elif, Else)
% ------------------------------------------------------------------------
\begin{frame}[fragile]{Condicionales en Python}
\begin{minted}[
  frame=lines,
  framesep=2mm,
  baselinestretch=1.1,
  bgcolor=LightGray,
  fontsize=\footnotesize
]{python}
if condicion1:
    # Bloque de código si condicion1 es True
elif condicion2:
    # Bloque de código si condicion2 es True
else:
    # Bloque de código si ninguna de las anteriores
\end{minted}
\begin{itemize}
  \item Cada condición se evalúa en orden.
  \item Solo se ejecuta el primer bloque que resulte \texttt{True}.
  \item \textbf{Uso común}: validaciones, menús, decisiones.
\end{itemize}
\end{frame}

% ------------------------------------------------------------------------
% Slide 6: While (Bucles Basados en Condición)
% ------------------------------------------------------------------------
\begin{frame}[fragile]{Bucle While en Python}
\begin{minted}[
  frame=lines,
  framesep=2mm,
  baselinestretch=1.2,
  bgcolor=LightGray,
  fontsize=\footnotesize
]{python}
while condicion:
    # bloque que se repite
    # mientras la condicion sea True

# Al salir, la condicion es False (o se rompió el bucle con break)
\end{minted}
\begin{itemize}
  \item Útil cuando no se sabe cuántas iteraciones exactas serán necesarias.
  \item \textbf{break}: fuerza la salida del bucle.
  \item \textbf{continue}: salta a la siguiente iteración.
\end{itemize}
\end{frame}

% ------------------------------------------------------------------------
% Slide 7: Ejemplo Breve - Menú Interactivo
% ------------------------------------------------------------------------
\begin{frame}[fragile]{Ejemplo Rápido: Menú Interactivo con While}
\begin{minted}[
  frame=lines,
  framesep=2mm,
  bgcolor=LightGray,
  fontsize=\footnotesize
]{python}
op = ""
while op != "q":
    print("Menú:")
    print("(1) Saludar")
    print("(2) Despedir")
    print("(q) Salir")
    op = input("Opción: ")

    if op == "1":
        print("Hola!")
    elif op == "2":
        print("Adiós!")
    elif op == "q":
        print("Saliendo...")
    else:
        print("Opción inválida")
\end{minted}
\end{frame}

% ----------------------------------------------------------------------------------------
% SECCIÓN 3: Laboratorio Práctico
% ----------------------------------------------------------------------------------------
\section{Laboratorio Práctico}

% ------------------------------------------------------------------------
% Slide 8: Actividad General
% ------------------------------------------------------------------------
\begin{frame}{Actividad General - Estructuras de Control}
  \begin{itemize}
    \item Se plantearán \textbf{tres problemas} de complejidad progresiva.
    \item Cada problema requiere el uso de condicionales y/o bucles \texttt{while}.
    \item Trabajaremos en \textbf{parejas} o \textbf{pequeños grupos}.
    \item Al final, compartiremos soluciones y discutiremos diferentes enfoques.
  \end{itemize}
\end{frame}

% ------------------------------------------------------------------------
% Slide 9: Problema 1 - Adivina el Número
% ------------------------------------------------------------------------
\begin{frame}{Problema 1: Adivina el Número}
  \begin{block}{Enunciado}
    \begin{itemize}
      \item El programa genera un número entero aleatorio entre 1 y 50.
      \item El usuario debe adivinar el número.
      \item Se le indica si su intento es \textbf{muy alto} o \textbf{muy bajo}, hasta acertar.
      \item Cuando acierta, se imprime cuántos intentos utilizó.
    \end{itemize}
  \end{block}
  \textbf{Indicaciones}:
  \begin{itemize}
    \item Usar \texttt{while} para repetir hasta acertar.
    \item \textbf{módulo random}: \texttt{import random; random.randint(1,50)}.
  \end{itemize}
\end{frame}

% ------------------------------------------------------------------------
% Slide 10: Problema 2 - Calculadora de Calificaciones
% ------------------------------------------------------------------------
\begin{frame}{Problema 2: Calculadora de Calificaciones}
  \begin{block}{Enunciado}
    \begin{itemize}
      \item Pedir repetidamente \textbf{notas} de estudiantes en el rango [1.0 - 7.0].
      \item Acumular la suma y el conteo de notas.
      \item Si se ingresa \texttt{-1}, termina la captura de datos.
      \item Imprimir el \textbf{promedio final} de las notas ingresadas (o mensaje si no se ingresó ninguna).
    \end{itemize}
  \end{block}
  \textbf{Puntos Clave}:
  \begin{itemize}
    \item \textbf{while} para la repetición.
    \item Validar que la nota esté en [1.0, 7.0] o sea \texttt{-1} para salir.
  \end{itemize}
\end{frame}

% ------------------------------------------------------------------------
% Slide 11: Problema 3 - Simulación de Movimiento Discreto
% ------------------------------------------------------------------------
\begin{frame}{Problema 3: Simulación de Movimiento Discreto}
  \begin{block}{Enunciado}
    \begin{itemize}
      \item Imaginemos un objeto en la posición \texttt{x=0} de una línea de números.
      \item Cada paso de tiempo, el usuario ingresa un comando:
        \begin{itemize}
          \item \texttt{R} (mover +1)
          \item \texttt{L} (mover -1)
          \item \texttt{S} (mantenerse)
          \item \texttt{Q} (terminar)
        \end{itemize}
      \item El programa muestra la posición actualizada tras cada comando.
      \item Al final, imprime cuántos pasos se realizaron.
    \end{itemize}
  \end{block}
  \textbf{Sugerencia}:
  \begin{itemize}
    \item Usar un \texttt{while} infinito y \texttt{break} al recibir \texttt{"Q"}.
    \item Contar la cantidad de pasos totales (excluyendo el comando “Q”).
  \end{itemize}
\end{frame}

% ------------------------------------------------------------------------
% Slide 12: Formación de Grupos
% ------------------------------------------------------------------------
\begin{frame}{Formación de Grupos y Organización}
  \begin{itemize}
    \item Dividir la clase en \textbf{parejas} o \textbf{tríos}, según tamaño del grupo total.
    \item Cada equipo elige uno o más problemas a resolver:
      \begin{itemize}
        \item Problema 1: Adivina el Número.
        \item Problema 2: Calculadora de Calificaciones.
        \item Problema 3: Simulación de Movimiento Discreto.
      \end{itemize}
    \item \textbf{Objetivo}: Implementar y probar en Google Colab, comentando el código.
  \end{itemize}
\end{frame}

% ------------------------------------------------------------------------
% Slide 13: Lineamientos Generales
% ------------------------------------------------------------------------
\begin{frame}{Lineamientos para Resolver los Problemas}
  \begin{itemize}
    \item Crear un \textbf{notebook} específico para esta sesión.
    \item Explicar brevemente la lógica al inicio de cada solución.
    \item Probar \textbf{múltiples casos} (valores límites, valores inválidos, etc.).
    \item Anotar cualquier dificultad o bug encontrado, junto con la forma de resolverlo.
  \end{itemize}
\end{frame}

% ----------------------------------------------------------------------------------------
% SECCIÓN 4: Desarrollo y Discusión
% ----------------------------------------------------------------------------------------
\section{Desarrollo y Discusión}

% ------------------------------------------------------------------------
% Slide 14: Trabajo en Clase
% ------------------------------------------------------------------------
\begin{frame}{Trabajo en Clase}
  \begin{block}{Instrucciones}
    \begin{itemize}
      \item Cada grupo trabaja \textbf{20-25 minutos} en los problemas.
      \item Se recomienda priorizar la corrección lógica antes que agregar extras.
      \item Aquellos que terminen rápido pueden \textbf{extender} sus soluciones (más validaciones, mensajes más detallados, etc.).
    \end{itemize}
  \end{block}
  \textbf{Consejo}: Revisen sintaxis de \texttt{while}, \texttt{if/elif/else}, y funciones de \texttt{random}.
\end{frame}

% ------------------------------------------------------------------------
% Slide 15: Espacio para Dudas
% ------------------------------------------------------------------------
\begin{frame}{Espacio para Dudas}
  \begin{itemize}
    \item ¿Algún error frecuente al usar \texttt{while}?
    \item ¿Cómo manejan valores fuera del rango permitido?
    \item ¿Cómo se maneja la salida del bucle (\texttt{break} vs condición \texttt{False})?
  \end{itemize}
  \vspace{0.3cm}
  \textbf{Levanta la mano o consulta en voz alta para que todos aprendan.}
\end{frame}

% ------------------------------------------------------------------------
% Slide 16: Comparación de Enfoques
% ------------------------------------------------------------------------
\begin{frame}{Comparación de Enfoques en la Clase}
  \begin{itemize}
    \item Distintas formas de plantear la condición principal del \texttt{while}.
    \item Manejo de variables “centinela” (por ejemplo, \texttt{-1} para terminar).
    \item Uso de \texttt{else} en condicionales anidados para simplificar la lógica.
    \item \textbf{Importancia}: Legibilidad vs. tamaño del código.
  \end{itemize}
\end{frame}

% ------------------------------------------------------------------------
% Slide 17: Ejemplo de Solución - Problema 1 (Adivina el Número)
% ------------------------------------------------------------------------
\begin{frame}[fragile]{Ejemplo de Solución: Adivina el Número}
\begin{minted}[
  frame=lines,
  framesep=2mm,
  bgcolor=LightGray,
  fontsize=\footnotesize
]{python}
import random

secreto = random.randint(1,50)
intentos = 0
adiv = 0

while adiv != secreto:
    adiv_str = input("Adivina un número (1-50): ")
    adiv = int(adiv_str)
    intentos += 1

    if adiv < secreto:
        print("Muy bajo!")
    elif adiv > secreto:
        print("Muy alto!")
    else:
        print("¡Acertaste!")

print(f'Número de intentos: {intentos}')
\end{minted}
\end{frame}

% ------------------------------------------------------------------------
% Slide 18: Ejemplo de Solución - Problema 2 (Calificaciones)
% ------------------------------------------------------------------------
\begin{frame}[fragile]{Ejemplo de Solución: Calculadora de Calificaciones}
\begin{minted}[
  frame=lines,
  framesep=2mm,
  bgcolor=LightGray,
  fontsize=\scriptsize
]{python}
count = 0
suma = 0.0

while True:
    nota_str = input("Ingrese una nota [1.0-7.0], o -1 para terminar: ")
    nota = float(nota_str)

    if nota == -1:
        break
    if nota < 1.0 or nota > 7.0:
        print("Nota inválida, intente nuevamente.")
        continue

    suma += nota
    count += 1

if count > 0:
    promedio = suma / count
    print(f'Promedio de las {count} notas: {promedio}')
else:
    print("No se ingresaron notas válidas.")
\end{minted}
\end{frame}

% ------------------------------------------------------------------------
% Slide 19: Ejemplo de Solución - Problema 3 (Movimiento Discreto)
% ------------------------------------------------------------------------
\begin{frame}[fragile]{Ejemplo de Solución: Movimiento Discreto}
\begin{minted}[
  frame=lines,
  framesep=2mm,
  bgcolor=LightGray,
  fontsize=\tiny
]{python}
pos = 0
pasos = 0

while True:
    comando = input("Ingrese (R, L, S, Q): ").upper()

    if comando == "R":
        pos += 1
        pasos += 1
    elif comando == "L":
        pos -= 1
        pasos += 1
    elif comando == "S":
        # no cambia pos
        pasos += 1
    elif comando == "Q":
        print("Terminando simulación.")
        break
    else:
        print("Comando inválido.")

    print(f'Posición actual: {pos}')

print(f'Total de pasos realizados: {pasos}')
\end{minted}
\end{frame}

% ------------------------------------------------------------------------
% Slide 20: Análisis de las Soluciones
% ------------------------------------------------------------------------
\begin{frame}{Análisis de las Soluciones}
  \begin{itemize}
    \item \textbf{Uso de banderas y centinelas}:
      \begin{itemize}
        \item \texttt{nota == -1} para terminar ciclo.
        \item \texttt{comando == "Q"} para salir.
      \end{itemize}
    \item \textbf{Validaciones} y manejo de casos inválidos (\texttt{continue}).
    \item \textbf{Contadores} de intentos y pasos, incrementándose en cada iteración.
    \item \textbf{Mejoras potenciales}:
      \begin{itemize}
        \item Mensajes más descriptivos.
        \item Manejo de excepciones para entradas no numéricas.
      \end{itemize}
  \end{itemize}
\end{frame}

% ----------------------------------------------------------------------------------------
% SECCIÓN 5: Retroalimentación y Conclusiones
% ----------------------------------------------------------------------------------------
\section{Conclusiones}

% ------------------------------------------------------------------------
% Slide 21: Integración de Conocimientos
% ------------------------------------------------------------------------
\begin{frame}{Integración de Conocimientos}
  \begin{itemize}
    \item \textbf{Condicionales} son la base de la \textit{toma de decisiones} en un programa.
    \item \textbf{Bucle While} posibilita la \textit{repetición basada en una condición}, útil para menús interactivos o lectura indefinida.
    \item Aprendimos la importancia de \textbf{validar} datos y usar \texttt{break}/\texttt{continue} en casos convenientes.
  \end{itemize}
\end{frame}

% ------------------------------------------------------------------------
% Slide 22: Recomendaciones para el Estudio
% ------------------------------------------------------------------------
\begin{frame}{Recomendaciones para el Estudio}
  \begin{itemize}
    \item Practica \textbf{problemas pequeños} con \texttt{if} y \texttt{while}, como minijuegos o menús.
    \item Lee la documentación de \textbf{Python} sobre control de flujo (if, while, for).
    \item Experimenta con valores \textbf{fuera de rango} para entender mejor la lógica y los errores potenciales.
  \end{itemize}
\end{frame}

% ------------------------------------------------------------------------
% Slide 23: Próximos Pasos
% ------------------------------------------------------------------------
\begin{frame}{Próximos Pasos}
  \begin{itemize}
    \item Siguiente tema: \textbf{Bucle For} y manejo de listas (Unidad III).
    \item Revisaremos \textbf{break} y \textbf{continue} con más detalle.
    \item Pronto veremos estructuras de datos (listas, arreglos) e interacción con bucles para manipular secuencias.
  \end{itemize}
\end{frame}

% ------------------------------------------------------------------------
% Slide 24: Recursos Adicionales
% ------------------------------------------------------------------------
\begin{frame}{Recursos Adicionales}
  \begin{itemize}
    \item \href{https://docs.python.org/3/tutorial/controlflow.html}{\textbf{Python Docs - Control Flow}}
    \item \href{https://www.learnpython.org/en/Conditions}{\textbf{Learn Python - Conditions}}
    \item \href{https://realpython.com/python-while-loop/}{\textbf{Real Python - While Loops}}
    \item \textbf{Foros y Comunidades}: Stack Overflow, Reddit \texttt{/r/learnpython}.
  \end{itemize}
\end{frame}

% ------------------------------------------------------------------------
% Slide 25: Cierre de la Sesión
% ------------------------------------------------------------------------
\begin{frame}
  \Huge{\centerline{¡Gracias por su atención!}}
  \vspace{0.4cm}
  \normalsize
  \begin{itemize}
    \item Aseguren de \textbf{guardar} sus notebooks de hoy.
    \item ¡Nos vemos en la próxima sesión para seguir avanzando con \texttt{for}, listas y más ejemplos!
  \end{itemize}
\end{frame}

\end{document}

