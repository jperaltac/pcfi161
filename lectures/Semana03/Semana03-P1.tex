\documentclass[10pt]{beamer}

% ------------------------------------------------------------------------
% Carga de tu preámbulo personalizado (preamble.tex)
% Asegúrate de tenerlo en la misma carpeta para que \usetheme[progressbar=frametitle]{metropolis}
\usepackage{appendixnumberbeamer}
\usepackage{fancyvrb}
\usepackage{booktabs}
\usepackage[scale=2]{ccicons}
\usepackage{pgfplots}
\usepgfplotslibrary{dateplot}
\usepackage{type1cm}
\usepackage{lettrine}
\usepackage{ragged2e}
\usepackage{xspace}
\newcommand{\themename}{\textbf{\textsc{metropolis}}\xspace}
\usepackage{graphicx} % Allows including images
\usepackage{booktabs} % Allows the use of \toprule, \midrule and \bottomrule in tables
\usepackage[utf8]{inputenc} %solucion del problema de los acentos.
\usepackage{xcolor}
\usepackage{verbatim}

\definecolor{LightGray}{gray}{0.9}
% Paleta de azules estilo Python/matplotlib
\definecolor{PythonBlue}{RGB}{31, 119, 180}      % Azul principal matplotlib
\definecolor{LightPythonBlue}{RGB}{174, 199, 232} % Azul claro para fondos
\definecolor{DarkPythonBlue}{RGB}{23, 90, 135}    % Azul oscuro para líneas

% Colores para tema oscuro tipo VS Code/Colab
\definecolor{DarkBackground}{RGB}{40, 44, 52}     % Fondo oscuro similar a VS Code
\definecolor{CodeText}{RGB}{171, 178, 191}        % Texto gris claro para código
\definecolor{DarkFrame}{RGB}{60, 63, 65}          % Color del marco/borde

% Colores para tema claro
\definecolor{LightBackground}{RGB}{248, 248, 242} % Fondo claro tipo GitHub
\definecolor{LightCodeText}{RGB}{51, 51, 51}      % Texto oscuro para tema claro
\definecolor{LightFrame}{RGB}{220, 220, 220}      % Marco gris claro

\usepackage{minted}

%%%%%%%%%%%%%%%%%%%%%%%%%%%%%%%%%%%%%%%%%%%%%%%%%%%%%%%%%%%%%%%%%%%%%%%%%%%%%%%%%%%%%%
% CONFIGURACIÓN DE TEMAS - CAMBIA AQUÍ PARA ALTERNAR
% ========================================================================
% Descomenta UNA de las siguientes líneas para elegir el tema:

% TEMA OSCURO (VS Code style)
%\newcommand{\mytheme}{oscuro}
%\usemintedstyle{monokai}
%\newcommand{\mybgcolor}{DarkBackground}

% TEMA CLARO (GitHub style) - comenta las 3 líneas de arriba y descomenta estas:
 \newcommand{\mytheme}{claro}
 \usemintedstyle{default}
 \newcommand{\mybgcolor}{LightBackground}

% ========================================================================
\newcommand{\mypyfile}[1]{\inputminted[linenos=true, fontsize=\footnotesize, frame=lines, framesep=5\fboxrule,framerule=1pt]{python}{#1}}

% Comando personalizado para código Python inline
\newcommand{\pycode}[1]{\begin{minted}{python}#1\end{minted}}

% ========================================================================
% COMANDOS ESPECÍFICOS PARA CADA TEMA
% ========================================================================
% Comando para tema oscuro específico
\newcommand{\pyoscuro}[1]{%
\begin{minted}[
  bgcolor=DarkBackground,
  style=monokai,
  breaklines,
  frame=lines,
  framesep=2mm,
  baselinestretch=1.2,
  linenos,
  fontsize=\footnotesize
]{python}
#1
\end{minted}
}

% Comando para tema claro específico
\newcommand{\pyclaro}[1]{%
\begin{minted}[
  bgcolor=LightBackground,
  style=default,
  breaklines,
  frame=lines,
  framesep=2mm,
  baselinestretch=1.2,
  linenos,
  fontsize=\footnotesize
]{python}
#1
\end{minted}
}
%%%%%%%%%%%%%%%%%%%%%%%%%%%%%%%%%%%%%%%%%%%%%%%%%%%%%%%%%%%%%%%%%%%%%%%%%%%%%%%%%%%%%%



%%%%%%%%%%%%%%%%%%%%%%%%%%%%%%%%%%%%%%%%%%%%%%%%%%%%%%%%%%%%%%%%%%%%%%%%%%%%%%%%%%%%%%
% Configuración global para todos los bloques minted de Python
% Usa automáticamente el tema seleccionado arriba
\setminted[python]{
  breaklines,
  frame=lines,
  framesep=2mm,
  baselinestretch=1.2,
  bgcolor=\mybgcolor,
  linenos, 
  fontsize=\footnotesize
} % Configuración dinámica según tema elegido
%%%%%%%%%%%%%%%%%%%%%%%%%%%%%%%%%%%%%%%%%%%%%%%%%%%%%%%%%%%%%%%%%%%%%%%%%%%%%%%%%%%%%%



%%%%%%%%%%%%%%%%%%%%%%%%%%%%%%%%%%%%%%%%%%%%%%%%%%%%%%%%%%%%%%%%%%%%%%%%%%%%%%%%%%%%%%
% Configuración de colores del tema con paleta azul Python
\setbeamercolor{progress bar}{fg=DarkPythonBlue,bg=LightPythonBlue}
\setbeamercolor{title separator}{fg=DarkPythonBlue,bg=white!50!black}
\setbeamercolor{frametitle}{fg=white,bg=PythonBlue}
\title[PCFI161]{Programaci\'on para F\'isica y Astronom\'ia}
\subtitle{Departamento de Física.}

\newcommand{\myfront}{
\author[PCFI161]{Corodinadora: C Loyola \\ Profesores C Femenías / F Bugini / D Basantes}
\institute[UNAB]{Universidad Andrés Bello \\ Departamento de Física y Astronomía}
\date{Primer Semestre 2025}
}

\titlegraphic{
  \includegraphics[width=.08\textwidth]{1) Logo/logo-tux.png}\hfill
  \includegraphics[width=.3\textwidth]{1) Logo/logo-unab.png}\hfill
  \includegraphics[width=.08\textwidth]{1) Logo/logo-python.png}
}

\makeatletter
\setbeamertemplate{title page}{
  \begin{minipage}[b][\paperheight]{\textwidth}
    \vfill%
    \ifx\inserttitle\@empty\else\usebeamertemplate*{title}\fi
    \ifx\insertsubtitle\@empty\else\usebeamertemplate*{subtitle}\fi
    \usebeamertemplate*{title separator}
    \ifx\beamer@shortauthor\@empty\else\usebeamertemplate*{author}\fi
    \ifx\insertdate\@empty\else\usebeamertemplate*{date}\fi
    \ifx\insertinstitute\@empty\else\usebeamertemplate*{institute}\fi
    \vfill
    \ifx\inserttitlegraphic\@empty\else\inserttitlegraphic\fi
    \vspace*{1cm}
  \end{minipage}
}
\makeatother

% Configuración personalizada del frametitle para incluir la sección
\makeatletter
\setbeamertemplate{frametitle}{%
  \nointerlineskip%
  \begin{beamercolorbox}[%
      wd=\paperwidth,%
      sep=0pt,%
      leftskip=\@ifundefined{metropolis@frametitle@padding}{12pt}{\metropolis@frametitle@padding},%
      rightskip=\@ifundefined{metropolis@frametitle@padding}{12pt}{\metropolis@frametitle@padding},%
    ]{frametitle}%
  \@ifundefined{metropolis@frametitlestrut@start}{}{\metropolis@frametitlestrut@start}%
  {\textcolor{LightPythonBlue}{\insertsectionhead}\ifx\insertsectionhead\@empty\else\ $\ni$  \fi}\insertframetitle%
  \nolinebreak%
  \@ifundefined{metropolis@frametitlestrut@end}{}{\metropolis@frametitlestrut@end}%
  \end{beamercolorbox}%
}
\makeatother

\makeatletter
% Configuración segura de longitudes del tema Metropolis
\@ifundefined{metropolis@titleseparator@linewidth}{}{\setlength{\metropolis@titleseparator@linewidth}{2pt}}
\@ifundefined{metropolis@progressonsectionpage@linewidth}{}{\setlength{\metropolis@progressonsectionpage@linewidth}{2pt}}
\@ifundefined{metropolis@progressinheadfoot@linewidth}{}{\setlength{\metropolis@progressinheadfoot@linewidth}{2pt}}
\makeatother

 funcione.
% ------------------------------------------------------------------------
\usetheme[progressbar=frametitle]{metropolis}
\usepackage{appendixnumberbeamer}
\usepackage{fancyvrb}
\usepackage{booktabs}
\usepackage[scale=2]{ccicons}
\usepackage{pgfplots}
\usepgfplotslibrary{dateplot}
\usepackage{type1cm}
\usepackage{lettrine}
\usepackage{ragged2e}
\usepackage{xspace}
\newcommand{\themename}{\textbf{\textsc{metropolis}}\xspace}
\usepackage{graphicx} % Allows including images
\usepackage{booktabs} % Allows the use of \toprule, \midrule and \bottomrule in tables
\usepackage[utf8]{inputenc} %solucion del problema de los acentos.
\usepackage{xcolor}
\usepackage{verbatim}

\definecolor{LightGray}{gray}{0.9}
% Paleta de azules estilo Python/matplotlib
\definecolor{PythonBlue}{RGB}{31, 119, 180}      % Azul principal matplotlib
\definecolor{LightPythonBlue}{RGB}{174, 199, 232} % Azul claro para fondos
\definecolor{DarkPythonBlue}{RGB}{23, 90, 135}    % Azul oscuro para líneas

% Colores para tema oscuro tipo VS Code/Colab
\definecolor{DarkBackground}{RGB}{40, 44, 52}     % Fondo oscuro similar a VS Code
\definecolor{CodeText}{RGB}{171, 178, 191}        % Texto gris claro para código
\definecolor{DarkFrame}{RGB}{60, 63, 65}          % Color del marco/borde

% Colores para tema claro
\definecolor{LightBackground}{RGB}{248, 248, 242} % Fondo claro tipo GitHub
\definecolor{LightCodeText}{RGB}{51, 51, 51}      % Texto oscuro para tema claro
\definecolor{LightFrame}{RGB}{220, 220, 220}      % Marco gris claro

\usepackage{minted}

%%%%%%%%%%%%%%%%%%%%%%%%%%%%%%%%%%%%%%%%%%%%%%%%%%%%%%%%%%%%%%%%%%%%%%%%%%%%%%%%%%%%%%
% CONFIGURACIÓN DE TEMAS - CAMBIA AQUÍ PARA ALTERNAR
% ========================================================================
% Descomenta UNA de las siguientes líneas para elegir el tema:

% TEMA OSCURO (VS Code style)
%\newcommand{\mytheme}{oscuro}
%\usemintedstyle{monokai}
%\newcommand{\mybgcolor}{DarkBackground}

% TEMA CLARO (GitHub style) - comenta las 3 líneas de arriba y descomenta estas:
 \newcommand{\mytheme}{claro}
 \usemintedstyle{default}
 \newcommand{\mybgcolor}{LightBackground}

% ========================================================================
\newcommand{\mypyfile}[1]{\inputminted[linenos=true, fontsize=\footnotesize, frame=lines, framesep=5\fboxrule,framerule=1pt]{python}{#1}}

% Comando personalizado para código Python inline
\newcommand{\pycode}[1]{\begin{minted}{python}#1\end{minted}}

% ========================================================================
% COMANDOS ESPECÍFICOS PARA CADA TEMA
% ========================================================================
% Comando para tema oscuro específico
\newcommand{\pyoscuro}[1]{%
\begin{minted}[
  bgcolor=DarkBackground,
  style=monokai,
  breaklines,
  frame=lines,
  framesep=2mm,
  baselinestretch=1.2,
  linenos,
  fontsize=\footnotesize
]{python}
#1
\end{minted}
}

% Comando para tema claro específico
\newcommand{\pyclaro}[1]{%
\begin{minted}[
  bgcolor=LightBackground,
  style=default,
  breaklines,
  frame=lines,
  framesep=2mm,
  baselinestretch=1.2,
  linenos,
  fontsize=\footnotesize
]{python}
#1
\end{minted}
}
%%%%%%%%%%%%%%%%%%%%%%%%%%%%%%%%%%%%%%%%%%%%%%%%%%%%%%%%%%%%%%%%%%%%%%%%%%%%%%%%%%%%%%



%%%%%%%%%%%%%%%%%%%%%%%%%%%%%%%%%%%%%%%%%%%%%%%%%%%%%%%%%%%%%%%%%%%%%%%%%%%%%%%%%%%%%%
% Configuración global para todos los bloques minted de Python
% Usa automáticamente el tema seleccionado arriba
\setminted[python]{
  breaklines,
  frame=lines,
  framesep=2mm,
  baselinestretch=1.2,
  bgcolor=\mybgcolor,
  linenos, 
  fontsize=\footnotesize
} % Configuración dinámica según tema elegido
%%%%%%%%%%%%%%%%%%%%%%%%%%%%%%%%%%%%%%%%%%%%%%%%%%%%%%%%%%%%%%%%%%%%%%%%%%%%%%%%%%%%%%



%%%%%%%%%%%%%%%%%%%%%%%%%%%%%%%%%%%%%%%%%%%%%%%%%%%%%%%%%%%%%%%%%%%%%%%%%%%%%%%%%%%%%%
% Configuración de colores del tema con paleta azul Python
\setbeamercolor{progress bar}{fg=DarkPythonBlue,bg=LightPythonBlue}
\setbeamercolor{title separator}{fg=DarkPythonBlue,bg=white!50!black}
\setbeamercolor{frametitle}{fg=white,bg=PythonBlue}
\title[PCFI161]{Programaci\'on para F\'isica y Astronom\'ia}
\subtitle{Departamento de Física.}

\newcommand{\myfront}{
\author[PCFI161]{Corodinadora: C Loyola \\ Profesores C Femenías / F Bugini / D Basantes}
\institute[UNAB]{Universidad Andrés Bello \\ Departamento de Física y Astronomía}
\date{Primer Semestre 2025}
}

\titlegraphic{
  \includegraphics[width=.08\textwidth]{1) Logo/logo-tux.png}\hfill
  \includegraphics[width=.3\textwidth]{1) Logo/logo-unab.png}\hfill
  \includegraphics[width=.08\textwidth]{1) Logo/logo-python.png}
}

\makeatletter
\setbeamertemplate{title page}{
  \begin{minipage}[b][\paperheight]{\textwidth}
    \vfill%
    \ifx\inserttitle\@empty\else\usebeamertemplate*{title}\fi
    \ifx\insertsubtitle\@empty\else\usebeamertemplate*{subtitle}\fi
    \usebeamertemplate*{title separator}
    \ifx\beamer@shortauthor\@empty\else\usebeamertemplate*{author}\fi
    \ifx\insertdate\@empty\else\usebeamertemplate*{date}\fi
    \ifx\insertinstitute\@empty\else\usebeamertemplate*{institute}\fi
    \vfill
    \ifx\inserttitlegraphic\@empty\else\inserttitlegraphic\fi
    \vspace*{1cm}
  \end{minipage}
}
\makeatother

% Configuración personalizada del frametitle para incluir la sección
\makeatletter
\setbeamertemplate{frametitle}{%
  \nointerlineskip%
  \begin{beamercolorbox}[%
      wd=\paperwidth,%
      sep=0pt,%
      leftskip=\@ifundefined{metropolis@frametitle@padding}{12pt}{\metropolis@frametitle@padding},%
      rightskip=\@ifundefined{metropolis@frametitle@padding}{12pt}{\metropolis@frametitle@padding},%
    ]{frametitle}%
  \@ifundefined{metropolis@frametitlestrut@start}{}{\metropolis@frametitlestrut@start}%
  {\textcolor{LightPythonBlue}{\insertsectionhead}\ifx\insertsectionhead\@empty\else\ $\ni$  \fi}\insertframetitle%
  \nolinebreak%
  \@ifundefined{metropolis@frametitlestrut@end}{}{\metropolis@frametitlestrut@end}%
  \end{beamercolorbox}%
}
\makeatother

\makeatletter
% Configuración segura de longitudes del tema Metropolis
\@ifundefined{metropolis@titleseparator@linewidth}{}{\setlength{\metropolis@titleseparator@linewidth}{2pt}}
\@ifundefined{metropolis@progressonsectionpage@linewidth}{}{\setlength{\metropolis@progressonsectionpage@linewidth}{2pt}}
\@ifundefined{metropolis@progressinheadfoot@linewidth}{}{\setlength{\metropolis@progressinheadfoot@linewidth}{2pt}}
\makeatother



\begin{document}

% ------------------------------------------------------------------------
% Portada personalizada (puedes usar \myfront si está definido en preamble.tex)
% ------------------------------------------------------------------------
\myfront{}

% ------------------------------------------------------------------------
% Slide 1: Título de la Sesión
% ------------------------------------------------------------------------
\begin{frame}
  \titlepage
  % Ejemplo:
  % \title{Semana 3 - Sesión 1 (Sesión 5): Estructuras de Control Fundamentales}
\end{frame}

% ------------------------------------------------------------------------
% Slide 2: Índice / Tabla de contenidos
% ------------------------------------------------------------------------
\begin{frame}
  \frametitle{Resumen - Semana 3, Sesión 1 (Sesión 5)}
  \tableofcontents
\end{frame}

% ------------------------------------------------------------------------
% Configuración de bloques
% ------------------------------------------------------------------------
\metroset{block=fill}

% ----------------------------------------------------------------------------------------
% SECCIÓN 1: Conexión con la Sesión Anterior
% ----------------------------------------------------------------------------------------
\section{Introducción y Repaso}

% ------------------------------------------------------------------------
% Slide 3: Recapitulación de la Sesión 4
% ------------------------------------------------------------------------
\begin{frame}{Recapitulación de la Sesión Anterior}
  \begin{itemize}
    \item \textbf{Semana 2, Sesión 2 (Sesión 4)} se enfocó en:
      \begin{itemize}
        \item Integrar y aplicar conceptos básicos de Python en problemas más elaborados.
        \item Ejercicios prácticos sobre ecuaciones de movimiento, varianza y conversiones de unidades.
        \item Manejo de pequeños bucles \texttt{for} y condicionales sencillos \texttt{if/elif}.
      \end{itemize}
    \item \textbf{Meta de hoy}: Introducir y profundizar en las estructuras de control fundamentales (condicionales y bucles).
  \end{itemize}
\end{frame}

% ------------------------------------------------------------------------
% Slide 4: Objetivos de la Sesión 5
% ------------------------------------------------------------------------
\begin{frame}{Objetivos de la Sesión 5}
  \begin{itemize}
    \item \textbf{Comprender} la sintaxis y el uso de las estructuras de control en Python (\texttt{if}, \texttt{elif}, \texttt{else} y \texttt{while}).
    \item \textbf{Analizar} ejemplos prácticos que requieran decisiones condicionales y repeticiones.
    \item \textbf{Diseñar} programas sencillos que apliquen bucles \texttt{while} con criterio de parada.
    \item \textbf{Relacionar} estas estructuras con problemas físicos o astronómicos básicos.
  \end{itemize}
\end{frame}

% ----------------------------------------------------------------------------------------
% SECCIÓN 2: Estructuras de Control (Condicionales)
% ----------------------------------------------------------------------------------------
\section{Estructuras de Control: Condicionales}

% ------------------------------------------------------------------------
% Slide 5: Condicionales \texttt{if}, \texttt{elif}, \texttt{else}
% ------------------------------------------------------------------------
\begin{frame}[fragile]
	\frametitle{La Estructura \texttt{if}, \texttt{elif}, \texttt{else}}
  \begin{itemize}
    \item Permiten ejecutar un bloque de código si se cumple una condición.
    \item \textbf{Sintaxis básica}:

\texttt{if <condicion1>:} \\
\quad \texttt{<bloque1>} \\
\texttt{elif <condicion2>:} \\
\quad \texttt{<bloque2>} \\
\texttt{else:} \\
\quad \texttt{<bloque3>}

    \item Se pueden tener \emph{múltiples} \texttt{elif} y \emph{opcionalmente} un \texttt{else}.
  \end{itemize}
\end{frame}

% ------------------------------------------------------------------------
% Slide 6: Ejemplo de Uso (Nota Aprobación)
% ------------------------------------------------------------------------
\begin{frame}[fragile]{Ejemplo: Determinar si una Nota Aprueba o No}
\begin{minted}[
frame=lines,
framesep=2mm,
baselinestretch=1.1,
bgcolor=LightGray,
fontsize=\footnotesize
]{python}
nota_str = input("Ingresa tu nota (0.0 - 7.0): ")
nota = float(nota_str)

if nota >= 4.0:
    print("Aprobado")
else:
    print("Reprobado")
\end{minted}
\textbf{Discusión}:
\begin{itemize}
  \item ¿Qué pasa si la nota está fuera del rango esperado?
  \item ¿Podríamos manejar \texttt{elif} para rangos (ej. “sobresaliente”, “suficiente”, etc.)?
\end{itemize}
\end{frame}

% ------------------------------------------------------------------------
% Slide 7: Actividad 1 - Menú Sencillo
% ------------------------------------------------------------------------
\begin{frame}{Actividad 1: Crear un Menú Sencillo}
  \begin{block}{Enunciado}
    \begin{itemize}
      \item Pedir al usuario una opción entre:
        \begin{itemize}
          \item \texttt{1} - Calcular el área de un cuadrado.
          \item \texttt{2} - Calcular el área de un triángulo.
          \item \texttt{3} - Salir.
        \end{itemize}
      \item Usar \textbf{if/elif/else} para procesar la opción.
      \item Si se elige \texttt{1} o \texttt{2}, pedir dimensiones, calcular y mostrar el resultado.
      \item Si se elige \texttt{3}, finalizar el programa.
    \end{itemize}
  \end{block}
  \textbf{Consejo}: Manejar opciones inválidas (mensaje de error).
\end{frame}

% ----------------------------------------------------------------------------------------
% SECCIÓN 3: Estructuras de Control: Bucles \texttt{while}
% ----------------------------------------------------------------------------------------
\section{Estructuras de Control: Bucles While}

% ------------------------------------------------------------------------
% Slide 8: El Bucle \texttt{while}
% ------------------------------------------------------------------------
\begin{frame}{El Bucle \texttt{while}}
  \begin{itemize}
    \item Repite un bloque de código \textbf{mientras} una condición sea verdadera.
    \item \textbf{Sintaxis}:

\texttt{while <condicion>:} \\
\quad \texttt{<bloque>}

    \item Importante asegurarse de que la condición \textbf{cambie} para evitar bucles infinitos.
  \end{itemize}
\end{frame}

% ------------------------------------------------------------------------
% Slide 9: Ejemplo de Uso
% ------------------------------------------------------------------------
\begin{frame}[fragile]{Ejemplo: Contador Descendente}
\begin{minted}[
frame=lines,
framesep=2mm,
baselinestretch=1.1,
bgcolor=LightGray,
fontsize=\footnotesize
]{python}
count = 5
while count > 0:
    print("Cuenta atrás:", count)
    count -= 1

print("¡Fin de la cuenta!")
\end{minted}
\textbf{Análisis}:
\begin{itemize}
  \item \texttt{count -= 1} evita que el bucle sea infinito.
  \item Se ejecuta el bloque mientras \texttt{count > 0}.
\end{itemize}
\end{frame}

% ------------------------------------------------------------------------
% Slide 10: Actividad 2 - Suma Iterativa
% ------------------------------------------------------------------------
\begin{frame}{Actividad 2: Suma Iterativa}
  \begin{block}{Enunciado}
    \begin{itemize}
      \item Pide repetidamente al usuario ingresar un número.
      \item Suma todos los números ingresados.
      \item Si el usuario ingresa \texttt{0}, terminar el bucle y mostrar la suma final.
    \end{itemize}
  \end{block}
  \textbf{Tip}: Usa un \texttt{while True} y \texttt{break} al detectar 0, o controla la condición en \texttt{while}.
\end{frame}

% ----------------------------------------------------------------------------------------
% SECCIÓN 4: Ejemplos Prácticos con \texttt{if} y \texttt{while}
% ----------------------------------------------------------------------------------------
\section{Ejemplos Prácticos}

% ------------------------------------------------------------------------
% Slide 11: Ejemplo 1 - Búsqueda de Raíz Sencilla (Iterativo)
% ------------------------------------------------------------------------
\begin{frame}{Ejemplo 1: Búsqueda de Raíz (Método Ingenuo)}
  \begin{itemize}
    \item \textbf{Problema}: Encontrar \(n\) tal que \(n^2 \approx m\) para un \(m\) dado.
    \item \textbf{Idea}:
      \begin{itemize}
        \item Partir de \(n=0\).
        \item Incrementar \(n\) en 1 hasta que \(n^2 \ge m\).
        \item Al final, \(n\) está cerca de \(\sqrt{m}\).
      \end{itemize}
    \item \textbf{Uso}: \texttt{while} para la búsqueda.
    \item Posteriormente: Mejorar con métodos numéricos (Newton-Raphson).
  \end{itemize}
\end{frame}

% ------------------------------------------------------------------------
% Slide 12: Código Ejemplo - Búsqueda de Raíz
% ------------------------------------------------------------------------
\begin{frame}[fragile]{Código: Búsqueda Sencilla}
\begin{minted}[
frame=lines,
framesep=2mm,
baselinestretch=1.1,
bgcolor=LightGray,
fontsize=\footnotesize
]{python}
m_str = input("Ingresa un número (m): ")
m = float(m_str)

n = 0
while (n**2) < m:
    n += 1

print("Resultado aproximado:", n)
print("n^2 =", n**2, ", m =", m)
\end{minted}
\textbf{Discusión}:
\begin{itemize}
  \item Exactitud vs. tiempo de ejecución.
  \item Controlar \texttt{n} o implementar un \texttt{break} si \texttt{n} se vuelve muy grande.
\end{itemize}
\end{frame}

% ------------------------------------------------------------------------
% Slide 13: Ejemplo 2 - Verificación de Input
% ------------------------------------------------------------------------
\begin{frame}[fragile]{Ejemplo 2: Verificación de Input}
\begin{minted}[
frame=lines,
framesep=2mm,
baselinestretch=1.1,
bgcolor=LightGray,
fontsize=\footnotesize
]{python}
clave_correcta = "astronomia"
intentos = 3

while intentos > 0:
    clave = input("Ingresa la clave: ")
    if clave == clave_correcta:
        print("¡Bienvenido/a!")
        break
    else:
        intentos -= 1
        print("Clave incorrecta. Intentos restantes:", intentos)

if intentos == 0:
    print("Has agotado todos los intentos. Acceso denegado.")
\end{minted}
\textbf{Comentario}:
\begin{itemize}
  \item Ejemplo de \texttt{if} + \texttt{while} para autenticar con un número limitado de intentos.
\end{itemize}
\end{frame}

% ------------------------------------------------------------------------
% Slide 14: Actividad 3 - Menú Interactivo con \texttt{while}
% ------------------------------------------------------------------------
\begin{frame}{Actividad 3: Menú Interactivo con \texttt{while}}
  \begin{block}{Enunciado}
    \begin{itemize}
      \item Repetir un menú hasta que el usuario seleccione la opción \texttt{"Salir"}.
      \item Opciones posibles:
        \begin{itemize}
          \item (1) Calcular área de círculo.
          \item (2) Calcular energía potencial: \(E_p = mgh\).
          \item (3) Salir.
        \end{itemize}
      \item Usa \texttt{if/elif/else} dentro de un \texttt{while True} y un \texttt{break} cuando sea \texttt{Salir}.
    \end{itemize}
  \end{block}
  \textbf{Tip}: Cada operación requiere datos distintos (masa, altura, radio, etc.).
\end{frame}

% ------------------------------------------------------------------------
% Slide 15: Organización del Trabajo en Clase
% ------------------------------------------------------------------------
\begin{frame}{Trabajar en Parejas o Grupos Pequeños}
  \begin{itemize}
    \item \textbf{Formen equipos} de 2-3 personas.
    \item \textbf{Objetivo}: Implementar el menú iterativo con \texttt{while}.
    \item \textbf{Desafío}: Agregar validaciones (por ej. valores negativos en masa/altura).
    \item Comparar soluciones y resaltar las diferencias de implementación.
  \end{itemize}
\end{frame}

% ------------------------------------------------------------------------
% Slide 16: Discusión Grupal y Retroalimentación
% ------------------------------------------------------------------------
\begin{frame}{Discusión y Retroalimentación}
  \begin{itemize}
    \item ¿Surgieron bucles infinitos? ¿Cómo se detectaron y resolvieron?
    \item ¿Alguna validación extra para datos imposibles o nulos?
    \item \textbf{Buenas prácticas}: Comentar el código y usar nombres de variables descriptivos.
  \end{itemize}
\end{frame}

% ------------------------------------------------------------------------
% Slide 17: Código Ejemplo (Menú Interactivo)
% ------------------------------------------------------------------------
\begin{frame}[fragile]{Ejemplo de Solución para el Menú Interactivo}
\begin{minted}[
frame=lines,
framesep=2mm,
baselinestretch=1.1,
bgcolor=LightGray,
fontsize=\tiny
]{python}
import math

while True:
    print("=== MENÚ ===")
    print("(1) Área de círculo")
    print("(2) Energía potencial (mgh)")
    print("(3) Salir")
    opcion = input("Elige una opción: ")

    if opcion == "1":
        r_str = input("Radio del círculo (m): ")
        r = float(r_str)
        area = math.pi * r**2
        print("Área =", area, "m^2")
    elif opcion == "2":
        m_str = input("Masa (kg): ")
        h_str = input("Altura (m): ")
        g = 9.8
        m = float(m_str)
        h = float(h_str)
        Ep = m * g * h
        print("Energía potencial =", Ep, "J")
    elif opcion == "3":
        print("Saliendo...")
        break
    else:
        print("Opción inválida, intenta de nuevo.")
\end{minted}
\end{frame}

% ------------------------------------------------------------------------
% Slide 18: Análisis de la Solución
% ------------------------------------------------------------------------
\begin{frame}{Análisis de la Solución}
  \begin{itemize}
    \item Uso de \textbf{while True} + \texttt{break} para controlar el flujo.
    \item Estructura \textbf{if/elif/else} para manejar opciones.
    \item Se separan las variables y cálculos de cada opción.
    \item \textbf{Posible mejora}: Manejo de errores al convertir \texttt{str} a \texttt{float}.
  \end{itemize}
\end{frame}

% ------------------------------------------------------------------------
% Slide 19: Ejercicio Adicional - Adivinar un Número
% ------------------------------------------------------------------------
\begin{frame}{Ejercicio Adicional (Adivinar un Número)}
  \begin{block}{Enunciado}
    \begin{itemize}
      \item El programa genera un número al azar entre 1 y 10.
      \item El usuario intenta adivinar el número ingresando valores.
      \item Indicar si el número es \textbf{mayor}, \textbf{menor} o \textbf{igual}.
      \item Terminar cuando el usuario acierte o supere los 5 intentos.
    \end{itemize}
  \end{block}
  \textbf{Tip}: Usa \texttt{import random} y \texttt{random.randint(1,10)} para generar el número.
\end{frame}

% ------------------------------------------------------------------------
% Slide 20: Trabajo en Solitario o con un Compañero
% ------------------------------------------------------------------------
\begin{frame}{Trabajo Individual o en Pareja}
  \begin{itemize}
    \item Implementa el juego de \textbf{Adivinar un Número} en un \texttt{notebook} de Colab.
    \item Añade mensajes claros para el usuario (“\texttt{El número es menor}” / “\texttt{El número es mayor}”).
    \item Verifica que se detenga en 5 intentos (o en el acierto).
    \item Al terminar, muestra cuántos intentos usó el jugador.
  \end{itemize}
\end{frame}

% ------------------------------------------------------------------------
% Slide 21: Retroalimentación y Observaciones
% ------------------------------------------------------------------------
\begin{frame}{Retroalimentación}
  \begin{itemize}
    \item ¿Lograste implementar la lógica de \texttt{if} dentro de un \texttt{while} sin errores?
    \item ¿Qué ocurrió cuando el usuario ingresa valores fuera del rango?
    \item \textbf{Extensión}: Permitir varios rangos de adivinanza o puntajes.
  \end{itemize}
\end{frame}

% ------------------------------------------------------------------------
% Slide 22: Conexión con Otros Problemas
% ------------------------------------------------------------------------
\begin{frame}{Conexión con Otros Problemas}
  \begin{itemize}
    \item \textbf{Simulaciones en Física}: Muchas implican decisiones (condiciones de borde, colisión, etc.).
    \item \textbf{Procesamiento de Datos}: Filtrar registros según criterios (\texttt{if} para descartar outliers).
    \item \textbf{Automatización}: Scripts que se repiten hasta cumplir una condición.
  \end{itemize}
  \textbf{Conclusión}: Las estructuras de control son la base de la lógica en programación.
\end{frame}

% ------------------------------------------------------------------------
% Slide 23: Recursos y Lecturas Recomendadas
% ------------------------------------------------------------------------
\begin{frame}{Recursos y Lecturas}
  \begin{itemize}
    \item \href{https://docs.python.org/3/tutorial/controlflow.html}{\textbf{Python Docs: Control Flow}}.
    \item \href{https://www.w3schools.com/python/python\_conditions.asp}{\textbf{W3Schools: Python Conditions}}.
    \item \href{https://automatetheboringstuff.com/}{\textbf{Automate the Boring Stuff}} (capítulos sobre \texttt{if} y \texttt{while}).
  \end{itemize}
\end{frame}

% ------------------------------------------------------------------------
% Slide 24: Sugerencias de Práctica
% ------------------------------------------------------------------------
\begin{frame}{Sugerencias de Práctica}
  \begin{itemize}
    \item \textbf{Combina} estructuras de control: Un \texttt{while} que usa \texttt{if/elif/else} internamente.
    \item \textbf{Gestiona} valores inválidos con \texttt{try/except} (si deseas profundizar en manejo de errores).
    \item \textbf{Experimenta} con ejemplos de la vida real (cálculo de impuestos, clasificación según edad, etc.).
  \end{itemize}
\end{frame}

% ------------------------------------------------------------------------
% Slide 25: Cierre de la Sesión
% ------------------------------------------------------------------------
\begin{frame}
  \huge{\centerline{¡Gracias y hasta la próxima sesión!}}
  \vspace{0.3cm}
  \normalsize
  \begin{itemize}
    \item Continúen practicando \texttt{if}, \texttt{elif}, \texttt{else}, \texttt{while}.
    \item En la siguiente sesión, profundizaremos en \texttt{for} y prácticas de control de flujo (break, continue).
    \item ¡No olviden guardar sus notebooks en Drive!
  \end{itemize}
\end{frame}

\end{document}

