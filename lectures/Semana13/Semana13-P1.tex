\documentclass[10pt]{beamer}

% ------------------------------------------------------------------------
% Preambulo personalizado (colores, comandos, logo, etc.)
% ------------------------------------------------------------------------
\usetheme[progressbar=frametitle]{metropolis}
\usepackage{appendixnumberbeamer}
\usepackage{fancyvrb}
\usepackage{booktabs}
\usepackage[scale=2]{ccicons}
\usepackage{pgfplots}
\usepgfplotslibrary{dateplot}
\usepackage{type1cm}
\usepackage{lettrine}
\usepackage{ragged2e}
\usepackage{xspace}
\newcommand{\themename}{\textbf{\textsc{metropolis}}\xspace}
\usepackage{graphicx} % Allows including images
\usepackage{booktabs} % Allows the use of \toprule, \midrule and \bottomrule in tables
\usepackage[utf8]{inputenc} %solucion del problema de los acentos.
\usepackage{xcolor}
\usepackage{verbatim}

\definecolor{LightGray}{gray}{0.9}
% Paleta de azules estilo Python/matplotlib
\definecolor{PythonBlue}{RGB}{31, 119, 180}      % Azul principal matplotlib
\definecolor{LightPythonBlue}{RGB}{174, 199, 232} % Azul claro para fondos
\definecolor{DarkPythonBlue}{RGB}{23, 90, 135}    % Azul oscuro para líneas

% Colores para tema oscuro tipo VS Code/Colab
\definecolor{DarkBackground}{RGB}{40, 44, 52}     % Fondo oscuro similar a VS Code
\definecolor{CodeText}{RGB}{171, 178, 191}        % Texto gris claro para código
\definecolor{DarkFrame}{RGB}{60, 63, 65}          % Color del marco/borde

% Colores para tema claro
\definecolor{LightBackground}{RGB}{248, 248, 242} % Fondo claro tipo GitHub
\definecolor{LightCodeText}{RGB}{51, 51, 51}      % Texto oscuro para tema claro
\definecolor{LightFrame}{RGB}{220, 220, 220}      % Marco gris claro

\usepackage{minted}

%%%%%%%%%%%%%%%%%%%%%%%%%%%%%%%%%%%%%%%%%%%%%%%%%%%%%%%%%%%%%%%%%%%%%%%%%%%%%%%%%%%%%%
% CONFIGURACIÓN DE TEMAS - CAMBIA AQUÍ PARA ALTERNAR
% ========================================================================
% Descomenta UNA de las siguientes líneas para elegir el tema:

% TEMA OSCURO (VS Code style)
%\newcommand{\mytheme}{oscuro}
%\usemintedstyle{monokai}
%\newcommand{\mybgcolor}{DarkBackground}

% TEMA CLARO (GitHub style) - comenta las 3 líneas de arriba y descomenta estas:
 \newcommand{\mytheme}{claro}
 \usemintedstyle{default}
 \newcommand{\mybgcolor}{LightBackground}

% ========================================================================
\newcommand{\mypyfile}[1]{\inputminted[linenos=true, fontsize=\footnotesize, frame=lines, framesep=5\fboxrule,framerule=1pt]{python}{#1}}

% Comando personalizado para código Python inline
\newcommand{\pycode}[1]{\begin{minted}{python}#1\end{minted}}

% ========================================================================
% COMANDOS ESPECÍFICOS PARA CADA TEMA
% ========================================================================
% Comando para tema oscuro específico
\newcommand{\pyoscuro}[1]{%
\begin{minted}[
  bgcolor=DarkBackground,
  style=monokai,
  breaklines,
  frame=lines,
  framesep=2mm,
  baselinestretch=1.2,
  linenos,
  fontsize=\footnotesize
]{python}
#1
\end{minted}
}

% Comando para tema claro específico
\newcommand{\pyclaro}[1]{%
\begin{minted}[
  bgcolor=LightBackground,
  style=default,
  breaklines,
  frame=lines,
  framesep=2mm,
  baselinestretch=1.2,
  linenos,
  fontsize=\footnotesize
]{python}
#1
\end{minted}
}
%%%%%%%%%%%%%%%%%%%%%%%%%%%%%%%%%%%%%%%%%%%%%%%%%%%%%%%%%%%%%%%%%%%%%%%%%%%%%%%%%%%%%%



%%%%%%%%%%%%%%%%%%%%%%%%%%%%%%%%%%%%%%%%%%%%%%%%%%%%%%%%%%%%%%%%%%%%%%%%%%%%%%%%%%%%%%
% Configuración global para todos los bloques minted de Python
% Usa automáticamente el tema seleccionado arriba
\setminted[python]{
  breaklines,
  frame=lines,
  framesep=2mm,
  baselinestretch=1.2,
  bgcolor=\mybgcolor,
  linenos, 
  fontsize=\footnotesize
} % Configuración dinámica según tema elegido
%%%%%%%%%%%%%%%%%%%%%%%%%%%%%%%%%%%%%%%%%%%%%%%%%%%%%%%%%%%%%%%%%%%%%%%%%%%%%%%%%%%%%%



%%%%%%%%%%%%%%%%%%%%%%%%%%%%%%%%%%%%%%%%%%%%%%%%%%%%%%%%%%%%%%%%%%%%%%%%%%%%%%%%%%%%%%
% Configuración de colores del tema con paleta azul Python
\setbeamercolor{progress bar}{fg=DarkPythonBlue,bg=LightPythonBlue}
\setbeamercolor{title separator}{fg=DarkPythonBlue,bg=white!50!black}
\setbeamercolor{frametitle}{fg=white,bg=PythonBlue}
\title[PCFI161]{Programaci\'on para F\'isica y Astronom\'ia}
\subtitle{Departamento de Física.}

\newcommand{\myfront}{
\author[PCFI161]{Corodinadora: C Loyola \\ Profesores C Femenías / F Bugini / D Basantes}
\institute[UNAB]{Universidad Andrés Bello \\ Departamento de Física y Astronomía}
\date{Primer Semestre 2025}
}

\titlegraphic{
  \includegraphics[width=.08\textwidth]{1) Logo/logo-tux.png}\hfill
  \includegraphics[width=.3\textwidth]{1) Logo/logo-unab.png}\hfill
  \includegraphics[width=.08\textwidth]{1) Logo/logo-python.png}
}

\makeatletter
\setbeamertemplate{title page}{
  \begin{minipage}[b][\paperheight]{\textwidth}
    \vfill%
    \ifx\inserttitle\@empty\else\usebeamertemplate*{title}\fi
    \ifx\insertsubtitle\@empty\else\usebeamertemplate*{subtitle}\fi
    \usebeamertemplate*{title separator}
    \ifx\beamer@shortauthor\@empty\else\usebeamertemplate*{author}\fi
    \ifx\insertdate\@empty\else\usebeamertemplate*{date}\fi
    \ifx\insertinstitute\@empty\else\usebeamertemplate*{institute}\fi
    \vfill
    \ifx\inserttitlegraphic\@empty\else\inserttitlegraphic\fi
    \vspace*{1cm}
  \end{minipage}
}
\makeatother

% Configuración personalizada del frametitle para incluir la sección
\makeatletter
\setbeamertemplate{frametitle}{%
  \nointerlineskip%
  \begin{beamercolorbox}[%
      wd=\paperwidth,%
      sep=0pt,%
      leftskip=\@ifundefined{metropolis@frametitle@padding}{12pt}{\metropolis@frametitle@padding},%
      rightskip=\@ifundefined{metropolis@frametitle@padding}{12pt}{\metropolis@frametitle@padding},%
    ]{frametitle}%
  \@ifundefined{metropolis@frametitlestrut@start}{}{\metropolis@frametitlestrut@start}%
  {\textcolor{LightPythonBlue}{\insertsectionhead}\ifx\insertsectionhead\@empty\else\ $\ni$  \fi}\insertframetitle%
  \nolinebreak%
  \@ifundefined{metropolis@frametitlestrut@end}{}{\metropolis@frametitlestrut@end}%
  \end{beamercolorbox}%
}
\makeatother

\makeatletter
% Configuración segura de longitudes del tema Metropolis
\@ifundefined{metropolis@titleseparator@linewidth}{}{\setlength{\metropolis@titleseparator@linewidth}{2pt}}
\@ifundefined{metropolis@progressonsectionpage@linewidth}{}{\setlength{\metropolis@progressonsectionpage@linewidth}{2pt}}
\@ifundefined{metropolis@progressinheadfoot@linewidth}{}{\setlength{\metropolis@progressinheadfoot@linewidth}{2pt}}
\makeatother



% ----------  minted (igual que S12-P2) ----------
\usepackage{minted}          % Requiere compilar con -shell-escape
\setminted{
  fontsize   = \scriptsize,
  linenos,
  baselinestretch = 1,
  breaklines
}

\title{S13 -- Sesión 25\\Ejercicios integradores antes de la Solemne 2}
\author{PCFI161 -- Programación para Física y Astronomía}
\date{26 mayo 2025}

\begin{document}
\myfront{}

% ──────────────────────────────────────────────────────────
\begin{frame}
  \titlepage
\end{frame}

\begin{frame}
  \frametitle{Ruta de la sesión (90 min)}
  \tableofcontents
\end{frame}

\metroset{block=fill}

% ============================================================
\section{Warm-up: ciclos \texttt{for} (20 min)}
% ============================================================

\begin{frame}[fragile]{Ejercicio 1 (10 min) – Caída libre discreta}
  \begin{block}{Enunciado}
    \begin{itemize}
      \item Parte un balón desde \(h_0=20\,\mathrm{m}\) con \(v_0=0\).
      \item Crea un \textbf{ciclo \texttt{for}} con paso \(\Delta t = 0.2\,\mathrm{s}\).
      \item En cada iteración imprime \(t,\,h,\,v\) hasta que \(h \le 0\).
      \item Verifica que la altura nunca es negativa.
    \end{itemize}
  \end{block}
  \begin{minted}{python}
g = 9.8        # m/s^2
h, v = 20.0, 0.0
dt = 0.2
for step in range(100):        # suficiente margen
    t = step * dt
    print(f"{t:4.1f} s  h={h:5.2f} m  v={v:5.2f} m/s")
    if h <= 0:
        break
    v = v - g*dt
    h = h + v*dt
  \end{minted}
\end{frame}

\begin{frame}[fragile]{Ejercicio 2 (10 min) – Fibonacci y proporción áurea}
  \begin{block}{Tarea}
    \begin{enumerate}
      \item Con un \texttt{for} genera los primeros 15 términos de Fibonacci.
      \item Durante el bucle imprime la razón \(F_{n+1}/F_n\).
      \item ¿A qué número converge la razón? (≈ 1.618)
    \end{enumerate}
  \end{block}
  \begin{minted}{python}
a, b = 1, 1
for n in range(15):
    print(f'F{n+1}={a},  razón={b/a:.5f}')
    a, b = b, a+b
  \end{minted}
\end{frame}

% ============================================================
\section{NumPy y estadística (30 min)}
% ============================================================

\begin{frame}[fragile]{Ejercicio 3 (15 min) -- Órbita circular con arrays}
\begin{block}{Pasos}
  \begin{enumerate}
    \item Usa \texttt{np.linspace(0,\,2\(\pi\),\,360, endpoint=False)} para 360 ángulos.
    \item Calcula \(\displaystyle x=r\cos\theta,\;y=r\sin\theta\) con \(r=1\).
    \item Muestra el promedio \(\bar x,\bar y\) y discute por qué se acerca a \(0,0\).
  \end{enumerate}
\end{block}
\begin{minted}{python}
import numpy as np
theta = np.linspace(0, 2*np.pi, 360, endpoint=False)
x = np.cos(theta);  y = np.sin(theta)
print(" <x> =", x.mean(), " <y> =", y.mean())
\end{minted}
\end{frame}

\begin{frame}[fragile]{Ejercicio 4 (15 min) – Temperaturas estelares simuladas}
\begin{block}{Enunciado}
  \begin{itemize}
    \item Genera 200 temperaturas: \texttt{T = np.random.normal(6000, 300, 200)}.
    \item Con NumPy calcula: media, desvío estándar, percentiles 25/75.
    \item ¿Cuántas estrellas caen fuera del rango \(\mu \pm 3\sigma\)?
  \end{itemize}
\end{block}
\begin{minted}{python}
T = np.random.normal(6000, 300, 200)
mu, sigma = T.mean(), T.std()
lo, hi = mu - 3*sigma, mu + 3*sigma
outliers = ((T < lo) | (T > hi)).sum()
print(f"{outliers} estrellas fuera de +- 3 sigma")
\end{minted}
\end{frame}

% ============================================================
\section{Visualización con Matplotlib (15 min)}
% ============================================================

\begin{frame}[fragile]{Ejercicio 5 (15 min) – Ley de Kepler (mini versión)}
\begin{block}{Datos de planetas interiores}
\begin{minted}{python}
import numpy as np, matplotlib.pyplot as plt
a = np.array([0.39, 0.72, 1.00, 1.52])   # semieje mayor (AU)
T = np.array([0.24, 0.61, 1.00, 1.88])   # periodo (años)
\end{minted}
\end{block}
\begin{enumerate}
  \item Dibuja un \textbf{scatter} de \(a\) vs \(T\).
  \item Ajusta \(\log_{10}T = m\,\log_{10}a + b\) con \texttt{np.polyfit}.
  \item Grafica la recta sobre los datos en escala log–log.
  \item ¿El exponente \(m\) se acerca a 1.5?
\end{enumerate}
\end{frame}

% ============================================================
\section{pandas y datos reales (20 min)}
% ============================================================

\begin{frame}[fragile]{Ejercicio 6 (15 min) – Mini catálogo HIPPARCOS}
\begin{block}{Archivo a usar}
\url{https://gitarra.cl/lectures/gfiles/-/raw/main/pcfi161/S12/stars_brightness.csv}
\end{block}
\begin{enumerate}
  \item Lee el CSV en un \texttt{DataFrame}.
  \item Muestra \texttt{head()} y \texttt{describe()}.
  \item Agrupa por \texttt{spectral\_class}:  
        \(\displaystyle \text{mean}(T_K)\) y mínima magnitud aparente.
  \item Concluye en 2 líneas qué clase es más caliente y brillante.
\end{enumerate}
\end{frame}

\begin{frame}[fragile]{Ejercicio 7 (opc., 10 min) – Filtrar exoplanetas}
\begin{itemize}
  \item Se entregará un CSV muy pequeño con columnas  
        \texttt{name}, \texttt{mass\_Mj}, \texttt{period\_d}.
  \item Filtra planetas con \(m < 1.5\,M_J\) y periodo < 10 d.
  \item Guarda el resultado en \texttt{exoplanets\_short.csv}.
\end{itemize}
\end{frame}

% ============================================================
\section{Cierre}
% ============================================================

\begin{frame}{Próximos pasos}
\begin{itemize}
  \item Repasen estos problemas: conceptos clave para la Solemne 2.
  \item Dudas $\longrightarrow$ consultas al ayudante o profesor.
  \item ¡Éxito en la evaluación!
\end{itemize}
\end{frame}

\end{document}
