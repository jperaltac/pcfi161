\documentclass[10pt]{beamer}

% ------------------------------------------------------------------------
% Carga de tu preámbulo personalizado (preamble.tex).
% Recuerda colocarlo en la misma carpeta para que \input funcione.
% ------------------------------------------------------------------------
\usetheme[progressbar=frametitle]{metropolis}
\usepackage{appendixnumberbeamer}
\usepackage{fancyvrb}
\usepackage{booktabs}
\usepackage[scale=2]{ccicons}
\usepackage{pgfplots}
\usepgfplotslibrary{dateplot}
\usepackage{type1cm}
\usepackage{lettrine}
\usepackage{ragged2e}
\usepackage{xspace}
\newcommand{\themename}{\textbf{\textsc{metropolis}}\xspace}
\usepackage{graphicx} % Allows including images
\usepackage{booktabs} % Allows the use of \toprule, \midrule and \bottomrule in tables
\usepackage[utf8]{inputenc} %solucion del problema de los acentos.
\usepackage{xcolor}
\usepackage{verbatim}

\definecolor{LightGray}{gray}{0.9}
% Paleta de azules estilo Python/matplotlib
\definecolor{PythonBlue}{RGB}{31, 119, 180}      % Azul principal matplotlib
\definecolor{LightPythonBlue}{RGB}{174, 199, 232} % Azul claro para fondos
\definecolor{DarkPythonBlue}{RGB}{23, 90, 135}    % Azul oscuro para líneas

% Colores para tema oscuro tipo VS Code/Colab
\definecolor{DarkBackground}{RGB}{40, 44, 52}     % Fondo oscuro similar a VS Code
\definecolor{CodeText}{RGB}{171, 178, 191}        % Texto gris claro para código
\definecolor{DarkFrame}{RGB}{60, 63, 65}          % Color del marco/borde

% Colores para tema claro
\definecolor{LightBackground}{RGB}{248, 248, 242} % Fondo claro tipo GitHub
\definecolor{LightCodeText}{RGB}{51, 51, 51}      % Texto oscuro para tema claro
\definecolor{LightFrame}{RGB}{220, 220, 220}      % Marco gris claro

\usepackage{minted}

%%%%%%%%%%%%%%%%%%%%%%%%%%%%%%%%%%%%%%%%%%%%%%%%%%%%%%%%%%%%%%%%%%%%%%%%%%%%%%%%%%%%%%
% CONFIGURACIÓN DE TEMAS - CAMBIA AQUÍ PARA ALTERNAR
% ========================================================================
% Descomenta UNA de las siguientes líneas para elegir el tema:

% TEMA OSCURO (VS Code style)
%\newcommand{\mytheme}{oscuro}
%\usemintedstyle{monokai}
%\newcommand{\mybgcolor}{DarkBackground}

% TEMA CLARO (GitHub style) - comenta las 3 líneas de arriba y descomenta estas:
 \newcommand{\mytheme}{claro}
 \usemintedstyle{default}
 \newcommand{\mybgcolor}{LightBackground}

% ========================================================================
\newcommand{\mypyfile}[1]{\inputminted[linenos=true, fontsize=\footnotesize, frame=lines, framesep=5\fboxrule,framerule=1pt]{python}{#1}}

% Comando personalizado para código Python inline
\newcommand{\pycode}[1]{\begin{minted}{python}#1\end{minted}}

% ========================================================================
% COMANDOS ESPECÍFICOS PARA CADA TEMA
% ========================================================================
% Comando para tema oscuro específico
\newcommand{\pyoscuro}[1]{%
\begin{minted}[
  bgcolor=DarkBackground,
  style=monokai,
  breaklines,
  frame=lines,
  framesep=2mm,
  baselinestretch=1.2,
  linenos,
  fontsize=\footnotesize
]{python}
#1
\end{minted}
}

% Comando para tema claro específico
\newcommand{\pyclaro}[1]{%
\begin{minted}[
  bgcolor=LightBackground,
  style=default,
  breaklines,
  frame=lines,
  framesep=2mm,
  baselinestretch=1.2,
  linenos,
  fontsize=\footnotesize
]{python}
#1
\end{minted}
}
%%%%%%%%%%%%%%%%%%%%%%%%%%%%%%%%%%%%%%%%%%%%%%%%%%%%%%%%%%%%%%%%%%%%%%%%%%%%%%%%%%%%%%



%%%%%%%%%%%%%%%%%%%%%%%%%%%%%%%%%%%%%%%%%%%%%%%%%%%%%%%%%%%%%%%%%%%%%%%%%%%%%%%%%%%%%%
% Configuración global para todos los bloques minted de Python
% Usa automáticamente el tema seleccionado arriba
\setminted[python]{
  breaklines,
  frame=lines,
  framesep=2mm,
  baselinestretch=1.2,
  bgcolor=\mybgcolor,
  linenos, 
  fontsize=\footnotesize
} % Configuración dinámica según tema elegido
%%%%%%%%%%%%%%%%%%%%%%%%%%%%%%%%%%%%%%%%%%%%%%%%%%%%%%%%%%%%%%%%%%%%%%%%%%%%%%%%%%%%%%



%%%%%%%%%%%%%%%%%%%%%%%%%%%%%%%%%%%%%%%%%%%%%%%%%%%%%%%%%%%%%%%%%%%%%%%%%%%%%%%%%%%%%%
% Configuración de colores del tema con paleta azul Python
\setbeamercolor{progress bar}{fg=DarkPythonBlue,bg=LightPythonBlue}
\setbeamercolor{title separator}{fg=DarkPythonBlue,bg=white!50!black}
\setbeamercolor{frametitle}{fg=white,bg=PythonBlue}
\title[PCFI161]{Programaci\'on para F\'isica y Astronom\'ia}
\subtitle{Departamento de Física.}

\newcommand{\myfront}{
\author[PCFI161]{Corodinadora: C Loyola \\ Profesores C Femenías / F Bugini / D Basantes}
\institute[UNAB]{Universidad Andrés Bello \\ Departamento de Física y Astronomía}
\date{Primer Semestre 2025}
}

\titlegraphic{
  \includegraphics[width=.08\textwidth]{1) Logo/logo-tux.png}\hfill
  \includegraphics[width=.3\textwidth]{1) Logo/logo-unab.png}\hfill
  \includegraphics[width=.08\textwidth]{1) Logo/logo-python.png}
}

\makeatletter
\setbeamertemplate{title page}{
  \begin{minipage}[b][\paperheight]{\textwidth}
    \vfill%
    \ifx\inserttitle\@empty\else\usebeamertemplate*{title}\fi
    \ifx\insertsubtitle\@empty\else\usebeamertemplate*{subtitle}\fi
    \usebeamertemplate*{title separator}
    \ifx\beamer@shortauthor\@empty\else\usebeamertemplate*{author}\fi
    \ifx\insertdate\@empty\else\usebeamertemplate*{date}\fi
    \ifx\insertinstitute\@empty\else\usebeamertemplate*{institute}\fi
    \vfill
    \ifx\inserttitlegraphic\@empty\else\inserttitlegraphic\fi
    \vspace*{1cm}
  \end{minipage}
}
\makeatother

% Configuración personalizada del frametitle para incluir la sección
\makeatletter
\setbeamertemplate{frametitle}{%
  \nointerlineskip%
  \begin{beamercolorbox}[%
      wd=\paperwidth,%
      sep=0pt,%
      leftskip=\@ifundefined{metropolis@frametitle@padding}{12pt}{\metropolis@frametitle@padding},%
      rightskip=\@ifundefined{metropolis@frametitle@padding}{12pt}{\metropolis@frametitle@padding},%
    ]{frametitle}%
  \@ifundefined{metropolis@frametitlestrut@start}{}{\metropolis@frametitlestrut@start}%
  {\textcolor{LightPythonBlue}{\insertsectionhead}\ifx\insertsectionhead\@empty\else\ $\ni$  \fi}\insertframetitle%
  \nolinebreak%
  \@ifundefined{metropolis@frametitlestrut@end}{}{\metropolis@frametitlestrut@end}%
  \end{beamercolorbox}%
}
\makeatother

\makeatletter
% Configuración segura de longitudes del tema Metropolis
\@ifundefined{metropolis@titleseparator@linewidth}{}{\setlength{\metropolis@titleseparator@linewidth}{2pt}}
\@ifundefined{metropolis@progressonsectionpage@linewidth}{}{\setlength{\metropolis@progressonsectionpage@linewidth}{2pt}}
\@ifundefined{metropolis@progressinheadfoot@linewidth}{}{\setlength{\metropolis@progressinheadfoot@linewidth}{2pt}}
\makeatother



\begin{document}

% ------------------------------------------------------------------------
% Portada de la Presentación
% ------------------------------------------------------------------------
\myfront{}

% ------------------------------------------------------------------------
% Slide 1: Título de la Sesión
% ------------------------------------------------------------------------
\begin{frame}
  \titlepage
  % Ejemplo:
  % \title{Semana 13 - Sesión 1 (Sesión 25): Avances de Proyectos y Revisión de Temas Avanzados}
\end{frame}

% ------------------------------------------------------------------------
% Slide 2: Índice / Tabla de Contenidos
% ------------------------------------------------------------------------
\begin{frame}
  \frametitle{Resumen - Semana 13, Sesión 1 (Sesión 25)}
  \tableofcontents
\end{frame}

% ------------------------------------------------------------------------
% Configuración de bloques
% ------------------------------------------------------------------------
\metroset{block=fill}

% ----------------------------------------------------------------------------------------
% SECCIÓN 1: Introducción y Conexión con la Semana 12
% ----------------------------------------------------------------------------------------
\section{Introducción y Repaso}

% ------------------------------------------------------------------------
% Slide 3: Contexto Posterior a Semana 12
% ------------------------------------------------------------------------
\begin{frame}{Repaso de la Semana 12}
  \begin{itemize}
    \item \textbf{Semana 12, Sesión 1 (Sesión 23)}:
      \begin{itemize}
        \item Iniciamos formalmente los \textbf{proyectos integradores} (requisitos, equipos, cronogramas).
        \item Exposición a nuevas herramientas (\textbf{SciPy}, \textbf{Sympy}) como apoyo.
      \end{itemize}
    \item \textbf{Semana 12, Sesión 2 (Sesión 24)}:
      \begin{itemize}
        \item \textbf{Problema a Evaluar} (25-30 min) basado en POO (composición) revisado en Semana 11.
        \item Integración opcional con Matplotlib para representar datos (\texttt{bar plot}).
      \end{itemize}
    \item \textbf{Objetivo de hoy (Semana 13, Sesión 1)}:
      \begin{itemize}
        \item Revisar \textbf{avances} de los proyectos integradores (primer checkpoint).
        \item Discutir \textbf{temas avanzados} que puedan surgir de los proyectos (optimización, data merges, animaciones, etc.).
      \end{itemize}
  \end{itemize}
\end{frame}

% ------------------------------------------------------------------------
% Slide 4: Objetivos de la Sesión 25
% ------------------------------------------------------------------------
\begin{frame}{Objetivos de la Sesión 25}
  \begin{itemize}
    \item \textbf{Obtener} un breve reporte de progreso de cada equipo sobre su proyecto (clases definidas, datos a usar, gráficas previstas).
    \item \textbf{Proponer} soluciones a problemas técnicos iniciales (estructura POO, lectura de datos, etc.).
    \item \textbf{Presentar} un tema avanzado o tips de \textbf{optimización} (performance) si el tiempo lo permite.
    \item \textbf{Motivar} a que todos sigan un plan de desarrollo organizado, con check-ins semanales.
  \end{itemize}
\end{frame}

% ----------------------------------------------------------------------------------------
% SECCIÓN 2: Revisión de Avances de Proyectos
% ----------------------------------------------------------------------------------------
\section{Revisión de Proyectos}

% ------------------------------------------------------------------------
% Slide 5: Checklist de Avance
% ------------------------------------------------------------------------
\begin{frame}{Checklist Inicial de los Proyectos}
  \begin{itemize}
    \item \textbf{Clases y Estructura POO}: 
      \begin{itemize}
        \item ¿Ya definieron las clases base y derivadas (si procede)?
        \item ¿Composición/polimorfismo aplicable?
      \end{itemize}
    \item \textbf{Datos a usar}:
      \begin{itemize}
        \item Archivos CSV, generados aleatoriamente, APIs externas (opcional).
        \item Formato y limpieza de datos (pandas).
      \end{itemize}
    \item \textbf{Operaciones principales}:
      \begin{itemize}
        \item Cálculos numéricos (\textbf{NumPy}, \textbf{SciPy}, \textbf{Sympy}).
        \item Visualizaciones (\textbf{Matplotlib}, 2D/3D, animaciones).
      \end{itemize}
    \item \textbf{Cronograma}:
      \begin{itemize}
        \item ¿Fechas de entregas intermedias? ¿Presentaciones?
      \end{itemize}
  \end{itemize}
\end{frame}

% ------------------------------------------------------------------------
% Slide 6: Presentación de Cada Equipo
% ------------------------------------------------------------------------
\begin{frame}{Reporte de Progreso por Equipo}
  \begin{itemize}
    \item Cada equipo da un \textbf{resumen} breve:
      \begin{itemize}
        \item Título/tema del proyecto.
        \item Estructura POO planeada.
        \item Fuente de datos (si aplica).
        \item Principales \textbf{dificultades} encontradas.
      \end{itemize}
    \item \textbf{Objetivo}: obtener retroalimentación temprana y resolver bloqueos.
  \end{itemize}
\end{frame}

% ------------------------------------------------------------------------
% Slide 7: Espacio para Dudas y Bloqueos
% ------------------------------------------------------------------------
\begin{frame}{Dudas y Bloqueos}
  \begin{itemize}
    \item Problemas con \textbf{lectura} de datos o formato CSV/Excel.
    \item Estructura de \textbf{clases} y relación (herencia vs. composición).
    \item Integración con \textbf{SciPy} (resolver ODEs, optimización) si es necesario.
    \item Cómo \textbf{visualizar} (2D vs 3D, \textbf{animaciones}, \textbf{dashboards}).
  \end{itemize}
\end{frame}

% ----------------------------------------------------------------------------------------
% SECCIÓN 3: Temas Avanzados o Consejos de Optimización
% ----------------------------------------------------------------------------------------
\section{Consejos de Optimización (Opcional)}

% ------------------------------------------------------------------------
% Slide 8: Tips de Optimización y Performance
% ------------------------------------------------------------------------
\begin{frame}{Tips de Optimización y Performance}
  \begin{itemize}
    \item \textbf{Uso correcto} de estructuras NumPy (evitar bucles en Python si se puede vectorizar).
    \item \textbf{Profiling} con \texttt{\%timeit} en Jupyter/Colab o \texttt{cProfile} en scripts.
    \item \textbf{Evitar cuellos de botella} en lectura/escritura de datos (use \textbf{pandas} read_csv con chunks si grande).
    \item \textbf{Multiprocessing} (tema avanzado), solo si el proyecto lo demanda.
    \item \textbf{Memory usage}: cuidar objetos grandes, \texttt{del} cuando no se necesite, etc.
  \end{itemize}
\end{frame}

% ------------------------------------------------------------------------
% Slide 9: Mini-Ejemplo de Profiling (Opcional)
% ------------------------------------------------------------------------
\begin{frame}[fragile]{Mini-Ejemplo: Timeit en Jupyter/Colab}
\begin{minted}{python}
import numpy as np

%%timeit
arr = np.random.rand(1000000)
res = np.sum(arr)
\end{minted}

\begin{itemize}
  \item Muestra el tiempo de ejecución en milisegundos.
  \item Compare con un bucle manual en Python, para ver la \textbf{ventaja} de vectorización.
\end{itemize}
\end{frame}

% ----------------------------------------------------------------------------------------
% SECCIÓN 4: Actividad en Clase (Refuerzo)
% ----------------------------------------------------------------------------------------
\section{Actividad en Clase}

% ------------------------------------------------------------------------
% Slide 10: Ejercicio de Integración (Opcional)
% ------------------------------------------------------------------------
\begin{frame}{Ejercicio de Integración (Opcional)}
  \begin{block}{Propuesta de 15-20 min}
    \begin{itemize}
      \item En grupos, tomen su proyecto y:
        \begin{itemize}
          \item Añadan un \textbf{método} de profiling a la parte más intensiva.
          \item Implementen un \textbf{pequeño test} de vectorización (si aplica).
          \item Generen un \textbf{gráfico} comparando tiempos (opcional).
        \end{itemize}
      \item Comenten si ven mejoras reales o no.
    \end{itemize}
  \end{block}
  \textbf{Objetivo}: reforzar la idea de \textbf{optimización} y la práctica de tiempo de ejecución.
\end{frame}

% ------------------------------------------------------------------------
% Slide 11: Espacio de Aclaraciones
% ------------------------------------------------------------------------
\begin{frame}{Espacio de Aclaraciones}
  \begin{itemize}
    \item Si el proyecto de algún equipo requiere \textbf{funcionalidades} no discutidas (e.g. \textbf{concurrent.futures}, \textbf{tkinter GUIs}), exponerlo aquí.
    \item Compartir \textbf{links o repositorios} en foros de CANVAS para feedback.
    \item Resolver problemas de \textbf{environment} (versiones de Python, librerías).
  \end{itemize}
\end{frame}

% ----------------------------------------------------------------------------------------
% SECCIÓN 5: Conclusiones y Próximos Pasos
% ----------------------------------------------------------------------------------------
\section{Conclusiones y Próximos Pasos}

% ------------------------------------------------------------------------
% Slide 12: Conclusiones de la Sesión 25
% ------------------------------------------------------------------------
\begin{frame}{Conclusiones de la Sesión 25}
  \begin{itemize}
    \item Conocimos el \textbf{estado actual} de los proyectos, primer checkpoint.
    \item Abordamos \textbf{temas avanzados} o dudas surgidas (optimización, etc.).
    \item Definimos \textbf{próximos hitos} (entregas parciales, revisiones) para cada equipo.
    \item Se sugiere seguir la \textbf{hoja de ruta} y pedir ayuda temprana ante bloqueos.
  \end{itemize}
\end{frame}

% ------------------------------------------------------------------------
% Slide 13: Próxima Sesión (Semana 13, Sesión 2)
% ------------------------------------------------------------------------
\begin{frame}{Próximos Temas}
  \begin{itemize}
    \item Continuar con \textbf{avances de proyectos}, posible mini-lab de soluciones parciales.
    \item Revisar \textbf{retroalimentación} de las actividades en clase.
    \item Temas adicionales según el Syllabus (estadísticas avanzadas, HPC, etc.).
  \end{itemize}
  \vspace{0.3cm}
  \textbf{Mantengan el cronograma y envíen actualizaciones si cambian planes.}
\end{frame}

% ------------------------------------------------------------------------
% Slide 14: Recursos Adicionales
% ------------------------------------------------------------------------
\begin{frame}{Recursos Adicionales}
  \begin{itemize}
    \item \href{https://docs.python.org/3/library/profile.html}{\textbf{Python Profiling Docs}} - \texttt{cProfile}, \texttt{pstats}.
    \item \href{https://scipy.org/}{\textbf{SciPy}} - más ejemplos (ODE, optimize, fft).
    \item \href{https://pandas.pydata.org/docs/}{\textbf{pandas Docs}} - merges, groupby, performance tips.
    \item \textbf{Canvas y Foros} - para compartir repos, dudas de implementación.
  \end{itemize}
\end{frame}

% ------------------------------------------------------------------------
% Slide 15: Cierre de la Sesión
% ------------------------------------------------------------------------
\begin{frame}
  \Huge{\centerline{¡Sigan con buen trabajo y hasta la próxima sesión!}}
  \vspace{0.5cm}
  \normalsize
  \begin{itemize}
    \item Apóyense en foros y grupos de estudio.
    \item Revisen su cronograma de proyectos y planifiquen tareas semanales.
  \end{itemize}
\end{frame}

\end{document}

