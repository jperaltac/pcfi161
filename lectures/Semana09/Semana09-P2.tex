\documentclass[10pt]{beamer}
\usetheme[progressbar=frametitle]{metropolis}
\usepackage{appendixnumberbeamer}
\usepackage{fancyvrb}
\usepackage{booktabs}
\usepackage[scale=2]{ccicons}
\usepackage{pgfplots}
\usepgfplotslibrary{dateplot}
\usepackage{type1cm}
\usepackage{lettrine}
\usepackage{ragged2e}
\usepackage{xspace}
\newcommand{\themename}{\textbf{\textsc{metropolis}}\xspace}
\usepackage{graphicx} % Allows including images
\usepackage{booktabs} % Allows the use of \toprule, \midrule and \bottomrule in tables
\usepackage[utf8]{inputenc} %solucion del problema de los acentos.
\usepackage{xcolor}
\usepackage{verbatim}

\definecolor{LightGray}{gray}{0.9}
% Paleta de azules estilo Python/matplotlib
\definecolor{PythonBlue}{RGB}{31, 119, 180}      % Azul principal matplotlib
\definecolor{LightPythonBlue}{RGB}{174, 199, 232} % Azul claro para fondos
\definecolor{DarkPythonBlue}{RGB}{23, 90, 135}    % Azul oscuro para líneas

% Colores para tema oscuro tipo VS Code/Colab
\definecolor{DarkBackground}{RGB}{40, 44, 52}     % Fondo oscuro similar a VS Code
\definecolor{CodeText}{RGB}{171, 178, 191}        % Texto gris claro para código
\definecolor{DarkFrame}{RGB}{60, 63, 65}          % Color del marco/borde

% Colores para tema claro
\definecolor{LightBackground}{RGB}{248, 248, 242} % Fondo claro tipo GitHub
\definecolor{LightCodeText}{RGB}{51, 51, 51}      % Texto oscuro para tema claro
\definecolor{LightFrame}{RGB}{220, 220, 220}      % Marco gris claro

\usepackage{minted}

%%%%%%%%%%%%%%%%%%%%%%%%%%%%%%%%%%%%%%%%%%%%%%%%%%%%%%%%%%%%%%%%%%%%%%%%%%%%%%%%%%%%%%
% CONFIGURACIÓN DE TEMAS - CAMBIA AQUÍ PARA ALTERNAR
% ========================================================================
% Descomenta UNA de las siguientes líneas para elegir el tema:

% TEMA OSCURO (VS Code style)
%\newcommand{\mytheme}{oscuro}
%\usemintedstyle{monokai}
%\newcommand{\mybgcolor}{DarkBackground}

% TEMA CLARO (GitHub style) - comenta las 3 líneas de arriba y descomenta estas:
 \newcommand{\mytheme}{claro}
 \usemintedstyle{default}
 \newcommand{\mybgcolor}{LightBackground}

% ========================================================================
\newcommand{\mypyfile}[1]{\inputminted[linenos=true, fontsize=\footnotesize, frame=lines, framesep=5\fboxrule,framerule=1pt]{python}{#1}}

% Comando personalizado para código Python inline
\newcommand{\pycode}[1]{\begin{minted}{python}#1\end{minted}}

% ========================================================================
% COMANDOS ESPECÍFICOS PARA CADA TEMA
% ========================================================================
% Comando para tema oscuro específico
\newcommand{\pyoscuro}[1]{%
\begin{minted}[
  bgcolor=DarkBackground,
  style=monokai,
  breaklines,
  frame=lines,
  framesep=2mm,
  baselinestretch=1.2,
  linenos,
  fontsize=\footnotesize
]{python}
#1
\end{minted}
}

% Comando para tema claro específico
\newcommand{\pyclaro}[1]{%
\begin{minted}[
  bgcolor=LightBackground,
  style=default,
  breaklines,
  frame=lines,
  framesep=2mm,
  baselinestretch=1.2,
  linenos,
  fontsize=\footnotesize
]{python}
#1
\end{minted}
}
%%%%%%%%%%%%%%%%%%%%%%%%%%%%%%%%%%%%%%%%%%%%%%%%%%%%%%%%%%%%%%%%%%%%%%%%%%%%%%%%%%%%%%



%%%%%%%%%%%%%%%%%%%%%%%%%%%%%%%%%%%%%%%%%%%%%%%%%%%%%%%%%%%%%%%%%%%%%%%%%%%%%%%%%%%%%%
% Configuración global para todos los bloques minted de Python
% Usa automáticamente el tema seleccionado arriba
\setminted[python]{
  breaklines,
  frame=lines,
  framesep=2mm,
  baselinestretch=1.2,
  bgcolor=\mybgcolor,
  linenos, 
  fontsize=\footnotesize
} % Configuración dinámica según tema elegido
%%%%%%%%%%%%%%%%%%%%%%%%%%%%%%%%%%%%%%%%%%%%%%%%%%%%%%%%%%%%%%%%%%%%%%%%%%%%%%%%%%%%%%



%%%%%%%%%%%%%%%%%%%%%%%%%%%%%%%%%%%%%%%%%%%%%%%%%%%%%%%%%%%%%%%%%%%%%%%%%%%%%%%%%%%%%%
% Configuración de colores del tema con paleta azul Python
\setbeamercolor{progress bar}{fg=DarkPythonBlue,bg=LightPythonBlue}
\setbeamercolor{title separator}{fg=DarkPythonBlue,bg=white!50!black}
\setbeamercolor{frametitle}{fg=white,bg=PythonBlue}
\title[PCFI161]{Programaci\'on para F\'isica y Astronom\'ia}
\subtitle{Departamento de Física.}

\newcommand{\myfront}{
\author[PCFI161]{Corodinadora: C Loyola \\ Profesores C Femenías / F Bugini / D Basantes}
\institute[UNAB]{Universidad Andrés Bello \\ Departamento de Física y Astronomía}
\date{Primer Semestre 2025}
}

\titlegraphic{
  \includegraphics[width=.08\textwidth]{1) Logo/logo-tux.png}\hfill
  \includegraphics[width=.3\textwidth]{1) Logo/logo-unab.png}\hfill
  \includegraphics[width=.08\textwidth]{1) Logo/logo-python.png}
}

\makeatletter
\setbeamertemplate{title page}{
  \begin{minipage}[b][\paperheight]{\textwidth}
    \vfill%
    \ifx\inserttitle\@empty\else\usebeamertemplate*{title}\fi
    \ifx\insertsubtitle\@empty\else\usebeamertemplate*{subtitle}\fi
    \usebeamertemplate*{title separator}
    \ifx\beamer@shortauthor\@empty\else\usebeamertemplate*{author}\fi
    \ifx\insertdate\@empty\else\usebeamertemplate*{date}\fi
    \ifx\insertinstitute\@empty\else\usebeamertemplate*{institute}\fi
    \vfill
    \ifx\inserttitlegraphic\@empty\else\inserttitlegraphic\fi
    \vspace*{1cm}
  \end{minipage}
}
\makeatother

% Configuración personalizada del frametitle para incluir la sección
\makeatletter
\setbeamertemplate{frametitle}{%
  \nointerlineskip%
  \begin{beamercolorbox}[%
      wd=\paperwidth,%
      sep=0pt,%
      leftskip=\@ifundefined{metropolis@frametitle@padding}{12pt}{\metropolis@frametitle@padding},%
      rightskip=\@ifundefined{metropolis@frametitle@padding}{12pt}{\metropolis@frametitle@padding},%
    ]{frametitle}%
  \@ifundefined{metropolis@frametitlestrut@start}{}{\metropolis@frametitlestrut@start}%
  {\textcolor{LightPythonBlue}{\insertsectionhead}\ifx\insertsectionhead\@empty\else\ $\ni$  \fi}\insertframetitle%
  \nolinebreak%
  \@ifundefined{metropolis@frametitlestrut@end}{}{\metropolis@frametitlestrut@end}%
  \end{beamercolorbox}%
}
\makeatother

\makeatletter
% Configuración segura de longitudes del tema Metropolis
\@ifundefined{metropolis@titleseparator@linewidth}{}{\setlength{\metropolis@titleseparator@linewidth}{2pt}}
\@ifundefined{metropolis@progressonsectionpage@linewidth}{}{\setlength{\metropolis@progressonsectionpage@linewidth}{2pt}}
\@ifundefined{metropolis@progressinheadfoot@linewidth}{}{\setlength{\metropolis@progressinheadfoot@linewidth}{2pt}}
\makeatother



\title{Semana 9 – Sesión 2 (18)\\Análisis y visualización con pandas, nivel intermedio}
\author{PCFI161 – Programación para Física y Astronomía}
\date{\today}

\begin{document}
\myfront{}

% 1 ──────────────────────────────────────────────────────────
\begin{frame}\titlepage\end{frame}

% 2 ──────────────────────────────────────────────────────────
\begin{frame}\frametitle{Índice}\tableofcontents\end{frame}

% 3 ──────────────────────────────────────────────────────────
\section{Fuentes de datos}
\begin{frame}{Fuentes para la práctica}
\begin{itemize}
  \item Meteoritos NASA (limitado a 5 000 filas)\newline
        \url{https://data.nasa.gov/resource/y77d-th95.csv?\$\$limit=5000}
  \item Exoplanetas Seaborn\newline
        \url{https://raw.githubusercontent.com/mwaskom/seaborn-data/master/planets.csv}
  \item Masas PDG (extracto)\newline
        \url{https://raw.githubusercontent.com/particle-physics-book/data/master/pdg_mass.csv}
\end{itemize}
Copia estas direcciones directamente en tu notebook.
\end{frame}

% 4 ──────────────────────────────────────────────────────────
\section{Carga y limpieza}
\begin{frame}[fragile]{Lectura de los tres conjuntos}
\begin{minted}{python}
met   = pd.read_csv("https://data.nasa.gov/resource/y77d-th95.csv?$$limit=5000")
pln   = pd.read_csv("https://raw.githubusercontent.com/mwaskom/seaborn-data/master/planets.csv")
pdg   = pd.read_csv("https://raw.githubusercontent.com/particle-physics-book/data/master/pdg_mass.csv")
\end{minted}
`read\_csv` descarga y parsea automáticamente; no requiere bibliotecas externas.
\end{frame}

% 5 ──────────────────────────────────────────────────────────
\begin{frame}[fragile]{Eliminar faltantes y outliers en meteoritos}
\begin{minted}{python}
met = met[met.mass.notna()].copy()
met["massKg"] = met.mass.astype(float) / 1000    # gramos a kg
met = met[met.massKg < 1e4]                      # quitar >10 t
\end{minted}
Tres líneas bastan para dejar el conjunto listo para análisis.
\end{frame}

% 6 ──────────────────────────────────────────────────────────
\section{Diagnóstico de faltantes}
\begin{frame}[fragile]{Porcentaje de valores nulos}
\begin{minted}{python}
faltantes = met.isna().mean() * 100
faltantes.sort_values(ascending=False).head()
\end{minted}
Así priorizas las columnas que necesitan limpieza.
\end{frame}

% 7 ──────────────────────────────────────────────────────────
\section{Agrupaciones avanzadas}
\begin{frame}[fragile]{Resumen por década}
\begin{minted}{python}
met["year"] = pd.to_datetime(met.year, errors="coerce").dt.year
decade = (met.groupby(met.year // 10 * 10)["massKg"]
              .agg(conteo='size', promedio='mean', total='sum'))
decade.head()
\end{minted}
Agrupar por transformación matemática del año genera la tabla por décadas.
\end{frame}

% 8 ──────────────────────────────────────────────────────────
\begin{frame}[fragile]{Pivot table visual}
\begin{minted}{python}
pln["decada"] = pln.year // 10 * 10
tabla = pln.pivot_table(index="method", columns="decada",
                        values="distance", aggfunc="count")
plt.imshow(tabla.fillna(0), aspect='auto')
plt.colorbar(label="número de exoplanetas"); plt.yticks(range(len(tabla)), tabla.index)
plt.xlabel("decada"); plt.show()
\end{minted}
La matriz calor ayuda a descubrir métodos dominantes en cada período.
\end{frame}

% 9 ──────────────────────────────────────────────────────────
\section{Reshape melt y pivot}
\begin{frame}[fragile]{Convertir tabla ancha a larga}
\begin{minted}{python}
wide = decade.reset_index().rename(columns={"year":"decada"})
long = wide.melt(id_vars="decada",
                 var_name="metrica", value_name="valor")
long.head()
\end{minted}
La forma “larga” es conveniente para bibliotecas de visualización como Seaborn.
\end{frame}

% 10 ──────────────────────────────────────────────────────────
\section{Series temporales}
\begin{frame}[fragile]{Resample anual de masa caída}
\begin{minted}{python}
metTs = (met.set_index(pd.to_datetime(met.year, format='%Y'))
              .massKg.resample("1Y").sum())
metTs.plot(title="Masa total de meteoritos por año"); plt.ylabel("kg"); plt.show()
\end{minted}
`resample` es la versión temporal de `groupby`.
\end{frame}

% 11 ──────────────────────────────────────────────────────────
\section{Merge ilustrativo}
\begin{frame}[fragile]{Normalizar masas de planetas}
\begin{minted}{python}
proton = pdg.loc[pdg.symbol == "p", "mass_GeV"].iat[0]
pln["massOverMp"] = pln.mass / proton
pln[["name", "mass", "massOverMp"]].head()
\end{minted}
Combinamos física de partículas con astronomía en un solo DataFrame.
\end{frame}

% 12 ──────────────────────────────────────────────────────────
\section{Visualización integrada}
\begin{frame}[fragile]{Scatters y densidad KDE}
\begin{minted}{python}
pln.plot.scatter(x="mass", y="orbital_period",
                 alpha=0.4, logx=True, logy=True,
                 title="Masa vs periodo orbital")
pln["distance"].plot.kde(title="Densidad de distancias"); plt.show()
\end{minted}
La API integrada evita llamadas adicionales a Matplotlib.
\end{frame}

% 13 ──────────────────────────────────────────────────────────
\section{Estilos}
\begin{frame}[fragile]{Aplicar estilo SciencePlots}
\begin{minted}{python}
plt.style.use("science")
decade.total.plot(marker='o')
plt.title("Toneladas de meteoritos por década"); plt.ylabel("kg"); plt.show()
\end{minted}
El resultado queda listo para copiar en un informe o paper.
\end{frame}

% 14 ──────────────────────────────────────────────────────────
\section{Exportar resultados}
\begin{frame}[fragile]{Guardar en disco}
\begin{minted}{python}
decade.to_csv("meteoritos_por_decada.csv", index=True)
pln.to_parquet("planetsCompact.parquet", index=False)
\end{minted}
Parquet brinda alta compresión y lectura más veloz que CSV.
\end{frame}

% 15 ──────────────────────────────────────────────────────────
\section{Rendimiento}
\begin{frame}{Consejos de eficiencia}
\begin{itemize}
  \item Columnas con pocos valores → `astype("category")` para reducir memoria.  
  \item `read\_csv` con `chunksize` permite procesar archivos gigantes en porciones.  
  \item `eval` y `query` compilan expresiones y evitan copias intermedias.
\end{itemize}
\end{frame}

% 16 ──────────────────────────────────────────────────────────
\section{Mini proyecto en parejas}
\begin{frame}{Instrucciones de treinta minutos}
\begin{enumerate}
  \item Clasifica cada meteorito en continente según latitud y longitud.  
  \item Histograma logarítmico de masa por continente.  
  \item Pivot table de cantidad de impactos por década y continente.  
  \item Exporta la pivot table a un archivo Excel para compartir.
\end{enumerate}
\end{frame}

% 17 ──────────────────────────────────────────────────────────
\section{Guía rápida de solución}
\begin{frame}[fragile,shrink=5]{Fragmento de código orientativo}
\begin{minted}{python}
bins = np.logspace(-2, 4, 30)
for cont in met.continent.unique():
    met.loc[met.continent == cont, "massKg"].plot.hist(
        bins=bins, alpha=0.5, label=cont)
plt.xscale("log"); plt.legend(); plt.show()
\end{minted}
\end{frame}

% 18 ──────────────────────────────────────────────────────────
\section{Recursos recomendados}
\begin{frame}{Para profundizar}
\begin{itemize}
  \item Cheat-sheet oficial: \url{https://pandas.pydata.org/Pandas_Cheat_Sheet.pdf}
  \item Libro “Python Data Science Handbook”, capítulos ocho a once.
  \item Lista Awesome Public Datasets en GitHub.
\end{itemize}
\end{frame}

% 19 ──────────────────────────────────────────────────────────
\section{Preguntas frecuentes}
\begin{frame}{FAQ en vivo}
\begin{itemize}
 \item \textbf{Warning SettingWithCopy} → usa `loc` y `copy`.  
 \item CSV muy grande → convierte a Parquet y gana 5-10 x en velocidad.  
 \item Problemas en merge → usa `validate="one\_to\_one"` para asegurar unicidad.
\end{itemize}
\end{frame}

% 20 ──────────────────────────────────────────────────────────
\section{Próximos temas}
\begin{frame}{En la próxima semana}
Estadística básica, bootstrap y visualización con Seaborn aplicadas a datos físicos.
\end{frame}

% 21 ──────────────────────────────────────────────────────────
\begin{frame}\Huge{\centerline{¡Fin de la sesión!}}\end{frame}

\end{document}
