\documentclass[10pt]{beamer}
\usetheme[progressbar=frametitle]{metropolis}
\usepackage{appendixnumberbeamer}
\usepackage{fancyvrb}
\usepackage{booktabs}
\usepackage[scale=2]{ccicons}
\usepackage{pgfplots}
\usepgfplotslibrary{dateplot}
\usepackage{type1cm}
\usepackage{lettrine}
\usepackage{ragged2e}
\usepackage{xspace}
\newcommand{\themename}{\textbf{\textsc{metropolis}}\xspace}
\usepackage{graphicx} % Allows including images
\usepackage{booktabs} % Allows the use of \toprule, \midrule and \bottomrule in tables
\usepackage[utf8]{inputenc} %solucion del problema de los acentos.
\usepackage{xcolor}
\usepackage{verbatim}

\definecolor{LightGray}{gray}{0.9}
% Paleta de azules estilo Python/matplotlib
\definecolor{PythonBlue}{RGB}{31, 119, 180}      % Azul principal matplotlib
\definecolor{LightPythonBlue}{RGB}{174, 199, 232} % Azul claro para fondos
\definecolor{DarkPythonBlue}{RGB}{23, 90, 135}    % Azul oscuro para líneas

% Colores para tema oscuro tipo VS Code/Colab
\definecolor{DarkBackground}{RGB}{40, 44, 52}     % Fondo oscuro similar a VS Code
\definecolor{CodeText}{RGB}{171, 178, 191}        % Texto gris claro para código
\definecolor{DarkFrame}{RGB}{60, 63, 65}          % Color del marco/borde

% Colores para tema claro
\definecolor{LightBackground}{RGB}{248, 248, 242} % Fondo claro tipo GitHub
\definecolor{LightCodeText}{RGB}{51, 51, 51}      % Texto oscuro para tema claro
\definecolor{LightFrame}{RGB}{220, 220, 220}      % Marco gris claro

\usepackage{minted}

%%%%%%%%%%%%%%%%%%%%%%%%%%%%%%%%%%%%%%%%%%%%%%%%%%%%%%%%%%%%%%%%%%%%%%%%%%%%%%%%%%%%%%
% CONFIGURACIÓN DE TEMAS - CAMBIA AQUÍ PARA ALTERNAR
% ========================================================================
% Descomenta UNA de las siguientes líneas para elegir el tema:

% TEMA OSCURO (VS Code style)
%\newcommand{\mytheme}{oscuro}
%\usemintedstyle{monokai}
%\newcommand{\mybgcolor}{DarkBackground}

% TEMA CLARO (GitHub style) - comenta las 3 líneas de arriba y descomenta estas:
 \newcommand{\mytheme}{claro}
 \usemintedstyle{default}
 \newcommand{\mybgcolor}{LightBackground}

% ========================================================================
\newcommand{\mypyfile}[1]{\inputminted[linenos=true, fontsize=\footnotesize, frame=lines, framesep=5\fboxrule,framerule=1pt]{python}{#1}}

% Comando personalizado para código Python inline
\newcommand{\pycode}[1]{\begin{minted}{python}#1\end{minted}}

% ========================================================================
% COMANDOS ESPECÍFICOS PARA CADA TEMA
% ========================================================================
% Comando para tema oscuro específico
\newcommand{\pyoscuro}[1]{%
\begin{minted}[
  bgcolor=DarkBackground,
  style=monokai,
  breaklines,
  frame=lines,
  framesep=2mm,
  baselinestretch=1.2,
  linenos,
  fontsize=\footnotesize
]{python}
#1
\end{minted}
}

% Comando para tema claro específico
\newcommand{\pyclaro}[1]{%
\begin{minted}[
  bgcolor=LightBackground,
  style=default,
  breaklines,
  frame=lines,
  framesep=2mm,
  baselinestretch=1.2,
  linenos,
  fontsize=\footnotesize
]{python}
#1
\end{minted}
}
%%%%%%%%%%%%%%%%%%%%%%%%%%%%%%%%%%%%%%%%%%%%%%%%%%%%%%%%%%%%%%%%%%%%%%%%%%%%%%%%%%%%%%



%%%%%%%%%%%%%%%%%%%%%%%%%%%%%%%%%%%%%%%%%%%%%%%%%%%%%%%%%%%%%%%%%%%%%%%%%%%%%%%%%%%%%%
% Configuración global para todos los bloques minted de Python
% Usa automáticamente el tema seleccionado arriba
\setminted[python]{
  breaklines,
  frame=lines,
  framesep=2mm,
  baselinestretch=1.2,
  bgcolor=\mybgcolor,
  linenos, 
  fontsize=\footnotesize
} % Configuración dinámica según tema elegido
%%%%%%%%%%%%%%%%%%%%%%%%%%%%%%%%%%%%%%%%%%%%%%%%%%%%%%%%%%%%%%%%%%%%%%%%%%%%%%%%%%%%%%



%%%%%%%%%%%%%%%%%%%%%%%%%%%%%%%%%%%%%%%%%%%%%%%%%%%%%%%%%%%%%%%%%%%%%%%%%%%%%%%%%%%%%%
% Configuración de colores del tema con paleta azul Python
\setbeamercolor{progress bar}{fg=DarkPythonBlue,bg=LightPythonBlue}
\setbeamercolor{title separator}{fg=DarkPythonBlue,bg=white!50!black}
\setbeamercolor{frametitle}{fg=white,bg=PythonBlue}
\title[PCFI161]{Programaci\'on para F\'isica y Astronom\'ia}
\subtitle{Departamento de Física.}

\newcommand{\myfront}{
\author[PCFI161]{Corodinadora: C Loyola \\ Profesores C Femenías / F Bugini / D Basantes}
\institute[UNAB]{Universidad Andrés Bello \\ Departamento de Física y Astronomía}
\date{Primer Semestre 2025}
}

\titlegraphic{
  \includegraphics[width=.08\textwidth]{1) Logo/logo-tux.png}\hfill
  \includegraphics[width=.3\textwidth]{1) Logo/logo-unab.png}\hfill
  \includegraphics[width=.08\textwidth]{1) Logo/logo-python.png}
}

\makeatletter
\setbeamertemplate{title page}{
  \begin{minipage}[b][\paperheight]{\textwidth}
    \vfill%
    \ifx\inserttitle\@empty\else\usebeamertemplate*{title}\fi
    \ifx\insertsubtitle\@empty\else\usebeamertemplate*{subtitle}\fi
    \usebeamertemplate*{title separator}
    \ifx\beamer@shortauthor\@empty\else\usebeamertemplate*{author}\fi
    \ifx\insertdate\@empty\else\usebeamertemplate*{date}\fi
    \ifx\insertinstitute\@empty\else\usebeamertemplate*{institute}\fi
    \vfill
    \ifx\inserttitlegraphic\@empty\else\inserttitlegraphic\fi
    \vspace*{1cm}
  \end{minipage}
}
\makeatother

% Configuración personalizada del frametitle para incluir la sección
\makeatletter
\setbeamertemplate{frametitle}{%
  \nointerlineskip%
  \begin{beamercolorbox}[%
      wd=\paperwidth,%
      sep=0pt,%
      leftskip=\@ifundefined{metropolis@frametitle@padding}{12pt}{\metropolis@frametitle@padding},%
      rightskip=\@ifundefined{metropolis@frametitle@padding}{12pt}{\metropolis@frametitle@padding},%
    ]{frametitle}%
  \@ifundefined{metropolis@frametitlestrut@start}{}{\metropolis@frametitlestrut@start}%
  {\textcolor{LightPythonBlue}{\insertsectionhead}\ifx\insertsectionhead\@empty\else\ $\ni$  \fi}\insertframetitle%
  \nolinebreak%
  \@ifundefined{metropolis@frametitlestrut@end}{}{\metropolis@frametitlestrut@end}%
  \end{beamercolorbox}%
}
\makeatother

\makeatletter
% Configuración segura de longitudes del tema Metropolis
\@ifundefined{metropolis@titleseparator@linewidth}{}{\setlength{\metropolis@titleseparator@linewidth}{2pt}}
\@ifundefined{metropolis@progressonsectionpage@linewidth}{}{\setlength{\metropolis@progressonsectionpage@linewidth}{2pt}}
\@ifundefined{metropolis@progressinheadfoot@linewidth}{}{\setlength{\metropolis@progressinheadfoot@linewidth}{2pt}}
\makeatother



\title{Semana 9 – Sesión 1 (17)\\Matplotlib avanzado, ciclos for y primeras destrezas con pandas}
\author{PCFI161 – Programación para Física y Astronomía}
\date{\today}

\begin{document}
\myfront{}

% 1 ──────────────── PORTADA ──────────────────────────────────
\begin{frame}\titlepage\end{frame}

% 2 ──────────────── ÍNDICE ───────────────────────────────────
\begin{frame}\frametitle{Índice}\tableofcontents\end{frame}

% 3 ──────────────── FUENTES DE DATOS ─────────────────────────
\section{Fuentes de datos}
\begin{frame}{Fuentes de datos que usaremos hoy}
\begin{itemize}
  \item Exoplanetas – Seaborn  
        \url{https://raw.githubusercontent.com/mwaskom/seaborn-data/master/planets.csv}
  \item Masas de partículas – PDG  
        \url{https://raw.githubusercontent.com/particle-physics-book/data/master/pdg\_mass.csv}
\end{itemize}
Copia las direcciones en tu notebook para seguir los ejemplos.
\end{frame}

% 4 ──────────────── REPASO FOR (INTRO) ──────────────────────
\section{Repaso rápido de ciclos for}
\begin{frame}{Por qué reforzar los ciclos for}
Los bucles siguen apareciendo en limpieza de datos, cálculos y gráficos.  
Observa tres patrones de uso frecuentes.
\end{frame}

% 5 ──────────────── FOR EN LISTAS ───────────────────────────
\begin{frame}[fragile]{Ejemplo 1 – recorrer una lista}
\textbf{Objetivo.} Calcular la energía cinética de varias masas sin NumPy.
\begin{minted}{python}
masas   = [1.0, 3.5, 2.2]
veloc   = [10.0, 7.5, 12.3]
energ   = []                        # salida vacía

for m, v in zip(masas, veloc):
    ec = 0.5 * m * v**2
    energ.append(ec)
print(energ)
\end{minted}
\textit{Punto docente.} Presenta la función zip y la técnica de lista vacía + append.
\end{frame}

% 6 ──────────────── ENUMERATE ───────────────────────────────
\begin{frame}[fragile]{Ejemplo 2 – enumerate para acceso dual}
\textbf{Objetivo.} Mostrar índice más valor cuando la lista es corta.
\begin{minted}{python}
planetas = ["Mercurio", "Venus", "Tierra", "Marte"]
for i, nombre in enumerate(planetas, start=1):
    print(f"{i}. {nombre}")
\end{minted}
\emph{Enumerate evita errores al llevar contadores manualmente.}
\end{frame}

% 7 ──────────────── FOR EN NUMPY ────────────────────────────
\begin{frame}[fragile]{Ejemplo 3 – bucle sobre array NumPy}
\textbf{Advertencia.} Siempre que sea posible usa vectorización,  
pero a veces un for es más claro para prototipos.
\begin{minted}{python}
import numpy as np
x = np.linspace(0, 2*np.pi, 5)
for valor in x:
    print(f"sen({valor:.2f}) = {np.sin(valor):.3f}")
\end{minted}
\end{frame}

% 8 ──────────────── MATPLOTLIB INTRO ────────────────────────
\section{Matplotlib avanzado}
\begin{frame}{Objetivo de la sección}
Crear figuras profesionales: subplots elegantes, superficies 3D y barras de color.  
Añadimos comentarios didácticos para que el profesor narre lo que sucede.
\end{frame}

% 9 ──────────────── SUBPLOTS ───────────────────────────────
\begin{frame}[fragile]{Subplots lado a lado con ejes compartidos}
\textbf{Qué haremos.} Graficar seno y coseno en la misma figura para comparar amplitud.
\begin{minted}{python}
fig, ejes = plt.subplots(1, 2, figsize=(8,3), sharex=True)
x = np.linspace(0, 2*np.pi, 200)
ejes[0].plot(x, np.sin(x),  label='sen')      # rojo por defecto C1
ejes[1].plot(x, np.cos(x), '--', label='cos') # línea discontinua
for ax in ejes:
    ax.grid(); ax.legend()
fig.suptitle("Comparación seno y coseno"); plt.tight_layout()
\end{minted}
\end{frame}

% 10 ─────────────── STYLES ────────────────────────────────
\begin{frame}[fragile]{Aplicar un estilo científico}
\textbf{Propósito.} Mejorar tipografía y espaciado con un comando.
\begin{minted}{python}
plt.style.use("science")  # pip install SciencePlots
\end{minted}
Ideal para informes formales o posters científicos.
\end{frame}

% 11 ────────────── HISTOGRAMAS ─────────────────────────────
\begin{frame}[fragile]{Histogramas superpuestos para dos muestras}
\textbf{Qué veremos.} Distribuciones A y B y cómo se solapan.
\begin{minted}{python}
A = np.random.normal(0, 1, 1000)
B = np.random.normal(1, 1.2, 1000)
plt.hist(A, bins=25, alpha=0.6, label='A')
plt.hist(B, bins=25, alpha=0.6, label='B')
plt.xlabel("valor"); plt.ylabel("frecuencia"); plt.legend(); plt.show()
\end{minted}
Discusión: ¿cuál tiene mayor varianza? ¿Hay intersección apreciable?
\end{frame}

% 12 ────────────── SUPERFICIE 3D ───────────────────────────
\begin{frame}[fragile]{Superficie 3D con barra de color}
\textbf{Idea.} Visualizar una campana bidimensional Gaussiana.
\begin{minted}{python}
x = y = np.linspace(-4, 4, 100)
X, Y = np.meshgrid(x, y)
Z = np.exp(-(X**2 + Y**2))
fig = plt.figure(figsize=(6,4))
ax  = fig.add_subplot(projection='3d')
surf = ax.plot_surface(X, Y, Z, cmap='viridis')
fig.colorbar(surf, label='amplitud', shrink=0.6)
plt.show()
\end{minted}
La barra de color codifica la altura de la superficie.
\end{frame}

% 13 ────────────── POR QUÉ PANDAS ─────────────────────────
\section{Introducción rápida a pandas}
\begin{frame}{Motivación}
\begin{itemize}
  \item Tabla etiquetada con operaciones vectorizadas.  
  \item Lectura y escritura de formatos comunes sin configuración.  
  \item Funciona de la mano con Matplotlib para gráficos instantáneos.
\end{itemize}
\end{frame}

% 14 ────────────── DATAFRAME CREAR ─────────────────────────
\begin{frame}[fragile]{Crear un DataFrame y describir estadísticas}
\textbf{Pasos.} Construir datos simples y solicitar resumen numérico.
\begin{minted}{python}
import pandas as pd
df = pd.DataFrame({"masa":[1.0, 3.5, 2.2],
                   "vel":[10.0, 7.5, 12.3]})
df["energia"] = 0.5 * df.masa * df.vel**2
df.describe()
\end{minted}
`describe` entrega media, desviación, mínimos y cuartiles.
\end{frame}

% 15 ────────────── LEER CSV REMOTO ─────────────────────────
\begin{frame}[fragile]{Leer un CSV remoto en una línea}
\textbf{URL usada.} Planetas extra solares en el repositorio Seaborn.
\begin{minted}{python}
planets = pd.read_csv(
  "https://raw.githubusercontent.com/mwaskom/seaborn-data/master/planets.csv")
planets.head()
\end{minted}
\end{frame}

% 16 ────────────── FILTRO LOC ─────────────────────────────
\begin{frame}[fragile]{Filtrar filas con loc y máscaras}
\textbf{Caso.} Planetas descubiertos mediante tránsito con masa definida.
\begin{minted}{python}
mask = (planets.method == "Transit") & planets.mass.notna()
transit = planets.loc[mask, ["name", "mass", "distance"]]
\end{minted}
\end{frame}

% 17 ────────────── GROUPBY ────────────────────────────────
\begin{frame}[fragile]{Resumen rápido con groupby y agg}
\textbf{Meta.} Contar y encontrar la mediana de distancias por método.
\begin{minted}{python}
resumen = (planets.groupby("method")["distance"]
                     .agg(conteo='size', mediana='median'))
resumen.head()
\end{minted}
\end{frame}

% 18 ────────────── PLOT INTEGRADO ─────────────────────────
\begin{frame}[fragile]{Histograma desde el propio DataFrame}
\textbf{Qué se graficará.} Distribución de distancias a la estrella.
\begin{minted}{python}
planets["distance"].plot.hist(bins=25, alpha=0.8)
plt.title("Distancias"); plt.xlabel("parsec"); plt.show()
\end{minted}
\end{frame}

% 19 ────────────── QUERY LECTURA ──────────────────────────
\begin{frame}[fragile]{Filtro legible con query}
\begin{minted}{python}
ligeros = planets.query("mass < 1 and distance < 100")
\end{minted}
`query` acepta una cadena al estilo SQL para mayor claridad.
\end{frame}

% 20 ────────────── MERGE ───────────────────────────────────
\begin{frame}[fragile]{Unir con merge para añadir información}
\textbf{Ejemplo.} Agregar la masa del protón a cada planeta para normalizar.
\begin{minted}{python}
pdg = pd.read_csv(
  "https://raw.githubusercontent.com/particle-physics-book/data/master/pdg_mass.csv")
proton = pdg.loc[pdg.symbol == "p", "mass_GeV"].iat[0]
planets["masaSobreMp"] = planets.mass / proton
\end{minted}
\end{frame}

% 21 ────────────── CONSEJOS ───────────────────────────────
\begin{frame}{Consejos prácticos}
\begin{itemize}
  \item Usa `memory\_usage(deep=True)` para revisar consumo.  
  \item Convierte columnas repetidas a categoría y ahorra memoria.  
  \item Prueba `eval` y `query` para filtros rápidos y legibles.
\end{itemize}
\end{frame}

% 22 ────────────── ACTIVIDAD ──────────────────────────────
\section{Actividad breve}
\begin{frame}{Desafío de quince minutos}
Filtra exoplanetas por “Radial Velocity”, dibuja masa versus periodo logarítmico y comenta la tendencia.
\end{frame}

% 23 ────────────── RESUMEN ────────────────────────────────
\section{Resumen}
\begin{frame}{Lo aprendido hoy}
Subplots, estilos, 3D en Matplotlib, repaso de ciclos for y manejo básico de pandas.
\end{frame}

% 24 ────────────── PRÓXIMA ────────────────────────────────
\section*{Próxima sesión}
\begin{frame}{Próximo paso}
La siguiente clase profundiza en pivotado, fechas y rendimiento en pandas.
\end{frame}

% 25 ────────────── CRÉDITOS / FIN ─────────────────────────
\begin{frame}\Huge{\centerline{¡Fin de la sesión!}}\end{frame}

\end{document}
