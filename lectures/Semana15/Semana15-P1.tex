\documentclass[10pt]{beamer}
\usetheme[progressbar=frametitle]{metropolis}
\usepackage{appendixnumberbeamer}
\usepackage{fancyvrb}
\usepackage{booktabs}
\usepackage[scale=2]{ccicons}
\usepackage{pgfplots}
\usepgfplotslibrary{dateplot}
\usepackage{type1cm}
\usepackage{lettrine}
\usepackage{ragged2e}
\usepackage{xspace}
\newcommand{\themename}{\textbf{\textsc{metropolis}}\xspace}
\usepackage{graphicx} % Allows including images
\usepackage{booktabs} % Allows the use of \toprule, \midrule and \bottomrule in tables
\usepackage[utf8]{inputenc} %solucion del problema de los acentos.
\usepackage{xcolor}
\usepackage{verbatim}

\definecolor{LightGray}{gray}{0.9}
% Paleta de azules estilo Python/matplotlib
\definecolor{PythonBlue}{RGB}{31, 119, 180}      % Azul principal matplotlib
\definecolor{LightPythonBlue}{RGB}{174, 199, 232} % Azul claro para fondos
\definecolor{DarkPythonBlue}{RGB}{23, 90, 135}    % Azul oscuro para líneas

% Colores para tema oscuro tipo VS Code/Colab
\definecolor{DarkBackground}{RGB}{40, 44, 52}     % Fondo oscuro similar a VS Code
\definecolor{CodeText}{RGB}{171, 178, 191}        % Texto gris claro para código
\definecolor{DarkFrame}{RGB}{60, 63, 65}          % Color del marco/borde

% Colores para tema claro
\definecolor{LightBackground}{RGB}{248, 248, 242} % Fondo claro tipo GitHub
\definecolor{LightCodeText}{RGB}{51, 51, 51}      % Texto oscuro para tema claro
\definecolor{LightFrame}{RGB}{220, 220, 220}      % Marco gris claro

\usepackage{minted}

%%%%%%%%%%%%%%%%%%%%%%%%%%%%%%%%%%%%%%%%%%%%%%%%%%%%%%%%%%%%%%%%%%%%%%%%%%%%%%%%%%%%%%
% CONFIGURACIÓN DE TEMAS - CAMBIA AQUÍ PARA ALTERNAR
% ========================================================================
% Descomenta UNA de las siguientes líneas para elegir el tema:

% TEMA OSCURO (VS Code style)
%\newcommand{\mytheme}{oscuro}
%\usemintedstyle{monokai}
%\newcommand{\mybgcolor}{DarkBackground}

% TEMA CLARO (GitHub style) - comenta las 3 líneas de arriba y descomenta estas:
 \newcommand{\mytheme}{claro}
 \usemintedstyle{default}
 \newcommand{\mybgcolor}{LightBackground}

% ========================================================================
\newcommand{\mypyfile}[1]{\inputminted[linenos=true, fontsize=\footnotesize, frame=lines, framesep=5\fboxrule,framerule=1pt]{python}{#1}}

% Comando personalizado para código Python inline
\newcommand{\pycode}[1]{\begin{minted}{python}#1\end{minted}}

% ========================================================================
% COMANDOS ESPECÍFICOS PARA CADA TEMA
% ========================================================================
% Comando para tema oscuro específico
\newcommand{\pyoscuro}[1]{%
\begin{minted}[
  bgcolor=DarkBackground,
  style=monokai,
  breaklines,
  frame=lines,
  framesep=2mm,
  baselinestretch=1.2,
  linenos,
  fontsize=\footnotesize
]{python}
#1
\end{minted}
}

% Comando para tema claro específico
\newcommand{\pyclaro}[1]{%
\begin{minted}[
  bgcolor=LightBackground,
  style=default,
  breaklines,
  frame=lines,
  framesep=2mm,
  baselinestretch=1.2,
  linenos,
  fontsize=\footnotesize
]{python}
#1
\end{minted}
}
%%%%%%%%%%%%%%%%%%%%%%%%%%%%%%%%%%%%%%%%%%%%%%%%%%%%%%%%%%%%%%%%%%%%%%%%%%%%%%%%%%%%%%



%%%%%%%%%%%%%%%%%%%%%%%%%%%%%%%%%%%%%%%%%%%%%%%%%%%%%%%%%%%%%%%%%%%%%%%%%%%%%%%%%%%%%%
% Configuración global para todos los bloques minted de Python
% Usa automáticamente el tema seleccionado arriba
\setminted[python]{
  breaklines,
  frame=lines,
  framesep=2mm,
  baselinestretch=1.2,
  bgcolor=\mybgcolor,
  linenos, 
  fontsize=\footnotesize
} % Configuración dinámica según tema elegido
%%%%%%%%%%%%%%%%%%%%%%%%%%%%%%%%%%%%%%%%%%%%%%%%%%%%%%%%%%%%%%%%%%%%%%%%%%%%%%%%%%%%%%



%%%%%%%%%%%%%%%%%%%%%%%%%%%%%%%%%%%%%%%%%%%%%%%%%%%%%%%%%%%%%%%%%%%%%%%%%%%%%%%%%%%%%%
% Configuración de colores del tema con paleta azul Python
\setbeamercolor{progress bar}{fg=DarkPythonBlue,bg=LightPythonBlue}
\setbeamercolor{title separator}{fg=DarkPythonBlue,bg=white!50!black}
\setbeamercolor{frametitle}{fg=white,bg=PythonBlue}
\title[PCFI161]{Programaci\'on para F\'isica y Astronom\'ia}
\subtitle{Departamento de Física.}

\newcommand{\myfront}{
\author[PCFI161]{Corodinadora: C Loyola \\ Profesores C Femenías / F Bugini / D Basantes}
\institute[UNAB]{Universidad Andrés Bello \\ Departamento de Física y Astronomía}
\date{Primer Semestre 2025}
}

\titlegraphic{
  \includegraphics[width=.08\textwidth]{1) Logo/logo-tux.png}\hfill
  \includegraphics[width=.3\textwidth]{1) Logo/logo-unab.png}\hfill
  \includegraphics[width=.08\textwidth]{1) Logo/logo-python.png}
}

\makeatletter
\setbeamertemplate{title page}{
  \begin{minipage}[b][\paperheight]{\textwidth}
    \vfill%
    \ifx\inserttitle\@empty\else\usebeamertemplate*{title}\fi
    \ifx\insertsubtitle\@empty\else\usebeamertemplate*{subtitle}\fi
    \usebeamertemplate*{title separator}
    \ifx\beamer@shortauthor\@empty\else\usebeamertemplate*{author}\fi
    \ifx\insertdate\@empty\else\usebeamertemplate*{date}\fi
    \ifx\insertinstitute\@empty\else\usebeamertemplate*{institute}\fi
    \vfill
    \ifx\inserttitlegraphic\@empty\else\inserttitlegraphic\fi
    \vspace*{1cm}
  \end{minipage}
}
\makeatother

% Configuración personalizada del frametitle para incluir la sección
\makeatletter
\setbeamertemplate{frametitle}{%
  \nointerlineskip%
  \begin{beamercolorbox}[%
      wd=\paperwidth,%
      sep=0pt,%
      leftskip=\@ifundefined{metropolis@frametitle@padding}{12pt}{\metropolis@frametitle@padding},%
      rightskip=\@ifundefined{metropolis@frametitle@padding}{12pt}{\metropolis@frametitle@padding},%
    ]{frametitle}%
  \@ifundefined{metropolis@frametitlestrut@start}{}{\metropolis@frametitlestrut@start}%
  {\textcolor{LightPythonBlue}{\insertsectionhead}\ifx\insertsectionhead\@empty\else\ $\ni$  \fi}\insertframetitle%
  \nolinebreak%
  \@ifundefined{metropolis@frametitlestrut@end}{}{\metropolis@frametitlestrut@end}%
  \end{beamercolorbox}%
}
\makeatother

\makeatletter
% Configuración segura de longitudes del tema Metropolis
\@ifundefined{metropolis@titleseparator@linewidth}{}{\setlength{\metropolis@titleseparator@linewidth}{2pt}}
\@ifundefined{metropolis@progressonsectionpage@linewidth}{}{\setlength{\metropolis@progressonsectionpage@linewidth}{2pt}}
\@ifundefined{metropolis@progressinheadfoot@linewidth}{}{\setlength{\metropolis@progressinheadfoot@linewidth}{2pt}}
\makeatother


\usetikzlibrary{arrows.meta,positioning}
\setminted{
  fontsize=\footnotesize,   % \tiny  \scriptsize  \footnotesize  \small …
  baselinestretch=1.05      % (opcional) separación de líneas
}

\title{S15P1 (Sesión 29):\
Primeros pasos en Optimización y Paralelización con Python}
\author{PCFI161 -- Programación para Física y Astronomía}
\date{09 jun 2025}

\begin{document}
\myfront{}
\begin{frame}\titlepage\end{frame}

% Índice
\begin{frame}\frametitle{Resumen -- Semana 15, Sesión 1}\tableofcontents\end{frame}
\metroset{block=fill}

%------------------------------------------------------------------
\section{¿Por qué optimizar?}
%------------------------------------------------------------------
\begin{frame}{Motivación}
\begin{itemize}
  \item Cálculos lentos $\Rightarrow$  menos iteraciones de prueba $\Leftrightarrow$  menos aprendizaje.
  \item Simular 1 000 000 lanzamientos de dados / trayectorias $Rightarrow$ minutos vs.\ segundos.
  \item Procesar grandes catálogos astronómicos: rapidez $=$ poder hacer más ciencia.
\end{itemize}
\end{frame}

%------------------------------------------------------------------
\section{Medir antes de optimizar}
%------------------------------------------------------------------
\begin{frame}[fragile]{\%timeit (recordatorio rápido)}
\begin{minted}{python}
import numpy as np
x = np.random.random(10_000)

%timeit [v**2 for v in x]      # lista y bucle
%timeit x**2                   # vectorización NumPy
\end{minted}
\alert{Regla de oro}: perfila~\footnote{Medir cuánto tiempo y cuántos recursos consume cada parte de tu programa durante la ejecución.} primero – optimiza después.
\end{frame}

%------------------------------------------------------------------
\section{Vectorización básica}
%------------------------------------------------------------------
\begin{frame}[fragile]{Ejemplo 1 -- Energía cinética de 50\_000 partículas}
\begin{minted}{python}
import numpy as np

N = 50_000
m = np.random.uniform(1, 5,  N)          # masas 1–5 kg
v = np.random.uniform(0, 20, N)          # vel. 0–20 m/s
# Versión lenta (bucle Python)
def ke_loop(m, v):
    e = []
    for mi, vi in zip(m, v):
        e.append(0.5 * mi * vi**2)
    return e
# Versión rápida (NumPy vectorizado)
e_vec = 0.5 * m * v**2
\end{minted}

\begin{block}{\small Mide la diferencia}
\begin{minted}{python}
%timeit ke_loop(m, v)
%timeit 0.5 * m * v**2
\end{minted}
\end{block}
\end{frame}

% ===================== NUEVA SECCIÓN: Paralelización =======================
% 
% --------------------------------------------------------------------------

\section{¿Qué significa “cálculo en paralelo”?}

% ---------- Slide: Procesador multinúcleo ----------------------------------
\begin{frame}{Procesador multinúcleo: la idea}
\begin{itemize}
  \item Un \textbf{núcleo} ejecuta \emph{una} instrucción a la vez.
  \item Portátiles actuales traen 2–8 núcleos; PCs de laboratorio, 12–32.
  \item \textbf{Cálculo en paralelo}: repartir el trabajo en varios núcleos para
        terminar antes. \\
        \(\displaystyle \text{Speedup}\;S = \frac{T_{\text{serie}}}{T_{\text{paralelo}}}\)
  \item Necesitamos:  
        \begin{enumerate}
          \item Dividir datos o tareas en \(\le N_{\text{núcleos}}\) trozos.  
          \item Ejecutarlos de forma simultánea.  
          \item Reunir los resultados.  
        \end{enumerate}
  \item Python puro usa un sólo núcleo \(\Rightarrow\) librería
        \texttt{multiprocessing} crea \emph{procesos hijos} para saltar esa
        limitación.
\end{itemize}
\end{frame}

% ---------- Slide: Serie vs. Paralelo (visual) ------------------------
\begin{frame}{Serie vs.\ Paralelo – Visualmente}
\centering\small

%----------- TikZ ------------------------------------------------------
\begin{tikzpicture}[
    % Escala: 1 unidad x = 5 mm  (ajusta a gusto) -----------------------
    x=5mm, y=6mm, >=Stealth,
    task/.style  ={draw,fill=blue!20,minimum width=1,minimum height=0.6},
    label/.style ={anchor=east,font=\bfseries}
  ]
  %--------------- Fila 1 : 1 núcleo -----------------------------------
  \node[label]  at (0,0) {1 núcleo};
  \foreach \i in {0,...,7}
      \node[task] at ( 1+\i,0) {};

  %--------------- Fila 2 : 4 núcleos ----------------------------------
  \node[label]  at (0,-1.3) {4 núcleos};
  % 4 tareas se procesan a la vez → un solo «bloque de tiempo»
  \node[task,minimum width=1] at (1,-1.3) {};
  % segunda oleada
  \node[task,minimum width=1] at (2,-1.3) {};

  %--------------- Flecha de tiempo ------------------------------------
  % flecha llega justo hasta el final de la fila superior (t_serie)
  \draw[thick,->] (1,-2.4) -- (9,-2.4) node[anchor=west] {tiempo};
\end{tikzpicture}

\vspace{4mm}

\begin{itemize}
  \item \textbf{Arriba}: 8 tareas en serie \(\bigl(T_{\text{serie}}\bigr)\).
  \item \textbf{Abajo}: 4 núcleos $\Rightarrow$ 2 “oleadas” de 4 tareas  
        \(\bigl(T_{\text{paralelo}}\approx T_{\text{serie}}/4\bigr)\).
  \item El paralelismo añade \textbf{sobrecarga}: creación de procesos,
        envío / recogida de datos… Si las tareas son muy pequeñas, no se gana
        tiempo.
\end{itemize}
\end{frame}




% ---------- Slide: Pequeña tabla de speedup real ---------------------------
\begin{frame}{¿Cuánto se gana en la práctica?}
\begin{center}
\begin{tabular}{c|ccc}
\hline
\textbf{Núcleos} & 1 & 2 & 4 \\
\hline
\textbf{Serie} (\(10^6\) raíces) & \multicolumn{3}{c}{\(\;T_{\text{serie}} = 1.20\ \text{s}\)} \\
\textbf{Paralelo} & --- & \(0.66\ \text{s}\) & \(0.35\ \text{s}\) \\
\(\displaystyle S = T_{\text{serie}}/T_{\text{paralelo}}\) & 1.0 & 1.8 & 3.4 \\
\hline
\end{tabular}
\end{center}

\small
Speedup real \(< N_{\text{núcleos}}\) por la sobrecarga, pero sigue siendo una
gran mejora.
\end{frame}

% ---------- Slide: Ejemplo 2 (revisado) ------------------------------------
\section{Introducción a multiprocesamiento}

\begin{frame}[fragile]{Ejemplo 2 – Cálculo paralelo de raíces cuadradas}
\textbf{Objetivo}: medir la ganancia de dividir \(N=200\,000\) cálculos entre
todos los núcleos disponibles.

\begin{minted}{python}
import multiprocessing as mp
import math, random

N = 200_000
datos = [random.random() for _ in range(N)]

def raiz(x):
    return math.sqrt(x)

# --- Versión serie ---
%timeit [raiz(d) for d in datos]

# --- Versión paralela ---
%%timeit
with mp.Pool(processes=mp.cpu_count()) as pool:
    pool.map(raiz, datos)
\end{minted}

\begin{itemize}
  \item \texttt{Pool} crea tantos procesos como núcleos (\texttt{cpu\_count()}).
  \item Cada proceso recibe un bloque de la lista \texttt{datos}.  
  \item \texttt{pool.map} combina los resultados en el orden original.
\end{itemize}
\end{frame}



%------------------------------------------------------------------
\section{Introducción a multiprocesamiento}
%------------------------------------------------------------------
\begin{frame}[fragile]{Ejemplo 2 – Cálculo paralelo de raíces cuadradas}
\begin{minted}{python}
import multiprocessing as mp
import math, time, random

def raiz(x):                       # tarea simple
    return math.sqrt(x)

datos = [random.random() for _ in range(200_000)]

# --- Serie ---
%timeit [raiz(d) for d in datos]

# --- Paralelo (Pool.map) ---
%%timeit
with mp.Pool(processes=mp.cpu_count()) as pool:
    pool.map(raiz, datos)
\end{minted}

\small
Observa cuántos \texttt{processes} hay en tu equipo (\texttt{cpu\_count}).
\end{frame}

%------------------------------------------------------------------
\section{Ejercicios guiados en clase}
%------------------------------------------------------------------
\begin{frame}{Ejercicio A – Ciclo vs. Vectorización}
\begin{block}{Objetivo}
  \begin{itemize}
    \item Genera \(n=100\,000\) posiciones \((x,y)\) en \([-1,1]\).
    \item Calcula la distancia al origen \(r=\sqrt{x^2+y^2}\) usando:
      \begin{enumerate}
        \item bucle \texttt{for},
        \item vectorización NumPy.
      \end{enumerate}
    \item Compara tiempos con \texttt{\%timeit}.
  \end{itemize}
\end{block}
\end{frame}

\begin{frame}{Ejercicio B – Paralelizar Monte Carlo \(\pi\)}
\begin{block}{Instrucciones mínimas}
  \begin{enumerate}
    \item Divide \(N=2\cdot10^6\) puntos en 4 sub-bloques.
    \item Cada proceso genera sus puntos y cuenta cuántos caen en el círculo.
    \item Combina resultados para estimar \(\pi\).
    \item Mide tiempo serie vs.\ paralelo.
  \end{enumerate}
\end{block}
\end{frame}

%------------------------------------------------------------------
\section{Cierre}
%------------------------------------------------------------------
\begin{frame}{Para la próxima sesión}
\begin{itemize}
  \item Revisa tus notebooks: apunta los mayores speedups logrados.  
  \item Trae dudas sobre \texttt{Pool.map} o problemas de vectorización.  
  \item \alert{TAREA V} para la próxima clase.  
\end{itemize}
\end{frame}

\end{document}
