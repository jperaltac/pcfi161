\documentclass[10pt]{beamer}
\usetheme[progressbar=frametitle]{metropolis}
\usepackage{appendixnumberbeamer}
\usepackage{fancyvrb}
\usepackage{booktabs}
\usepackage[scale=2]{ccicons}
\usepackage{pgfplots}
\usepgfplotslibrary{dateplot}
\usepackage{type1cm}
\usepackage{lettrine}
\usepackage{ragged2e}
\usepackage{xspace}
\newcommand{\themename}{\textbf{\textsc{metropolis}}\xspace}
\usepackage{graphicx} % Allows including images
\usepackage{booktabs} % Allows the use of \toprule, \midrule and \bottomrule in tables
\usepackage[utf8]{inputenc} %solucion del problema de los acentos.
\usepackage{xcolor}
\usepackage{verbatim}

\definecolor{LightGray}{gray}{0.9}
% Paleta de azules estilo Python/matplotlib
\definecolor{PythonBlue}{RGB}{31, 119, 180}      % Azul principal matplotlib
\definecolor{LightPythonBlue}{RGB}{174, 199, 232} % Azul claro para fondos
\definecolor{DarkPythonBlue}{RGB}{23, 90, 135}    % Azul oscuro para líneas

% Colores para tema oscuro tipo VS Code/Colab
\definecolor{DarkBackground}{RGB}{40, 44, 52}     % Fondo oscuro similar a VS Code
\definecolor{CodeText}{RGB}{171, 178, 191}        % Texto gris claro para código
\definecolor{DarkFrame}{RGB}{60, 63, 65}          % Color del marco/borde

% Colores para tema claro
\definecolor{LightBackground}{RGB}{248, 248, 242} % Fondo claro tipo GitHub
\definecolor{LightCodeText}{RGB}{51, 51, 51}      % Texto oscuro para tema claro
\definecolor{LightFrame}{RGB}{220, 220, 220}      % Marco gris claro

\usepackage{minted}

%%%%%%%%%%%%%%%%%%%%%%%%%%%%%%%%%%%%%%%%%%%%%%%%%%%%%%%%%%%%%%%%%%%%%%%%%%%%%%%%%%%%%%
% CONFIGURACIÓN DE TEMAS - CAMBIA AQUÍ PARA ALTERNAR
% ========================================================================
% Descomenta UNA de las siguientes líneas para elegir el tema:

% TEMA OSCURO (VS Code style)
%\newcommand{\mytheme}{oscuro}
%\usemintedstyle{monokai}
%\newcommand{\mybgcolor}{DarkBackground}

% TEMA CLARO (GitHub style) - comenta las 3 líneas de arriba y descomenta estas:
 \newcommand{\mytheme}{claro}
 \usemintedstyle{default}
 \newcommand{\mybgcolor}{LightBackground}

% ========================================================================
\newcommand{\mypyfile}[1]{\inputminted[linenos=true, fontsize=\footnotesize, frame=lines, framesep=5\fboxrule,framerule=1pt]{python}{#1}}

% Comando personalizado para código Python inline
\newcommand{\pycode}[1]{\begin{minted}{python}#1\end{minted}}

% ========================================================================
% COMANDOS ESPECÍFICOS PARA CADA TEMA
% ========================================================================
% Comando para tema oscuro específico
\newcommand{\pyoscuro}[1]{%
\begin{minted}[
  bgcolor=DarkBackground,
  style=monokai,
  breaklines,
  frame=lines,
  framesep=2mm,
  baselinestretch=1.2,
  linenos,
  fontsize=\footnotesize
]{python}
#1
\end{minted}
}

% Comando para tema claro específico
\newcommand{\pyclaro}[1]{%
\begin{minted}[
  bgcolor=LightBackground,
  style=default,
  breaklines,
  frame=lines,
  framesep=2mm,
  baselinestretch=1.2,
  linenos,
  fontsize=\footnotesize
]{python}
#1
\end{minted}
}
%%%%%%%%%%%%%%%%%%%%%%%%%%%%%%%%%%%%%%%%%%%%%%%%%%%%%%%%%%%%%%%%%%%%%%%%%%%%%%%%%%%%%%



%%%%%%%%%%%%%%%%%%%%%%%%%%%%%%%%%%%%%%%%%%%%%%%%%%%%%%%%%%%%%%%%%%%%%%%%%%%%%%%%%%%%%%
% Configuración global para todos los bloques minted de Python
% Usa automáticamente el tema seleccionado arriba
\setminted[python]{
  breaklines,
  frame=lines,
  framesep=2mm,
  baselinestretch=1.2,
  bgcolor=\mybgcolor,
  linenos, 
  fontsize=\footnotesize
} % Configuración dinámica según tema elegido
%%%%%%%%%%%%%%%%%%%%%%%%%%%%%%%%%%%%%%%%%%%%%%%%%%%%%%%%%%%%%%%%%%%%%%%%%%%%%%%%%%%%%%



%%%%%%%%%%%%%%%%%%%%%%%%%%%%%%%%%%%%%%%%%%%%%%%%%%%%%%%%%%%%%%%%%%%%%%%%%%%%%%%%%%%%%%
% Configuración de colores del tema con paleta azul Python
\setbeamercolor{progress bar}{fg=DarkPythonBlue,bg=LightPythonBlue}
\setbeamercolor{title separator}{fg=DarkPythonBlue,bg=white!50!black}
\setbeamercolor{frametitle}{fg=white,bg=PythonBlue}
\title[PCFI161]{Programaci\'on para F\'isica y Astronom\'ia}
\subtitle{Departamento de Física.}

\newcommand{\myfront}{
\author[PCFI161]{Corodinadora: C Loyola \\ Profesores C Femenías / F Bugini / D Basantes}
\institute[UNAB]{Universidad Andrés Bello \\ Departamento de Física y Astronomía}
\date{Primer Semestre 2025}
}

\titlegraphic{
  \includegraphics[width=.08\textwidth]{1) Logo/logo-tux.png}\hfill
  \includegraphics[width=.3\textwidth]{1) Logo/logo-unab.png}\hfill
  \includegraphics[width=.08\textwidth]{1) Logo/logo-python.png}
}

\makeatletter
\setbeamertemplate{title page}{
  \begin{minipage}[b][\paperheight]{\textwidth}
    \vfill%
    \ifx\inserttitle\@empty\else\usebeamertemplate*{title}\fi
    \ifx\insertsubtitle\@empty\else\usebeamertemplate*{subtitle}\fi
    \usebeamertemplate*{title separator}
    \ifx\beamer@shortauthor\@empty\else\usebeamertemplate*{author}\fi
    \ifx\insertdate\@empty\else\usebeamertemplate*{date}\fi
    \ifx\insertinstitute\@empty\else\usebeamertemplate*{institute}\fi
    \vfill
    \ifx\inserttitlegraphic\@empty\else\inserttitlegraphic\fi
    \vspace*{1cm}
  \end{minipage}
}
\makeatother

% Configuración personalizada del frametitle para incluir la sección
\makeatletter
\setbeamertemplate{frametitle}{%
  \nointerlineskip%
  \begin{beamercolorbox}[%
      wd=\paperwidth,%
      sep=0pt,%
      leftskip=\@ifundefined{metropolis@frametitle@padding}{12pt}{\metropolis@frametitle@padding},%
      rightskip=\@ifundefined{metropolis@frametitle@padding}{12pt}{\metropolis@frametitle@padding},%
    ]{frametitle}%
  \@ifundefined{metropolis@frametitlestrut@start}{}{\metropolis@frametitlestrut@start}%
  {\textcolor{LightPythonBlue}{\insertsectionhead}\ifx\insertsectionhead\@empty\else\ $\ni$  \fi}\insertframetitle%
  \nolinebreak%
  \@ifundefined{metropolis@frametitlestrut@end}{}{\metropolis@frametitlestrut@end}%
  \end{beamercolorbox}%
}
\makeatother

\makeatletter
% Configuración segura de longitudes del tema Metropolis
\@ifundefined{metropolis@titleseparator@linewidth}{}{\setlength{\metropolis@titleseparator@linewidth}{2pt}}
\@ifundefined{metropolis@progressonsectionpage@linewidth}{}{\setlength{\metropolis@progressonsectionpage@linewidth}{2pt}}
\@ifundefined{metropolis@progressinheadfoot@linewidth}{}{\setlength{\metropolis@progressinheadfoot@linewidth}{2pt}}
\makeatother



\title{Semana 15 -- Sesión 2 (Sesión 30):\
Taller Evaluado y Lanzamiento de la TAREA VI}
\author{PCFI161}
\date{11 jun 2025}

\begin{document}
\myfront{}
\begin{frame}\titlepage\end{frame}

% Índice
\begin{frame}\frametitle{Resumen -- Semana 15, Sesión 2}\tableofcontents\end{frame}
\metroset{block=fill}

%------------------------------------------------------------------
\section{Objetivos del Taller}
%------------------------------------------------------------------
\begin{frame}{Qué haremos hoy}
\begin{itemize}
  \item Aplicar vectorización y multiprocesamiento en un problema sencillo.  
  \item Medir correctamente tiempos con \texttt{\%timeit}.  
  \item Entregar un notebook (\textbf{Taller 15}) en 45 min.  
  \item Presentar la \textbf{TAREA VI} (para casa, 1 semana).  
\end{itemize}
\end{frame}

%------------------------------------------------------------------
\section{Datos y tarea del taller}
%------------------------------------------------------------------
\begin{frame}{Datasets}
\begin{itemize}
  \item \texttt{velocidades.csv} -- 50 000 velocidades (m/s).  
  \item \texttt{random\_points.npy} -- matriz \(1\times10^6\) puntos \((x,y)\).  
\end{itemize}
\end{frame}

\begin{frame}{Taller 15 -- Instrucciones}
\textbf{Parte A -- Energía cinética (20 min)}
\begin{enumerate}
  \item Cargar velocidades y masa \(m=0.150\;\text{kg}\).  
  \item Calcular energía cinética con bucle y con vectorización.  
  \item Medir tiempos y listar mínimo, máximo, promedio de \(E_k\).
\end{enumerate}

\textbf{Parte B -- Monte Carlo \(\pi\) (25 min)}
\begin{enumerate}
  \setcounter{enumi}{3}
  \item Leer \texttt{random\_points.npy}.  
  \item Implementar estimación de \(\pi\) serie y luego paralela con \texttt{multiprocessing.Pool}.  
  \item Registrar speedup (\texttt{tiempo\_serie/tiempo\_paralelo}).  
\end{enumerate}

\textbf{Entrega}: \texttt{Taller15\_GrupoX.ipynb} en CANVAS.
\end{frame}

%------------------------------------------------------------------
\section{Pauta de evaluación}
%------------------------------------------------------------------
\begin{frame}{Criterios de corrección}
\begin{itemize}
  \item Funciona y produce resultados correctos (40 %).  
  \item Uso apropiado de vectorización / multiprocesos (25 %).  
  \item Medición de tiempos y análisis breve (20 %).  
  \item Claridad de código y comentarios (15 %).  
\end{itemize}
\end{frame}

%------------------------------------------------------------------
\section{Lanzamiento de la TAREA VI}
%------------------------------------------------------------------
\begin{frame}{TAREA V -- Mini-proyecto de optimización}
\begin{block}{Resumen}
  \begin{itemize}
    \item Elegir \textbf{uno} de 3 retos (detallados en CANVAS):  
      \begin{enumerate}
        \item Simulación de caída libre con resistencia (vectorizar vs.\ bucle).  
        \item Búsqueda del número primo más grande < 10\(^7\) (serie vs.\ parallel).  
        \item Procesar un espectro (FFT) de 5 MB (optimizar con NumPy).  
      \end{enumerate}
    \item Comparar dos versiones y reportar speedup.  
    \item Entregar notebook + PDF breve (máx. 3 pág.) antes del \textbf{18 jun 2025, 23:55}.  
  \end{itemize}
\end{block}
\end{frame}

%------------------------------------------------------------------
\section{Cuenta regresiva del taller}
%------------------------------------------------------------------
\begin{frame}{¡A trabajar!}
\Large{\centering 45 min de tiempo efectivo}\\[0.4cm]
\begin{itemize}
  \item Reparte roles: Parte A / Parte B.  
  \item Usa \texttt{\%\%timeit} para bloques largos.  
  \item Guarda periódicamente.
\end{itemize}
\end{frame}

%------------------------------------------------------------------
\section{Cierre}
%------------------------------------------------------------------
\begin{frame}{Próximos pasos}
\begin{itemize}
  \item Feedback del Taller 15 en 48 h.  
  \item TAREA VI: 1 semana. Dudas → foro CANVAS.  
\end{itemize}
\end{frame}

\end{document}