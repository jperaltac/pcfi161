\documentclass[10pt]{beamer}
\usetheme[progressbar=frametitle]{metropolis}
\usepackage{appendixnumberbeamer}
\usepackage{fancyvrb}
\usepackage{booktabs}
\usepackage[scale=2]{ccicons}
\usepackage{pgfplots}
\usepgfplotslibrary{dateplot}
\usepackage{type1cm}
\usepackage{lettrine}
\usepackage{ragged2e}
\usepackage{xspace}
\newcommand{\themename}{\textbf{\textsc{metropolis}}\xspace}
\usepackage{graphicx} % Allows including images
\usepackage{booktabs} % Allows the use of \toprule, \midrule and \bottomrule in tables
\usepackage[utf8]{inputenc} %solucion del problema de los acentos.
\usepackage{xcolor}
\usepackage{verbatim}

\definecolor{LightGray}{gray}{0.9}
% Paleta de azules estilo Python/matplotlib
\definecolor{PythonBlue}{RGB}{31, 119, 180}      % Azul principal matplotlib
\definecolor{LightPythonBlue}{RGB}{174, 199, 232} % Azul claro para fondos
\definecolor{DarkPythonBlue}{RGB}{23, 90, 135}    % Azul oscuro para líneas

% Colores para tema oscuro tipo VS Code/Colab
\definecolor{DarkBackground}{RGB}{40, 44, 52}     % Fondo oscuro similar a VS Code
\definecolor{CodeText}{RGB}{171, 178, 191}        % Texto gris claro para código
\definecolor{DarkFrame}{RGB}{60, 63, 65}          % Color del marco/borde

% Colores para tema claro
\definecolor{LightBackground}{RGB}{248, 248, 242} % Fondo claro tipo GitHub
\definecolor{LightCodeText}{RGB}{51, 51, 51}      % Texto oscuro para tema claro
\definecolor{LightFrame}{RGB}{220, 220, 220}      % Marco gris claro

\usepackage{minted}

%%%%%%%%%%%%%%%%%%%%%%%%%%%%%%%%%%%%%%%%%%%%%%%%%%%%%%%%%%%%%%%%%%%%%%%%%%%%%%%%%%%%%%
% CONFIGURACIÓN DE TEMAS - CAMBIA AQUÍ PARA ALTERNAR
% ========================================================================
% Descomenta UNA de las siguientes líneas para elegir el tema:

% TEMA OSCURO (VS Code style)
%\newcommand{\mytheme}{oscuro}
%\usemintedstyle{monokai}
%\newcommand{\mybgcolor}{DarkBackground}

% TEMA CLARO (GitHub style) - comenta las 3 líneas de arriba y descomenta estas:
 \newcommand{\mytheme}{claro}
 \usemintedstyle{default}
 \newcommand{\mybgcolor}{LightBackground}

% ========================================================================
\newcommand{\mypyfile}[1]{\inputminted[linenos=true, fontsize=\footnotesize, frame=lines, framesep=5\fboxrule,framerule=1pt]{python}{#1}}

% Comando personalizado para código Python inline
\newcommand{\pycode}[1]{\begin{minted}{python}#1\end{minted}}

% ========================================================================
% COMANDOS ESPECÍFICOS PARA CADA TEMA
% ========================================================================
% Comando para tema oscuro específico
\newcommand{\pyoscuro}[1]{%
\begin{minted}[
  bgcolor=DarkBackground,
  style=monokai,
  breaklines,
  frame=lines,
  framesep=2mm,
  baselinestretch=1.2,
  linenos,
  fontsize=\footnotesize
]{python}
#1
\end{minted}
}

% Comando para tema claro específico
\newcommand{\pyclaro}[1]{%
\begin{minted}[
  bgcolor=LightBackground,
  style=default,
  breaklines,
  frame=lines,
  framesep=2mm,
  baselinestretch=1.2,
  linenos,
  fontsize=\footnotesize
]{python}
#1
\end{minted}
}
%%%%%%%%%%%%%%%%%%%%%%%%%%%%%%%%%%%%%%%%%%%%%%%%%%%%%%%%%%%%%%%%%%%%%%%%%%%%%%%%%%%%%%



%%%%%%%%%%%%%%%%%%%%%%%%%%%%%%%%%%%%%%%%%%%%%%%%%%%%%%%%%%%%%%%%%%%%%%%%%%%%%%%%%%%%%%
% Configuración global para todos los bloques minted de Python
% Usa automáticamente el tema seleccionado arriba
\setminted[python]{
  breaklines,
  frame=lines,
  framesep=2mm,
  baselinestretch=1.2,
  bgcolor=\mybgcolor,
  linenos, 
  fontsize=\footnotesize
} % Configuración dinámica según tema elegido
%%%%%%%%%%%%%%%%%%%%%%%%%%%%%%%%%%%%%%%%%%%%%%%%%%%%%%%%%%%%%%%%%%%%%%%%%%%%%%%%%%%%%%



%%%%%%%%%%%%%%%%%%%%%%%%%%%%%%%%%%%%%%%%%%%%%%%%%%%%%%%%%%%%%%%%%%%%%%%%%%%%%%%%%%%%%%
% Configuración de colores del tema con paleta azul Python
\setbeamercolor{progress bar}{fg=DarkPythonBlue,bg=LightPythonBlue}
\setbeamercolor{title separator}{fg=DarkPythonBlue,bg=white!50!black}
\setbeamercolor{frametitle}{fg=white,bg=PythonBlue}
\title[PCFI161]{Programaci\'on para F\'isica y Astronom\'ia}
\subtitle{Departamento de Física.}

\newcommand{\myfront}{
\author[PCFI161]{Corodinadora: C Loyola \\ Profesores C Femenías / F Bugini / D Basantes}
\institute[UNAB]{Universidad Andrés Bello \\ Departamento de Física y Astronomía}
\date{Primer Semestre 2025}
}

\titlegraphic{
  \includegraphics[width=.08\textwidth]{1) Logo/logo-tux.png}\hfill
  \includegraphics[width=.3\textwidth]{1) Logo/logo-unab.png}\hfill
  \includegraphics[width=.08\textwidth]{1) Logo/logo-python.png}
}

\makeatletter
\setbeamertemplate{title page}{
  \begin{minipage}[b][\paperheight]{\textwidth}
    \vfill%
    \ifx\inserttitle\@empty\else\usebeamertemplate*{title}\fi
    \ifx\insertsubtitle\@empty\else\usebeamertemplate*{subtitle}\fi
    \usebeamertemplate*{title separator}
    \ifx\beamer@shortauthor\@empty\else\usebeamertemplate*{author}\fi
    \ifx\insertdate\@empty\else\usebeamertemplate*{date}\fi
    \ifx\insertinstitute\@empty\else\usebeamertemplate*{institute}\fi
    \vfill
    \ifx\inserttitlegraphic\@empty\else\inserttitlegraphic\fi
    \vspace*{1cm}
  \end{minipage}
}
\makeatother

% Configuración personalizada del frametitle para incluir la sección
\makeatletter
\setbeamertemplate{frametitle}{%
  \nointerlineskip%
  \begin{beamercolorbox}[%
      wd=\paperwidth,%
      sep=0pt,%
      leftskip=\@ifundefined{metropolis@frametitle@padding}{12pt}{\metropolis@frametitle@padding},%
      rightskip=\@ifundefined{metropolis@frametitle@padding}{12pt}{\metropolis@frametitle@padding},%
    ]{frametitle}%
  \@ifundefined{metropolis@frametitlestrut@start}{}{\metropolis@frametitlestrut@start}%
  {\textcolor{LightPythonBlue}{\insertsectionhead}\ifx\insertsectionhead\@empty\else\ $\ni$  \fi}\insertframetitle%
  \nolinebreak%
  \@ifundefined{metropolis@frametitlestrut@end}{}{\metropolis@frametitlestrut@end}%
  \end{beamercolorbox}%
}
\makeatother

\makeatletter
% Configuración segura de longitudes del tema Metropolis
\@ifundefined{metropolis@titleseparator@linewidth}{}{\setlength{\metropolis@titleseparator@linewidth}{2pt}}
\@ifundefined{metropolis@progressonsectionpage@linewidth}{}{\setlength{\metropolis@progressonsectionpage@linewidth}{2pt}}
\@ifundefined{metropolis@progressinheadfoot@linewidth}{}{\setlength{\metropolis@progressinheadfoot@linewidth}{2pt}}
\makeatother



% ----------  NUEVO: minted ----------
\usepackage{minted}        % Requiere compilar con -shell-escape
\setminted{
  fontsize=\tiny,
  linenos,
  baselinestretch=1,
  breaklines
}

\title{S15P2 (Sesión 30)\\Repaso guiado y Tarea V}
\author{PCFI161}
\date{11 jun 2025}

\begin{document}
\myfront{}

% ───────────────────────────────────────────────
\begin{frame}
  \titlepage
\end{frame}

% Índice de la sesión
\begin{frame}
  \frametitle{Ruta de la sesión (60 min)}
  \tableofcontents
\end{frame}

\metroset{block=fill}

% ============================================================
\section{Repaso \& Calentamiento (40 min)}
% ============================================================

% ------------------------------------------------------------
\begin{frame}{Mini-quiz relámpago (3 min)}
\begin{block}{¿Recuerdas…?}
\begin{enumerate}
  \item ¿Qué hace \texttt{\%timeit -n 10}?
  \item ¿Cómo sabes cuántos núcleos tiene tu CPU en Python?
  \item ¿Qué función NumPy usarías para reemplazar un bucle que calcula $x^2$ sobre un vector?
\end{enumerate}
\end{block}
{\scriptsize Respuestas en voz alta, sin nota.}
\end{frame}

% ------------------------------------------------------------
\begin{frame}[fragile]{Vectorización vs. bucle – Ejemplo sintético (10 min)}
\vspace{-0.3cm}
\begin{minted}{python}
# --- Setup (solo una vez) ---
import numpy as np
np.random.seed(42)
N = 50_000
v = np.random.uniform(0, 30, N)
m = 0.150
\end{minted}
\vspace{-0.3cm}
\begin{minted}{python}
# --- 1) bucle puro Python ---
%%timeit
E_loop = [0.5*m*vi**2 for vi in v]
\end{minted}
\vspace{-0.3cm}
\begin{minted}{python}
# --- 2) vectorizado NumPy ---
%%timeit
E_vec = 0.5*m*v**2
\end{minted}

\begin{itemize}\small
  \item ¿Son identidos los tiempos? ¿Y los valores obtenidos (busque información sobre \texttt{np.allclose})? 
  \item Pregunta: ¿por qué la orden \texttt{v**2} es tan veloz?
\end{itemize}
\end{frame}

% ------------------------------------------------------------
\begin{frame}[fragile]{Midiendo correctamente con \%timeit (5 min)}
\begin{minted}{python}
%timeit [0.5 * m * vi**2 for vi in v]          # bucle Python
%timeit 0.5 * m * v**2                        # vectorizado
\end{minted}
\begin{block}{Buenas prácticas}
\begin{itemize}
  \item Usa \texttt{-n} y \texttt{-r} para fijar repeticiones si los tiempos son muy bajos.
  \item Mide funciones aisladas con la magia \texttt{\%timeit} sobre el nombre.
  \item Reporta siempre el mejor tiempo o el promedio junto al desvío.
\end{itemize}
\end{block}
\end{frame}

% ------------------------------------------------------------
%\begin{frame}[fragile]{Paralelismo rápido: Monte Carlo $\pi$ (10 min)}
%\begin{minted}{python}
%import numpy as np, multiprocessing as mp
%points = np.load("random_points.npy")  # (1e6, 2)
%
%# Versión serie
%def estimate_pi_serial(pts):
%    inside = np.sum(np.square(pts).sum(axis=1) <= 1)
%    return 4 * inside / len(pts)
%# Versión paralela
%cores = mp.cpu_count()
%
%def _count_inside(chunk):
%    return np.sum(np.square(chunk).sum(axis=1) <= 1)
%
%with mp.Pool(cores) as pool:
%    chunks = np.array_split(points, cores)
%    inside_total = sum(pool.map(_count_inside, chunks))
%pi_parallel = 4 * inside_total / len(points)
%\end{minted}
%\begin{itemize}
%  \item Discusión: Speedup teórico vs. medido con \texttt{\%timeit}.
%  \item Overhead de \texttt{multiprocessing} para problemas pequeños.
%\end{itemize}
%\end{frame}

% ------------------------------------------------------------
\begin{frame}{Transición a la Tarea}
\Large
Repaso completado. ¡A trabajar en la Tarea V!

\vspace{0.4cm}
\small
Recuerden crear su notebook grupal antes de comenzar.
\end{frame}

% ============================================================
\section{Tarea V (20 min)}
% ============================================================

\begin{frame}{Datasets proporcionados}
\begin{itemize}
  \item \texttt{velocidades.csv} — 50 000 velocidades (m/s).
  \item url:
  {\footnotesize \texttt{https://gitarra.cl/lectures/gfiles/-/raw/main/pcfi161/S15/velocidades.csv}}
  %\item \texttt{random\_points.npy} — matriz $1\times10^6$ puntos $(x,y)$ uniformes en $[-1,1]^2$.
\end{itemize}
\end{frame}

\begin{frame}{Instrucciones}
\textbf{Energía cinética}
\begin{enumerate}
  \item Cargar las velocidades y usar masa $m = 0.150\,\text{kg}$.
  \item Calcular la energía cinética con (a) bucle clásico y (b) vectorización NumPy.
  \item Medir tiempos con \texttt{\%timeit}. Reportar $\min$, $\max$ y $\bar{E_k}$.
\end{enumerate}

\textbf{Desafío adicional}: Modifique el nombre del archivo a velocidades\_big.csv (el mismo PATH), y observe nuevamente los tiempos.

%\textbf{Parte B — Monte Carlo $\pi$ (10 min)}
%\begin{enumerate}
%  \setcounter{enumi}{3}
%  \item Leer \texttt{random\_points.npy}.
%  \item Implementar estimación de $\pi$ (i) serie y (ii) paralela con \texttt{multiprocessing.Pool}.
%  \item Calcular el \textit{speedup}: $S = t_{\text{serie}}/t_{\text{paralelo}}$.
%\end{enumerate}

\textbf{Entrega}: \texttt{Tarea15\_GrupoX.ipynb} antes del cierre en \textsc{Canvas}.
\end{frame}

% ============================================================
\section{Pauta de evaluación}
% ============================================================

\begin{frame}{Criterios de corrección}
\begin{itemize}
  \item Resultados correctos y reproducibles — 40 \%.
  \item Uso apropiado de vectorización / multiprocesamiento — 25 \%.
  \item Medición de tiempos y breve análisis — 20 \%.
  \item Claridad de código y comentarios — 15 \%.
\end{itemize}
\end{frame}

% ============================================================
\section{Cierre}
% ============================================================
\begin{frame}{Próximos pasos}
\begin{itemize}
  \item Suban su notebook antes de salir de la sala.
  \item Si tienen dudas, ¡pregunten al profesor o ayudante!
  \item La próxima semana comenzamos a preparar nuestra solemne.
\end{itemize}
\end{frame}

\end{document}