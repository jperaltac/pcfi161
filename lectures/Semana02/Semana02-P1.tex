\documentclass[10pt]{beamer}

% Copia local de preamble.tex:
\usetheme[progressbar=frametitle]{metropolis}
\usepackage{appendixnumberbeamer}
\usepackage{fancyvrb}
\usepackage{booktabs}
\usepackage[scale=2]{ccicons}
\usepackage{pgfplots}
\usepgfplotslibrary{dateplot}
\usepackage{type1cm}
\usepackage{lettrine}
\usepackage{ragged2e}
\usepackage{xspace}
\newcommand{\themename}{\textbf{\textsc{metropolis}}\xspace}
\usepackage{graphicx} % Allows including images
\usepackage{booktabs} % Allows the use of \toprule, \midrule and \bottomrule in tables
\usepackage[utf8]{inputenc} %solucion del problema de los acentos.
\usepackage{xcolor}
\usepackage{verbatim}

\definecolor{LightGray}{gray}{0.9}
% Paleta de azules estilo Python/matplotlib
\definecolor{PythonBlue}{RGB}{31, 119, 180}      % Azul principal matplotlib
\definecolor{LightPythonBlue}{RGB}{174, 199, 232} % Azul claro para fondos
\definecolor{DarkPythonBlue}{RGB}{23, 90, 135}    % Azul oscuro para líneas

% Colores para tema oscuro tipo VS Code/Colab
\definecolor{DarkBackground}{RGB}{40, 44, 52}     % Fondo oscuro similar a VS Code
\definecolor{CodeText}{RGB}{171, 178, 191}        % Texto gris claro para código
\definecolor{DarkFrame}{RGB}{60, 63, 65}          % Color del marco/borde

% Colores para tema claro
\definecolor{LightBackground}{RGB}{248, 248, 242} % Fondo claro tipo GitHub
\definecolor{LightCodeText}{RGB}{51, 51, 51}      % Texto oscuro para tema claro
\definecolor{LightFrame}{RGB}{220, 220, 220}      % Marco gris claro

\usepackage{minted}

%%%%%%%%%%%%%%%%%%%%%%%%%%%%%%%%%%%%%%%%%%%%%%%%%%%%%%%%%%%%%%%%%%%%%%%%%%%%%%%%%%%%%%
% CONFIGURACIÓN DE TEMAS - CAMBIA AQUÍ PARA ALTERNAR
% ========================================================================
% Descomenta UNA de las siguientes líneas para elegir el tema:

% TEMA OSCURO (VS Code style)
%\newcommand{\mytheme}{oscuro}
%\usemintedstyle{monokai}
%\newcommand{\mybgcolor}{DarkBackground}

% TEMA CLARO (GitHub style) - comenta las 3 líneas de arriba y descomenta estas:
 \newcommand{\mytheme}{claro}
 \usemintedstyle{default}
 \newcommand{\mybgcolor}{LightBackground}

% ========================================================================
\newcommand{\mypyfile}[1]{\inputminted[linenos=true, fontsize=\footnotesize, frame=lines, framesep=5\fboxrule,framerule=1pt]{python}{#1}}

% Comando personalizado para código Python inline
\newcommand{\pycode}[1]{\begin{minted}{python}#1\end{minted}}

% ========================================================================
% COMANDOS ESPECÍFICOS PARA CADA TEMA
% ========================================================================
% Comando para tema oscuro específico
\newcommand{\pyoscuro}[1]{%
\begin{minted}[
  bgcolor=DarkBackground,
  style=monokai,
  breaklines,
  frame=lines,
  framesep=2mm,
  baselinestretch=1.2,
  linenos,
  fontsize=\footnotesize
]{python}
#1
\end{minted}
}

% Comando para tema claro específico
\newcommand{\pyclaro}[1]{%
\begin{minted}[
  bgcolor=LightBackground,
  style=default,
  breaklines,
  frame=lines,
  framesep=2mm,
  baselinestretch=1.2,
  linenos,
  fontsize=\footnotesize
]{python}
#1
\end{minted}
}
%%%%%%%%%%%%%%%%%%%%%%%%%%%%%%%%%%%%%%%%%%%%%%%%%%%%%%%%%%%%%%%%%%%%%%%%%%%%%%%%%%%%%%



%%%%%%%%%%%%%%%%%%%%%%%%%%%%%%%%%%%%%%%%%%%%%%%%%%%%%%%%%%%%%%%%%%%%%%%%%%%%%%%%%%%%%%
% Configuración global para todos los bloques minted de Python
% Usa automáticamente el tema seleccionado arriba
\setminted[python]{
  breaklines,
  frame=lines,
  framesep=2mm,
  baselinestretch=1.2,
  bgcolor=\mybgcolor,
  linenos, 
  fontsize=\footnotesize
} % Configuración dinámica según tema elegido
%%%%%%%%%%%%%%%%%%%%%%%%%%%%%%%%%%%%%%%%%%%%%%%%%%%%%%%%%%%%%%%%%%%%%%%%%%%%%%%%%%%%%%



%%%%%%%%%%%%%%%%%%%%%%%%%%%%%%%%%%%%%%%%%%%%%%%%%%%%%%%%%%%%%%%%%%%%%%%%%%%%%%%%%%%%%%
% Configuración de colores del tema con paleta azul Python
\setbeamercolor{progress bar}{fg=DarkPythonBlue,bg=LightPythonBlue}
\setbeamercolor{title separator}{fg=DarkPythonBlue,bg=white!50!black}
\setbeamercolor{frametitle}{fg=white,bg=PythonBlue}
\title[PCFI161]{Programaci\'on para F\'isica y Astronom\'ia}
\subtitle{Departamento de Física.}

\newcommand{\myfront}{
\author[PCFI161]{Corodinadora: C Loyola \\ Profesores C Femenías / F Bugini / D Basantes}
\institute[UNAB]{Universidad Andrés Bello \\ Departamento de Física y Astronomía}
\date{Primer Semestre 2025}
}

\titlegraphic{
  \includegraphics[width=.08\textwidth]{1) Logo/logo-tux.png}\hfill
  \includegraphics[width=.3\textwidth]{1) Logo/logo-unab.png}\hfill
  \includegraphics[width=.08\textwidth]{1) Logo/logo-python.png}
}

\makeatletter
\setbeamertemplate{title page}{
  \begin{minipage}[b][\paperheight]{\textwidth}
    \vfill%
    \ifx\inserttitle\@empty\else\usebeamertemplate*{title}\fi
    \ifx\insertsubtitle\@empty\else\usebeamertemplate*{subtitle}\fi
    \usebeamertemplate*{title separator}
    \ifx\beamer@shortauthor\@empty\else\usebeamertemplate*{author}\fi
    \ifx\insertdate\@empty\else\usebeamertemplate*{date}\fi
    \ifx\insertinstitute\@empty\else\usebeamertemplate*{institute}\fi
    \vfill
    \ifx\inserttitlegraphic\@empty\else\inserttitlegraphic\fi
    \vspace*{1cm}
  \end{minipage}
}
\makeatother

% Configuración personalizada del frametitle para incluir la sección
\makeatletter
\setbeamertemplate{frametitle}{%
  \nointerlineskip%
  \begin{beamercolorbox}[%
      wd=\paperwidth,%
      sep=0pt,%
      leftskip=\@ifundefined{metropolis@frametitle@padding}{12pt}{\metropolis@frametitle@padding},%
      rightskip=\@ifundefined{metropolis@frametitle@padding}{12pt}{\metropolis@frametitle@padding},%
    ]{frametitle}%
  \@ifundefined{metropolis@frametitlestrut@start}{}{\metropolis@frametitlestrut@start}%
  {\textcolor{LightPythonBlue}{\insertsectionhead}\ifx\insertsectionhead\@empty\else\ $\ni$  \fi}\insertframetitle%
  \nolinebreak%
  \@ifundefined{metropolis@frametitlestrut@end}{}{\metropolis@frametitlestrut@end}%
  \end{beamercolorbox}%
}
\makeatother

\makeatletter
% Configuración segura de longitudes del tema Metropolis
\@ifundefined{metropolis@titleseparator@linewidth}{}{\setlength{\metropolis@titleseparator@linewidth}{2pt}}
\@ifundefined{metropolis@progressonsectionpage@linewidth}{}{\setlength{\metropolis@progressonsectionpage@linewidth}{2pt}}
\@ifundefined{metropolis@progressinheadfoot@linewidth}{}{\setlength{\metropolis@progressinheadfoot@linewidth}{2pt}}
\makeatother



\begin{document}

\myfront{}

\begin{frame}
  \titlepage
\end{frame}

\begin{frame}
  \frametitle{Resumen - Parte 1 (Semana 2)}
  \tableofcontents
\end{frame}

%----------------------------------------------------------------------------------------
%   PRESENTATION SLIDES
%----------------------------------------------------------------------------------------
\metroset{block=fill}

%------------------------------------------------
\section{BASH, comandos fundamentales}

\subsection{Elementos B\'asicos de BASH}

\begin{frame}{¿Qué es BASH?}
	\begin{itemize}
		\item Bash es una \texttt{shell} interactiva que se carga por defecto en la mayoría de las distribuciones GNU/Linux basadas en Debian.
		\item Existen otras \texttt{shell}, como csh o tcsh, pero nos centramos en bash por ser muy común.
		\item Al abrir un terminal, éste siempre abre un shell. Cuando es \texttt{bash}, suele verse:
        \begin{block}{Prompt típico de BASH}
        \texttt{username@machine:~\$}
        \end{block}
        \item El símbolo \texttt{\$} indica que está listo para recibir instrucciones.
	\end{itemize}
\end{frame}

\begin{frame}{Repasando lo básico}
	\begin{itemize}
		\item Algunas instrucciones básicas de bash:
        \begin{itemize}
        	\item \texttt{pwd}: Muestra en qué directorio estamos.
            \item \texttt{ls}: Lista el contenido del directorio.
            \item \texttt{cp}: Copia archivos y/o directorios.
            \item \texttt{mkdir}: Crea un nuevo directorio.
            \item \texttt{rm}: Elimina archivos y/o directorios.
        \end{itemize}
        \item Para ver una lista más completa:  
              https://ss64.com/bash/
        \item En la filosofía GNU/Linux, muchas herramientas pequeñas (comandos) pueden \emph{encadenarse} para resolver problemas complejos.  
        \item Esto se logra con las tuberías (\texttt{|}), que permiten conectar la salida de un comando con la entrada de otro.
	\end{itemize}
\end{frame}

\begin{frame}[fragile]{Tuberías (\textit{Pipes})}
	\begin{itemize}
		\item El uso de tuberías en GNU/Linux es una herramienta muy potente.
        \item Ejemplo: tenemos \(\sim 150000\) archivos (\texttt{fichero1.dat}, \texttt{fichero2.dat}, …, \texttt{fichero150000.dat}). Cada archivo tiene 5 columnas y queremos sumar la columna 5 en todos los archivos.
        \item ¿Cómo hacerlo eficientemente?
	\end{itemize}
\end{frame}

\begin{frame}[fragile]{Tuberías (\textit{Pipes})}
	\begin{itemize}
		\item Se puede usar \texttt{sed} o \texttt{awk} para “filtrar” datos.
		\item Para cada archivo:
        \begin{itemize}
        	\item \textbf{Iterar} sobre cada archivo.
            \item \textbf{Filtrar} (con \texttt{awk}) la quinta columna.
            \item \textbf{Sumar} esos valores (también con \texttt{awk}).
        \end{itemize}
	\end{itemize}
\end{frame}

\begin{frame}[fragile]{Tuberías (\textit{Pipes}) - Ejemplo}
\begin{minted}[fontsize=\scriptsize]{bash}
for i in `seq 1 150000`; do
  cat fichero$i.dat | awk '{SUM += $5} END {print SUM}' >> sums-c5.dat
done
cat sums-c5.dat | awk '{SUM += $1} END { print SUM/150000 }'
\end{minted}

\begin{block}{Nota}
Al final del curso, se espera que puedas componer este tipo de soluciones.  
Ahora basta con comprender el razonamiento de \textbf{encadenar} comandos.
\end{block}
\end{frame}

%------------------------------------------------
\begin{frame}[fragile]{Algunas cosas útiles de \texttt{awk}}
\texttt{awk} es un programa para manejar columnas. Es muy poderoso y programable.  
Ejemplos:
\begin{table}\footnotesize
\begin{tabular}{ll}
\textbf{Comando} & \textbf{Descripción} \\
\verb|awk 'NR % 6'| & Imprime todo excepto líneas divisibles por 6. \\
\verb|awk 'NR > 5'| & Imprime a partir de la línea 6. \\
\verb|awk '$2 == "foo"'| & Imprime líneas cuyo segundo campo sea “foo”. \\
\verb|awk 'NF >= 6'| & Imprime líneas con 6 o más campos. \\
\verb|awk '/foo/ && /bar/'| & Imprime líneas que contengan /foo/ y /bar/. \\
\verb|awk '/foo/,/bar/'| & Imprime desde línea con /foo/ hasta línea con /bar/. \\
\verb|awk 'NF'| & Imprime sólo líneas no vacías. \\
\verb|awk 'NF--'| & Remueve el último campo e imprime la línea. \\
\verb|awk '$0 = NR" "$0'| & Antepone número de línea a cada fila.
\end{tabular}
\end{table}
\end{frame}

\begin{frame}[fragile]{Algunas cosas útiles de \texttt{sed}}
\texttt{sed} es para manejo de filas y edición rápida.  
Ejemplos:
\begin{table}\footnotesize
\begin{tabular}{ll}
\textbf{Comando} & \textbf{Descripción} \\
\verb|sed -n '5,10p' file.txt| & Imprime entre la línea 5 y 10. \\
\verb|sed '20,35d' file.txt| & Imprime todo excepto líneas 20 a 35. \\
\verb|sed 's/version/story/g' file.txt| & Reemplaza “version” por “story”. \\
\verb|sed 's/  */ /g'| & Reemplaza múltiples espacios por uno solo. \\
\verb|sed '30,40 s/version/story/g' file.txt| & Reemplaza sólo entre líneas 30 y 40. \\
\verb|sed '$d' file.txt| & Borra la última línea del archivo. \\
\verb|sed 'nd' file.txt| & Borra la n-ésima línea del archivo.
\end{tabular}
\end{table}
\end{frame}

%------------------------------------------------
% EJEMPLO / ACTIVIDAD ADICIONAL
%------------------------------------------------

\begin{frame}{Actividad Sugerida: Exploración BASH}
\begin{itemize}
    \item \textbf{Objetivo:} Mejorar la fluidez con los comandos básicos de bash.
    \item \textbf{Instrucciones:}
    \begin{enumerate}
        \item Crea un directorio llamado \texttt{mi-semana2} y dentro de él tres subdirectorios: \texttt{datos}, \texttt{scripts}, \texttt{resultados}.
        \item Copia o genera algunos archivos de texto simples en \texttt{datos}.
        \item Usa \texttt{awk} para filtrar líneas que contengan la palabra \texttt{hola} y redirige la salida a \texttt{resultados/filtrado.txt}.
        \item Edita el archivo final con \texttt{sed} para reemplazar la palabra \texttt{hola} por \texttt{HELLO}.
    \end{enumerate}
    \item \textbf{Conclusión:} Practicar tuberías y conceptos de filtrado (\texttt{awk} y \texttt{sed}).
\end{itemize}
\end{frame}

\begin{frame}
\Huge{\centerline{Fin de la Parte 1}}
\end{frame}

\end{document}

