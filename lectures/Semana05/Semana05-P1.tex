\documentclass[10pt]{beamer}

% ------------------------------------------------------------------------
% Carga del preámbulo personalizado (preamble.tex)
% Asegúrate de tenerlo en la misma carpeta para que \input funcione.
% ------------------------------------------------------------------------
\usetheme[progressbar=frametitle]{metropolis}
\usepackage{appendixnumberbeamer}
\usepackage{fancyvrb}
\usepackage{booktabs}
\usepackage[scale=2]{ccicons}
\usepackage{pgfplots}
\usepgfplotslibrary{dateplot}
\usepackage{type1cm}
\usepackage{lettrine}
\usepackage{ragged2e}
\usepackage{xspace}
\newcommand{\themename}{\textbf{\textsc{metropolis}}\xspace}
\usepackage{graphicx} % Allows including images
\usepackage{booktabs} % Allows the use of \toprule, \midrule and \bottomrule in tables
\usepackage[utf8]{inputenc} %solucion del problema de los acentos.
\usepackage{xcolor}
\usepackage{verbatim}

\definecolor{LightGray}{gray}{0.9}
% Paleta de azules estilo Python/matplotlib
\definecolor{PythonBlue}{RGB}{31, 119, 180}      % Azul principal matplotlib
\definecolor{LightPythonBlue}{RGB}{174, 199, 232} % Azul claro para fondos
\definecolor{DarkPythonBlue}{RGB}{23, 90, 135}    % Azul oscuro para líneas

% Colores para tema oscuro tipo VS Code/Colab
\definecolor{DarkBackground}{RGB}{40, 44, 52}     % Fondo oscuro similar a VS Code
\definecolor{CodeText}{RGB}{171, 178, 191}        % Texto gris claro para código
\definecolor{DarkFrame}{RGB}{60, 63, 65}          % Color del marco/borde

% Colores para tema claro
\definecolor{LightBackground}{RGB}{248, 248, 242} % Fondo claro tipo GitHub
\definecolor{LightCodeText}{RGB}{51, 51, 51}      % Texto oscuro para tema claro
\definecolor{LightFrame}{RGB}{220, 220, 220}      % Marco gris claro

\usepackage{minted}

%%%%%%%%%%%%%%%%%%%%%%%%%%%%%%%%%%%%%%%%%%%%%%%%%%%%%%%%%%%%%%%%%%%%%%%%%%%%%%%%%%%%%%
% CONFIGURACIÓN DE TEMAS - CAMBIA AQUÍ PARA ALTERNAR
% ========================================================================
% Descomenta UNA de las siguientes líneas para elegir el tema:

% TEMA OSCURO (VS Code style)
%\newcommand{\mytheme}{oscuro}
%\usemintedstyle{monokai}
%\newcommand{\mybgcolor}{DarkBackground}

% TEMA CLARO (GitHub style) - comenta las 3 líneas de arriba y descomenta estas:
 \newcommand{\mytheme}{claro}
 \usemintedstyle{default}
 \newcommand{\mybgcolor}{LightBackground}

% ========================================================================
\newcommand{\mypyfile}[1]{\inputminted[linenos=true, fontsize=\footnotesize, frame=lines, framesep=5\fboxrule,framerule=1pt]{python}{#1}}

% Comando personalizado para código Python inline
\newcommand{\pycode}[1]{\begin{minted}{python}#1\end{minted}}

% ========================================================================
% COMANDOS ESPECÍFICOS PARA CADA TEMA
% ========================================================================
% Comando para tema oscuro específico
\newcommand{\pyoscuro}[1]{%
\begin{minted}[
  bgcolor=DarkBackground,
  style=monokai,
  breaklines,
  frame=lines,
  framesep=2mm,
  baselinestretch=1.2,
  linenos,
  fontsize=\footnotesize
]{python}
#1
\end{minted}
}

% Comando para tema claro específico
\newcommand{\pyclaro}[1]{%
\begin{minted}[
  bgcolor=LightBackground,
  style=default,
  breaklines,
  frame=lines,
  framesep=2mm,
  baselinestretch=1.2,
  linenos,
  fontsize=\footnotesize
]{python}
#1
\end{minted}
}
%%%%%%%%%%%%%%%%%%%%%%%%%%%%%%%%%%%%%%%%%%%%%%%%%%%%%%%%%%%%%%%%%%%%%%%%%%%%%%%%%%%%%%



%%%%%%%%%%%%%%%%%%%%%%%%%%%%%%%%%%%%%%%%%%%%%%%%%%%%%%%%%%%%%%%%%%%%%%%%%%%%%%%%%%%%%%
% Configuración global para todos los bloques minted de Python
% Usa automáticamente el tema seleccionado arriba
\setminted[python]{
  breaklines,
  frame=lines,
  framesep=2mm,
  baselinestretch=1.2,
  bgcolor=\mybgcolor,
  linenos, 
  fontsize=\footnotesize
} % Configuración dinámica según tema elegido
%%%%%%%%%%%%%%%%%%%%%%%%%%%%%%%%%%%%%%%%%%%%%%%%%%%%%%%%%%%%%%%%%%%%%%%%%%%%%%%%%%%%%%



%%%%%%%%%%%%%%%%%%%%%%%%%%%%%%%%%%%%%%%%%%%%%%%%%%%%%%%%%%%%%%%%%%%%%%%%%%%%%%%%%%%%%%
% Configuración de colores del tema con paleta azul Python
\setbeamercolor{progress bar}{fg=DarkPythonBlue,bg=LightPythonBlue}
\setbeamercolor{title separator}{fg=DarkPythonBlue,bg=white!50!black}
\setbeamercolor{frametitle}{fg=white,bg=PythonBlue}
\title[PCFI161]{Programaci\'on para F\'isica y Astronom\'ia}
\subtitle{Departamento de Física.}

\newcommand{\myfront}{
\author[PCFI161]{Corodinadora: C Loyola \\ Profesores C Femenías / F Bugini / D Basantes}
\institute[UNAB]{Universidad Andrés Bello \\ Departamento de Física y Astronomía}
\date{Primer Semestre 2025}
}

\titlegraphic{
  \includegraphics[width=.08\textwidth]{1) Logo/logo-tux.png}\hfill
  \includegraphics[width=.3\textwidth]{1) Logo/logo-unab.png}\hfill
  \includegraphics[width=.08\textwidth]{1) Logo/logo-python.png}
}

\makeatletter
\setbeamertemplate{title page}{
  \begin{minipage}[b][\paperheight]{\textwidth}
    \vfill%
    \ifx\inserttitle\@empty\else\usebeamertemplate*{title}\fi
    \ifx\insertsubtitle\@empty\else\usebeamertemplate*{subtitle}\fi
    \usebeamertemplate*{title separator}
    \ifx\beamer@shortauthor\@empty\else\usebeamertemplate*{author}\fi
    \ifx\insertdate\@empty\else\usebeamertemplate*{date}\fi
    \ifx\insertinstitute\@empty\else\usebeamertemplate*{institute}\fi
    \vfill
    \ifx\inserttitlegraphic\@empty\else\inserttitlegraphic\fi
    \vspace*{1cm}
  \end{minipage}
}
\makeatother

% Configuración personalizada del frametitle para incluir la sección
\makeatletter
\setbeamertemplate{frametitle}{%
  \nointerlineskip%
  \begin{beamercolorbox}[%
      wd=\paperwidth,%
      sep=0pt,%
      leftskip=\@ifundefined{metropolis@frametitle@padding}{12pt}{\metropolis@frametitle@padding},%
      rightskip=\@ifundefined{metropolis@frametitle@padding}{12pt}{\metropolis@frametitle@padding},%
    ]{frametitle}%
  \@ifundefined{metropolis@frametitlestrut@start}{}{\metropolis@frametitlestrut@start}%
  {\textcolor{LightPythonBlue}{\insertsectionhead}\ifx\insertsectionhead\@empty\else\ $\ni$  \fi}\insertframetitle%
  \nolinebreak%
  \@ifundefined{metropolis@frametitlestrut@end}{}{\metropolis@frametitlestrut@end}%
  \end{beamercolorbox}%
}
\makeatother

\makeatletter
% Configuración segura de longitudes del tema Metropolis
\@ifundefined{metropolis@titleseparator@linewidth}{}{\setlength{\metropolis@titleseparator@linewidth}{2pt}}
\@ifundefined{metropolis@progressonsectionpage@linewidth}{}{\setlength{\metropolis@progressonsectionpage@linewidth}{2pt}}
\@ifundefined{metropolis@progressinheadfoot@linewidth}{}{\setlength{\metropolis@progressinheadfoot@linewidth}{2pt}}
\makeatother



\begin{document}

% ------------------------------------------------------------------------
% Portada personalizada (puedes usar \myfront si está definido en tu preamble.tex)
% ------------------------------------------------------------------------
\myfront{}

% ------------------------------------------------------------------------
% Slide 1: Título de la Sesión
% ------------------------------------------------------------------------
\begin{frame}
  \titlepage
  % Por ejemplo:
  % \title{Semana 5 - Sesión 1 (Sesión 9): Repaso Integral y Ejercicios Previos a Solemne I}
\end{frame}

% ------------------------------------------------------------------------
% Slide 2: Índice / Tabla de Contenidos
% ------------------------------------------------------------------------
\begin{frame}
  \frametitle{Resumen - Semana 5, Sesión 1 (Sesión 9)}
  \tableofcontents
\end{frame}

% ------------------------------------------------------------------------
% Configuración de bloques
% ------------------------------------------------------------------------
\metroset{block=fill}

% ----------------------------------------------------------------------------------------
% SECCIÓN 1: Introducción y Contexto
% ----------------------------------------------------------------------------------------
\section{Introducción y Contexto}

% ------------------------------------------------------------------------
% Slide 3: Objetivos de esta Sesión Especial
% ------------------------------------------------------------------------
\begin{frame}{Objetivos de la Sesión 9}
  \begin{itemize}
    \item \textbf{Repasar} de forma integral los contenidos vistos en las semanas anteriores.
    \item \textbf{Resolver} ejercicios y dudas previas a la evaluación \textbf{Solemne I}.
    \item \textbf{Fortalecer} la comprensión de sintaxis, estructuras de control, funciones y módulos.
    \item \textbf{Identificar} áreas con más dificultades y reforzarlas antes del examen parcial.
  \end{itemize}
\end{frame}

% ------------------------------------------------------------------------
% Slide 4: Conexión con las Sesiones Previas
% ------------------------------------------------------------------------
\begin{frame}{Recapitulación de Semanas 1-4}
  \begin{itemize}
    \item \textbf{Semana 1}: Configuración de Google Colab, operaciones básicas, \texttt{input/print}, variables.
    \item \textbf{Semana 2}: Sintaxis, aritmética, estructuras de control iniciales (\texttt{if, while}).
    \item \textbf{Semana 3}: Profundización de \texttt{while}, introducción al \texttt{for}, ejercicios colaborativos.
    \item \textbf{Semana 4}: \textbf{Funciones}, \textbf{módulos} y uso de \textbf{librerías externas} (\texttt{pip}, \texttt{numpy}, etc.).
  \end{itemize}
\end{frame}

% ----------------------------------------------------------------------------------------
% SECCIÓN 2: Resumen de Contenidos Clave
% ----------------------------------------------------------------------------------------
\section{Resumen de Contenidos Clave}

% ------------------------------------------------------------------------
% Slide 5: Sintaxis y Tipos de Datos
% ------------------------------------------------------------------------
\begin{frame}{Sintaxis General y Tipos de Datos}
  \begin{itemize}
    \item \textbf{Indentación}: bloques de código (\texttt{if}, \texttt{while}, \texttt{for}, \texttt{def}).
    \item \textbf{Tipos básicos}: \texttt{int}, \texttt{float}, \texttt{str}, \texttt{bool}, \texttt{complex}.
    \item \textbf{Operaciones}: \texttt{+ - * / // \% **}, prioridad de operadores, paréntesis.
    \item \textbf{Manejo de E/S}: \texttt{input()}, \texttt{print()}, conversión de tipos (\texttt{int()}, \texttt{float()}).
  \end{itemize}
\end{frame}

% ------------------------------------------------------------------------
% Slide 6: Estructuras de Control (if, elif, else)
% ------------------------------------------------------------------------
\begin{frame}{Condicionales \texttt{if/elif/else}}
  \begin{itemize}
    \item \textbf{if condicion:} bloque si \texttt{True}.
    \item \textbf{elif condicion\_2:} bloque si \texttt{condicion\_2} es \texttt{True}.
    \item \textbf{else:} bloque final si ninguna condición anterior se cumple.
    \item \textbf{Ejemplos}:
\begin{itemize}
  \item Menús interactivos.
  \item Validaciones de rango (p.e. notas entre 1.0 y 7.0).
\end{itemize}
  \end{itemize}
\end{frame}

% ------------------------------------------------------------------------
% Slide 7: Estructuras de Control (while, for)
% ------------------------------------------------------------------------
\begin{frame}{Bucles \texttt{while} y \texttt{for}}
  \begin{itemize}
    \item \textbf{while condicion:} repite bloque mientras la condición sea \texttt{True}.
      \begin{itemize}
        \item \texttt{break} para salir, \texttt{continue} para saltar a la siguiente iteración.
      \end{itemize}
    \item \textbf{for} sobre secuencias (\texttt{for i in range(...):}), ideal cuando conocemos la cantidad de iteraciones.
    \item \textbf{Aplicaciones}:
      \begin{itemize}
        \item Lectura indefinida de datos.
        \item Iteración sobre listas y rangos.
      \end{itemize}
  \end{itemize}
\end{frame}

% ------------------------------------------------------------------------
% Slide 8: Funciones
% ------------------------------------------------------------------------
\begin{frame}{Funciones}
  \begin{itemize}
    \item \textbf{def nombre(parámetros):} define una nueva función.
    \item \textbf{return} para devolver un valor (opcional).
    \item \textbf{Parámetros por defecto} (\texttt{def f(x=10): ...}) y \textbf{keyword arguments}.
    \item \textbf{Ventajas}: modularidad, reuso, claridad de código.
    \item \textbf{Scope}: variables locales dentro de la función.
  \end{itemize}
\end{frame}

% ------------------------------------------------------------------------
% Slide 9: Módulos y Paquetes
% ------------------------------------------------------------------------
\begin{frame}{Módulos y Paquetes}
  \begin{itemize}
    \item \textbf{Módulo =} archivo \texttt{.py} con funciones, clases, variables reutilizables.
    \item \textbf{Paquete =} carpeta con \texttt{\_\_init\_\_.py} y varios módulos.
    \item Importación mediante \texttt{import modulo} o \texttt{from modulo import func}.
    \item \textbf{Ejemplos}:
      \begin{itemize}
        \item \texttt{import math}, \texttt{import numpy as np}, \texttt{import mi\_modulo}.
      \end{itemize}
  \end{itemize}
\end{frame}

% ------------------------------------------------------------------------
% Slide 10: Librerías Externas
% ------------------------------------------------------------------------
\begin{frame}{Librerías Externas}
  \begin{itemize}
    \item Instalación con \texttt{pip install <paquete>} o \texttt{!pip install <paquete>} en Colab.
    \item Ejemplo: \texttt{numpy} para cálculo numérico, \texttt{matplotlib} para gráficas.
    \item Importancia de la \textbf{colaboración open-source} y la documentación.
  \end{itemize}
\end{frame}

% ----------------------------------------------------------------------------------------
% SECCIÓN 3: Ejercicios de Repaso
% ----------------------------------------------------------------------------------------
\section{Ejercicios de Repaso}

% ------------------------------------------------------------------------
% Slide 11: Ejercicio 1 - Fundamentos
% ------------------------------------------------------------------------
\begin{frame}{Ejercicio 1: Repaso de Fundamentos}
  \begin{block}{Enunciado}
    \begin{itemize}
      \item Pide un número entero \texttt{n}.
      \item Calcula su factorial (\texttt{n!}) de dos formas:
        \begin{enumerate}
          \item Con un \textbf{for}.
          \item Mediante una \textbf{función recursiva}.
        \end{enumerate}
      \item Muestra ambos resultados y valida que coincidan.
    \end{itemize}
  \end{block}
  \textbf{Objetivo}: Reforzar sintaxis de bucles y definición de funciones.
\end{frame}

% ------------------------------------------------------------------------
% Slide 12: Ejercicio 2 - Estructuras de Control
% ------------------------------------------------------------------------
\begin{frame}{Ejercicio 2: Análisis de Notas con \texttt{while}}
  \begin{block}{Enunciado}
    \begin{itemize}
      \item Solicitar notas en un \texttt{while} hasta que el usuario ingrese \texttt{-1}.
      \item Validar que cada nota esté en el rango [1.0, 7.0].
      \item Llevar conteo del número de notas válidas, suma total y promedio.
      \item Al final, imprimir el promedio o un mensaje si no hay datos válidos.
    \end{itemize}
  \end{block}
  \textbf{Objetivo}: Revisar \textbf{while}, validaciones, conteo y promedio.
\end{frame}

% ------------------------------------------------------------------------
% Slide 13: Ejercicio 3 - Módulos y Funciones
% ------------------------------------------------------------------------
\begin{frame}{Ejercicio 3: Módulo de Conversión de Unidades}
  \begin{block}{Enunciado}
    \begin{itemize}
      \item Crea un archivo \texttt{conversor.py} con varias funciones:
        \begin{itemize}
          \item \texttt{cm\_a\_m}, \texttt{m\_a\_km}, \texttt{km\_a\_cm}, etc.
        \end{itemize}
      \item Importa \texttt{conversor} en \texttt{main.py} y pide al usuario un valor y la conversión deseada (ej. cm \(\rightarrow\) km).
      \item Muestra el resultado final.
    \end{itemize}
  \end{block}
  \textbf{Objetivo}: Practicar la creación de \textbf{módulos}, importación y lógica simple de funciones.
\end{frame}

% ------------------------------------------------------------------------
% Slide 14: Ejercicio 4 - Biblioteca Externa
% ------------------------------------------------------------------------
\begin{frame}{Ejercicio 4: Usando \texttt{numpy}}
  \begin{block}{Enunciado}
    \begin{itemize}
      \item Instala e importa \texttt{numpy}.
      \item Genera un \textbf{arreglo} de 10 números aleatorios (\texttt{np.random.rand(10)}).
      \item Calcula su \textbf{media} y \textbf{desviación estándar}.
      \item \textbf{Opcional}: filtrar solo valores > 0.5 y mostrarlos.
    \end{itemize}
  \end{block}
  \textbf{Objetivo}: Reforzar \textbf{numpy}, acceso a funciones \texttt{mean}, \texttt{std}, slicing y condicionales.
\end{frame}

% ------------------------------------------------------------------------
% Slide 15: Actividad Colaborativa
% ------------------------------------------------------------------------
\begin{frame}{Actividad Colaborativa}
  \begin{itemize}
    \item Trabaja en \textbf{parejas} o \textbf{grupos de 3}.
    \item Selecciona 2 ejercicios (o más) y discútelos en conjunto.
    \item Anota cualquier duda o error que surja durante la implementación.
    \item Comparte tus soluciones y reflexiona sobre los puntos más complicados.
  \end{itemize}
\end{frame}

% ------------------------------------------------------------------------
% Slide 16: Ayuda y Orientación
% ------------------------------------------------------------------------
\begin{frame}{Ayuda y Orientación}
  \begin{itemize}
    \item Recuerda el uso de \textbf{docstrings} en funciones para clarificar su propósito.
    \item Maneja \textbf{ValueError} cuando conviertes \texttt{input} a \texttt{int} o \texttt{float}.
    \item Asegúrate de importar módulos correctamente y de guardar los archivos en la misma carpeta (o configurar la ruta).
    \item Si usas \texttt{numpy} en Colab, revisa si necesitas \textbf{!pip install numpy} o si ya está incluido (por defecto Colab lo incluye).
  \end{itemize}
\end{frame}

% ----------------------------------------------------------------------------------------
% SECCIÓN 4: Discusión y Consolidación
% ----------------------------------------------------------------------------------------
\section{Discusión y Consolidación}

% ------------------------------------------------------------------------
% Slide 17: Puesta en Común
% ------------------------------------------------------------------------
\begin{frame}{Puesta en Común}
  \begin{itemize}
    \item ¿Qué ejercicios causaron más dificultad?
    \item ¿Qué estrategias de resolución fueron más efectivas?
    \item ¿Dudas persistentes sobre \textbf{tipos}, \textbf{bucles} o \textbf{importaciones}?
  \end{itemize}
  \vspace{0.3cm}
  \textbf{Comparte con la clase para beneficio de todos.}
\end{frame}

% ------------------------------------------------------------------------
% Slide 18: Revisión de Dudas Frecuentes
% ------------------------------------------------------------------------
\begin{frame}{Dudas Frecuentes Identificadas}
  \begin{itemize}
    \item \textbf{Scope de variables} dentro de funciones.
    \item Mezclar \texttt{if} y \texttt{while} en un mismo flujo (ej.: menús interactivos).
    \item \textbf{Errores de importación} con rutas inadecuadas.
    \item \textbf{Excepción} al convertir datos de entrada que no son numéricos.
  \end{itemize}
\end{frame}

% ------------------------------------------------------------------------
% Slide 19: Buenas Prácticas para el Examen
% ------------------------------------------------------------------------
\begin{frame}{Buenas Prácticas para el Solemne}
  \begin{itemize}
    \item \textbf{Lee bien el enunciado} y comprende la tarea antes de programar.
    \item \textbf{Pseudocódigo breve}: planifica el flujo de control (if, while, for).
    \item \textbf{Funciones claras}: si el problema lo amerita, divide la lógica en funciones.
    \item \textbf{Pruebas con casos simples}: revisa siempre el comportamiento con inputs distintos.
    \item \textbf{Comentarios y docstrings}: facilitan la compresión de tu solución (y parcial o total puntaje).
  \end{itemize}
\end{frame}

% ------------------------------------------------------------------------
% Slide 20: Ejemplo de Pregunta Tipo Solemne
% ------------------------------------------------------------------------
\begin{frame}[fragile]{Ejemplo de Pregunta: \textit{Función y Validación}}
  \begin{block}{Enunciado}
    \begin{itemize}
      \item Define una función \texttt{leer\_entero\_positivo(msg)} que:
        \begin{itemize}
          \item Muestra el mensaje \texttt{msg}.
          \item Lee un entero desde \texttt{input()}.
          \item Valida que sea \textbf{positivo}.
          \item Si no lo es, vuelve a pedir el número hasta que sea válido.
          \item Retorna el valor correcto.
        \end{itemize}
      \item Usa esta función para leer \texttt{n} y luego imprimir la suma de 1 a \texttt{n}.
    \end{itemize}
  \end{block}
\end{frame}

% ----------------------------------------------------------------------------------------
% SECCIÓN 5: Consejos Finales y Cierre
% ----------------------------------------------------------------------------------------
\section{Consejos y Cierre}

% ------------------------------------------------------------------------
% Slide 21: Consejos de Estudio para la Solemne
% ------------------------------------------------------------------------
\begin{frame}{Consejos de Estudio}
  \begin{itemize}
    \item \textbf{Revisar apuntes} y ejercicios de las clases pasadas.
    \item \textbf{Practicar} sintaxis de Python en Colab (ej.: if, while, for, funciones).
    \item \textbf{Hacer miniprogramas} que integren varios conceptos (entradas, salidas, validaciones).
    \item \textbf{Releer} la documentación o apuntes de funciones clave (\texttt{math}, \texttt{random}, \texttt{numpy}, etc.).
  \end{itemize}
\end{frame}

% ------------------------------------------------------------------------
% Slide 22: Ejemplo de Autoevaluación
% ------------------------------------------------------------------------
\begin{frame}{Autoevaluación: Preguntas Guía}
  \begin{itemize}
    \item ¿Puedo leer datos de usuario y convertirlos correctamente?
    \item ¿Sé escribir un \textbf{bucle while} que termina en la condición correcta?
    \item ¿Entiendo cómo \texttt{if/elif/else} deciden el flujo?
    \item ¿Me siento cómodo definiendo funciones con \texttt{def} y \texttt{return}?
    \item ¿Podría crear un módulo \texttt{mimodulo.py} y usarlo en un script principal?
  \end{itemize}
\end{frame}

% ------------------------------------------------------------------------
% Slide 23: Recursos Adicionales
% ------------------------------------------------------------------------
\begin{frame}{Recursos Adicionales}
  \begin{itemize}
    \item \href{https://docs.python.org/3/tutorial/}{\textbf{Python Tutorial Oficial}} (repaso secciones 1-6).
    \item \href{https://www.w3schools.com/python/}{\textbf{W3Schools Python}} (ejemplos básicos).
    \item \textbf{Stack Overflow} (buscar soluciones a errores comunes).
    \item Videos y guías en YouTube: \texttt{“Python for Beginners”}, \texttt{“Programación en Python”}.
  \end{itemize}
\end{frame}

% ------------------------------------------------------------------------
% Slide 24: Próxima Sesión (Solemne I)
% ------------------------------------------------------------------------
\begin{frame}{Próxima Sesión: Solemne I}
  \begin{itemize}
    \item \textbf{No hay contenido nuevo}, sino la \textbf{evaluación} de los temas ya vistos:
      \begin{itemize}
        \item Unidades I y II (Syllabus).
        \item Sintaxis básica, control de flujo, funciones y primeros usos de módulos.
      \end{itemize}
    \item \textbf{Formato de la Solemne}: problemas cortos y medianos que requieren programar en Python y quizá un par de preguntas conceptuales.
  \end{itemize}
\end{frame}

% ------------------------------------------------------------------------
% Slide 25: Cierre de la Sesión
% ------------------------------------------------------------------------
\begin{frame}
  \Huge{\centerline{¡Mucho éxito en la Solemne!}}
  \vspace{0.4cm}
  \normalsize
  \begin{itemize}
    \item Aprovechen de \textbf{revisar ejercicios}.
    \item No duden en \textbf{consultar} a través de foros o con sus compañeros.
    \item ¡Nos vemos en la siguiente sesión (Solemne I)!
  \end{itemize}
\end{frame}

\end{document}

