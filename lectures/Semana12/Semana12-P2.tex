\documentclass[10pt]{beamer}

% ------------------------------------------------------------------------
% Carga de tu preámbulo personalizado (preamble.tex).
% Asegúrate de tenerlo en la misma carpeta para que \input funcione.
% ------------------------------------------------------------------------
\usetheme[progressbar=frametitle]{metropolis}
\usepackage{appendixnumberbeamer}
\usepackage{fancyvrb}
\usepackage{booktabs}
\usepackage[scale=2]{ccicons}
\usepackage{pgfplots}
\usepgfplotslibrary{dateplot}
\usepackage{type1cm}
\usepackage{lettrine}
\usepackage{ragged2e}
\usepackage{xspace}
\newcommand{\themename}{\textbf{\textsc{metropolis}}\xspace}
\usepackage{graphicx} % Allows including images
\usepackage{booktabs} % Allows the use of \toprule, \midrule and \bottomrule in tables
\usepackage[utf8]{inputenc} %solucion del problema de los acentos.
\usepackage{xcolor}
\usepackage{verbatim}

\definecolor{LightGray}{gray}{0.9}
% Paleta de azules estilo Python/matplotlib
\definecolor{PythonBlue}{RGB}{31, 119, 180}      % Azul principal matplotlib
\definecolor{LightPythonBlue}{RGB}{174, 199, 232} % Azul claro para fondos
\definecolor{DarkPythonBlue}{RGB}{23, 90, 135}    % Azul oscuro para líneas

% Colores para tema oscuro tipo VS Code/Colab
\definecolor{DarkBackground}{RGB}{40, 44, 52}     % Fondo oscuro similar a VS Code
\definecolor{CodeText}{RGB}{171, 178, 191}        % Texto gris claro para código
\definecolor{DarkFrame}{RGB}{60, 63, 65}          % Color del marco/borde

% Colores para tema claro
\definecolor{LightBackground}{RGB}{248, 248, 242} % Fondo claro tipo GitHub
\definecolor{LightCodeText}{RGB}{51, 51, 51}      % Texto oscuro para tema claro
\definecolor{LightFrame}{RGB}{220, 220, 220}      % Marco gris claro

\usepackage{minted}

%%%%%%%%%%%%%%%%%%%%%%%%%%%%%%%%%%%%%%%%%%%%%%%%%%%%%%%%%%%%%%%%%%%%%%%%%%%%%%%%%%%%%%
% CONFIGURACIÓN DE TEMAS - CAMBIA AQUÍ PARA ALTERNAR
% ========================================================================
% Descomenta UNA de las siguientes líneas para elegir el tema:

% TEMA OSCURO (VS Code style)
%\newcommand{\mytheme}{oscuro}
%\usemintedstyle{monokai}
%\newcommand{\mybgcolor}{DarkBackground}

% TEMA CLARO (GitHub style) - comenta las 3 líneas de arriba y descomenta estas:
 \newcommand{\mytheme}{claro}
 \usemintedstyle{default}
 \newcommand{\mybgcolor}{LightBackground}

% ========================================================================
\newcommand{\mypyfile}[1]{\inputminted[linenos=true, fontsize=\footnotesize, frame=lines, framesep=5\fboxrule,framerule=1pt]{python}{#1}}

% Comando personalizado para código Python inline
\newcommand{\pycode}[1]{\begin{minted}{python}#1\end{minted}}

% ========================================================================
% COMANDOS ESPECÍFICOS PARA CADA TEMA
% ========================================================================
% Comando para tema oscuro específico
\newcommand{\pyoscuro}[1]{%
\begin{minted}[
  bgcolor=DarkBackground,
  style=monokai,
  breaklines,
  frame=lines,
  framesep=2mm,
  baselinestretch=1.2,
  linenos,
  fontsize=\footnotesize
]{python}
#1
\end{minted}
}

% Comando para tema claro específico
\newcommand{\pyclaro}[1]{%
\begin{minted}[
  bgcolor=LightBackground,
  style=default,
  breaklines,
  frame=lines,
  framesep=2mm,
  baselinestretch=1.2,
  linenos,
  fontsize=\footnotesize
]{python}
#1
\end{minted}
}
%%%%%%%%%%%%%%%%%%%%%%%%%%%%%%%%%%%%%%%%%%%%%%%%%%%%%%%%%%%%%%%%%%%%%%%%%%%%%%%%%%%%%%



%%%%%%%%%%%%%%%%%%%%%%%%%%%%%%%%%%%%%%%%%%%%%%%%%%%%%%%%%%%%%%%%%%%%%%%%%%%%%%%%%%%%%%
% Configuración global para todos los bloques minted de Python
% Usa automáticamente el tema seleccionado arriba
\setminted[python]{
  breaklines,
  frame=lines,
  framesep=2mm,
  baselinestretch=1.2,
  bgcolor=\mybgcolor,
  linenos, 
  fontsize=\footnotesize
} % Configuración dinámica según tema elegido
%%%%%%%%%%%%%%%%%%%%%%%%%%%%%%%%%%%%%%%%%%%%%%%%%%%%%%%%%%%%%%%%%%%%%%%%%%%%%%%%%%%%%%



%%%%%%%%%%%%%%%%%%%%%%%%%%%%%%%%%%%%%%%%%%%%%%%%%%%%%%%%%%%%%%%%%%%%%%%%%%%%%%%%%%%%%%
% Configuración de colores del tema con paleta azul Python
\setbeamercolor{progress bar}{fg=DarkPythonBlue,bg=LightPythonBlue}
\setbeamercolor{title separator}{fg=DarkPythonBlue,bg=white!50!black}
\setbeamercolor{frametitle}{fg=white,bg=PythonBlue}
\title[PCFI161]{Programaci\'on para F\'isica y Astronom\'ia}
\subtitle{Departamento de Física.}

\newcommand{\myfront}{
\author[PCFI161]{Corodinadora: C Loyola \\ Profesores C Femenías / F Bugini / D Basantes}
\institute[UNAB]{Universidad Andrés Bello \\ Departamento de Física y Astronomía}
\date{Primer Semestre 2025}
}

\titlegraphic{
  \includegraphics[width=.08\textwidth]{1) Logo/logo-tux.png}\hfill
  \includegraphics[width=.3\textwidth]{1) Logo/logo-unab.png}\hfill
  \includegraphics[width=.08\textwidth]{1) Logo/logo-python.png}
}

\makeatletter
\setbeamertemplate{title page}{
  \begin{minipage}[b][\paperheight]{\textwidth}
    \vfill%
    \ifx\inserttitle\@empty\else\usebeamertemplate*{title}\fi
    \ifx\insertsubtitle\@empty\else\usebeamertemplate*{subtitle}\fi
    \usebeamertemplate*{title separator}
    \ifx\beamer@shortauthor\@empty\else\usebeamertemplate*{author}\fi
    \ifx\insertdate\@empty\else\usebeamertemplate*{date}\fi
    \ifx\insertinstitute\@empty\else\usebeamertemplate*{institute}\fi
    \vfill
    \ifx\inserttitlegraphic\@empty\else\inserttitlegraphic\fi
    \vspace*{1cm}
  \end{minipage}
}
\makeatother

% Configuración personalizada del frametitle para incluir la sección
\makeatletter
\setbeamertemplate{frametitle}{%
  \nointerlineskip%
  \begin{beamercolorbox}[%
      wd=\paperwidth,%
      sep=0pt,%
      leftskip=\@ifundefined{metropolis@frametitle@padding}{12pt}{\metropolis@frametitle@padding},%
      rightskip=\@ifundefined{metropolis@frametitle@padding}{12pt}{\metropolis@frametitle@padding},%
    ]{frametitle}%
  \@ifundefined{metropolis@frametitlestrut@start}{}{\metropolis@frametitlestrut@start}%
  {\textcolor{LightPythonBlue}{\insertsectionhead}\ifx\insertsectionhead\@empty\else\ $\ni$  \fi}\insertframetitle%
  \nolinebreak%
  \@ifundefined{metropolis@frametitlestrut@end}{}{\metropolis@frametitlestrut@end}%
  \end{beamercolorbox}%
}
\makeatother

\makeatletter
% Configuración segura de longitudes del tema Metropolis
\@ifundefined{metropolis@titleseparator@linewidth}{}{\setlength{\metropolis@titleseparator@linewidth}{2pt}}
\@ifundefined{metropolis@progressonsectionpage@linewidth}{}{\setlength{\metropolis@progressonsectionpage@linewidth}{2pt}}
\@ifundefined{metropolis@progressinheadfoot@linewidth}{}{\setlength{\metropolis@progressinheadfoot@linewidth}{2pt}}
\makeatother



\begin{document}

% ------------------------------------------------------------------------
% Portada de la Presentación
% ------------------------------------------------------------------------
\myfront{}

% ------------------------------------------------------------------------
% Slide 1: Título de la Sesión
% ------------------------------------------------------------------------
\begin{frame}
  \titlepage
  % Por ejemplo:
  % \title{Semana 12 - Sesión 2 (Sesión 24): Evaluación en Clase (25-30 min) sobre Temas de la Semana 11}
\end{frame}

% ------------------------------------------------------------------------
% Slide 2: Índice / Tabla de Contenidos
% ------------------------------------------------------------------------
\begin{frame}
  \frametitle{Resumen - Semana 12, Sesión 2 (Sesión 24)}
  \tableofcontents
\end{frame}

% ------------------------------------------------------------------------
% Configuración de bloques
% ------------------------------------------------------------------------
\metroset{block=fill}

% ----------------------------------------------------------------------------------------
% SECCIÓN 1: Introducción y Repaso
% ----------------------------------------------------------------------------------------
\section{Introducción y Conexión}

% ------------------------------------------------------------------------
% Slide 3: Conexión con la Semana 11
% ------------------------------------------------------------------------
\begin{frame}{Repaso y Contexto}
  \begin{itemize}
    \item \textbf{Semana 11} profundizamos:
      \begin{itemize}
        \item \textbf{POO avanzada}: polimorfismo, composición (relación has-a), ejemplos en sistemas físicos/astronómicos.
        \item \textbf{Integración} con NumPy y Matplotlib para análisis y visualización.
        \item Retroalimentación post-Solemne II e ideas de proyectos.
      \end{itemize}
    \item \textbf{Semana 12, Sesión 1 (Sesión 23)}:
      \begin{itemize}
        \item Lanzamiento formal de proyectos integradores (requisitos, equipos, cronogramas).
        \item Introducción a \textbf{SciPy} o \textbf{Sympy} como herramientas complementarias.
      \end{itemize}
    \item \textbf{Objetivo de hoy (Sesión 24)}:
      \begin{itemize}
        \item Resolver un \textbf{Problema a Evaluar} (25-30 min) relacionado a lo visto en Semana 11 (POO avanzada).
        \item Subir la solución a \textbf{CANVAS}.
        \item Discutir brevemente las soluciones.
      \end{itemize}
  \end{itemize}
\end{frame}

% ------------------------------------------------------------------------
% Slide 4: Objetivos de la Sesión 24
% ------------------------------------------------------------------------
\begin{frame}{Objetivos de la Sesión 24}
  \begin{itemize}
    \item \textbf{Aplicar} los conceptos de polimorfismo y/o composición en un ejercicio evaluado.
    \item \textbf{Combinar} estas ideas con NumPy/Matplotlib (si corresponde) en un problema corto.
    \item \textbf{Reforzar} la práctica de trabajo en grupo y la entrega en CANVAS bajo un tiempo limitado (25-30 min).
    \item \textbf{Retroalimentar} los resultados y aclarar dudas finales.
  \end{itemize}
\end{frame}

% ----------------------------------------------------------------------------------------
% SECCIÓN 2: Problema a Evaluar (25-30 min)
% ----------------------------------------------------------------------------------------
\section{Problema a Evaluar}

% ------------------------------------------------------------------------
% Slide 5: Descripción del Problema
% ------------------------------------------------------------------------
\begin{frame}{Problema a Evaluar: Clases y Composición (Semana 11 Refuerzo)}
  \textbf{Contexto (Semana 11):}  
  \begin{itemize}
    \item Vimos relaciones \textbf{has-a} (composición), polimorfismo, y ejemplos como \texttt{SolarSystem} conteniendo \texttt{Body}, etc.
  \end{itemize}

  \textbf{Tareas}:
  \begin{enumerate}
    \item Definir una clase base \textbf{Item} con atributos \texttt{name, quantity} y un método \texttt{show\_info()}.
    \item Crear una clase \textbf{Inventory} que tiene una lista de \texttt{Item} (composición).
    \item En \texttt{Inventory}, incluir métodos:
      \begin{itemize}
        \item \texttt{add\_item(item)} para agregar un objeto \texttt{Item}.
        \item \texttt{total\_quantity()} que retorna la suma de \texttt{quantity} en la lista.
      \end{itemize}
    \item (Opcional) Graficar un \textbf{bar plot} con \(\texttt{name}\) vs. \(\texttt{quantity}\).
  \end{enumerate}
\end{frame}

% ------------------------------------------------------------------------
% Slide 6: Instrucciones y Formato de Entrega
% ------------------------------------------------------------------------
\begin{frame}{Instrucciones para la Evaluación}
  \begin{itemize}
    \item \textbf{Grupos} de 2-3 estudiantes.
    \item Crear un \textbf{notebook} (Colab) o script local llamado \texttt{Eval\_Semana12\_Apellidos.ipynb}.
    \item \textbf{Desarrollar}:
      \begin{enumerate}
        \item Clase base \texttt{Item} (\_\_init\_\_, \texttt{show\_info()}).
        \item Clase \texttt{Inventory} con su lista interna \(\texttt{self.items}\).
        \item Añadir 3-5 \texttt{Item}, mostrar info y calcular \texttt{total\_quantity()}.
        \item \textbf{Bonus}: \texttt{plt.bar()} con \(\texttt{tick\_label = [item.name for item in ...]}\).
      \end{enumerate}
    \item Subir el archivo a \textbf{CANVAS} en un plazo de \textbf{25-30 min}.
  \end{itemize}
\end{frame}

% ------------------------------------------------------------------------
% Slide 7: Pautas de Evaluación
% ------------------------------------------------------------------------
\begin{frame}{Pautas de Evaluación}
  \begin{itemize}
    \item \textbf{Funcionalidad} (40\%): Ejecución sin errores, correcto uso de clases, métodos, e instancias.
    \item \textbf{POO Avanzada} (20\%): Se evidencia \textbf{composición} (Inventory \emph{has-a} list of Items).  
    \item \textbf{Output/Visualización} (20\%): \texttt{show\_info()} útil, \textbf{bar plot} si lo hacen (opcional).
    \item \textbf{Estilo y Organización} (20\%): Comentarios, claridad de código, nombres de variables y métodos.
  \end{itemize}
  \vspace{0.3cm}
  \textbf{Nota}: El bonus de la gráfica no es obligatorio, pero suma puntos o mejora la presentación.
\end{frame}

% ------------------------------------------------------------------------
% Slide 8: Tiempo de Desarrollo (25-30 min)
% ------------------------------------------------------------------------
\begin{frame}{Tiempo de Desarrollo}
  \begin{block}{}
    \huge{\textbf{Tienen 25-30 minutos para resolver y subir a CANVAS.}}
  \end{block}
  \vspace{0.3cm}
  \textbf{Sugerencias}:
  \begin{itemize}
    \item Iniciar rápido con la clase base, luego la clase \texttt{Inventory}.
    \item Probar la \texttt{total\_quantity()} y \texttt{show\_info()} antes de añadir la gráfica.
    \item Revisar \textbf{import matplotlib.pyplot as plt} si hacen \textbf{bar plot}.
    \item Asegurarse de \textbf{ejecutar y ver} el output final antes de subir.
  \end{itemize}
\end{frame}

% ----------------------------------------------------------------------------------------
% SECCIÓN 3: Trabajo y Discusión
% ----------------------------------------------------------------------------------------
\section{Trabajo y Discusión}

% ------------------------------------------------------------------------
% Slide 9: Espacio de Resolución y Entrega
% ------------------------------------------------------------------------
\begin{frame}{Espacio de Resolución}
  \begin{itemize}
    \item \textbf{Trabajen en voz baja}, cada grupo crea su notebook o script.
    \item Preguntas puntuales a mí si bloquea totalmente la resolución.
    \item Subir a CANVAS antes de que finalice el tiempo (25-30 min).
  \end{itemize}
\end{frame}

% ------------------------------------------------------------------------
% Slide 10: Cierre de la Evaluación
% ------------------------------------------------------------------------
\begin{frame}{Entrega Final}
  \begin{block}{Subir a CANVAS}
    \begin{itemize}
      \item Un integrante sube el \textbf{notebook .ipynb} o \textbf{.py} dentro del tiempo.
      \item Revisen que todo el código y \textbf{resultados} (prints, gráficas) estén correctos.
      \item Añadan \textbf{comentarios} o Markdown para explicar brevemente su solución.
    \end{itemize}
  \end{block}
  \vspace{0.3cm}
  \textbf{Al final}, haremos una breve \textbf{discusión de soluciones} y errores comunes.
\end{frame}

% ------------------------------------------------------------------------
% Slide 11: Discusión de Resultados
% ------------------------------------------------------------------------
\begin{frame}{Discusión Posterior}
  \begin{itemize}
    \item ¿Algún problema al manejar la \textbf{lista de Items} en \texttt{Inventory}?
    \item ¿Cómo implementaron \texttt{show\_info()}? (Uso de \_\_str\_\_, etc.)
    \item ¿Quién logró la \textbf{bar plot}? ¿Qué parámetros usaron?
    \item ¿El tiempo (25-30 min) fue suficiente?
  \end{itemize}
\end{frame}

% ------------------------------------------------------------------------
% Slide 12: Resumen de la Evaluación
% ------------------------------------------------------------------------
\begin{frame}{Resumen de la Evaluación}
  \begin{itemize}
    \item Actividad integró \textbf{composición} en POO (Semana 11) y manejo básico de objetos.
    \item Opción de visualizar con \textbf{Matplotlib} en un bar chart.
    \item Se realizó en tiempo acotado, comprobando agilidad en la sintaxis de POO.
    \item Retroalimentación más detallada se dará tras la corrección.
  \end{itemize}
\end{frame}

% ----------------------------------------------------------------------------------------
% SECCIÓN 4: Cierre de la Sesión y Próximos Pasos
% ----------------------------------------------------------------------------------------
\section{Conclusiones y Próximos Pasos}

% ------------------------------------------------------------------------
% Slide 13: Conclusiones de la Sesión 24
% ------------------------------------------------------------------------
\begin{frame}{Conclusiones de la Sesión 24}
  \begin{itemize}
    \item Realizamos un \textbf{Problema a Evaluar} enfocado en \textbf{composición} en POO, vinculado a Semana 11.
    \item Pudimos (opcionalmente) reforzar la \textbf{visualización} con un diagrama de barras.
    \item Mantenemos la práctica de entregar resultados vía \textbf{CANVAS} en un tiempo limitado.
    \item Esto refuerza la \textbf{consolidación} de POO avanzada antes del proyecto final (si aplica).
  \end{itemize}
\end{frame}

% ------------------------------------------------------------------------
% Slide 14: Próxima Sesión (Semana 13)
% ------------------------------------------------------------------------
\begin{frame}{Próximos Temas}
  \begin{itemize}
    \item \textbf{Semana 13}: Seguimiento de proyectos (avances, revisiones parciales).
    \item Consultas de \textbf{POO}, \textbf{datos}, \textbf{visualización}, o integración con \texttt{scipy/sympy}.
    \item Si el tiempo lo permite, exploraremos \textbf{statistical modules} o \textbf{Machine Learning} superficial (depende de Syllabus).
  \end{itemize}
\end{frame}

% ------------------------------------------------------------------------
% Slide 15: Recursos Adicionales
% ------------------------------------------------------------------------
\begin{frame}{Recursos Adicionales}
  \begin{itemize}
    \item \href{https://docs.python.org/3/tutorial/classes.html}{\textbf{Python Tutorial - Classes}} (repasar POO).
    \item \href{https://matplotlib.org/stable/}{\textbf{Matplotlib Docs}} - para personalizar bar charts.
    \item \textbf{Canvas y foros} - para consultas e intercambio de soluciones.
  \end{itemize}
\end{frame}

% ------------------------------------------------------------------------
% Slide 16: Cierre de la Sesión
% ------------------------------------------------------------------------
\begin{frame}
  \Huge{\centerline{¡Excelente trabajo y hasta la próxima sesión!}}
  \vspace{0.4cm}
  \normalsize
  \begin{itemize}
    \item Revisen \textbf{Canvas} para feedback y notificaciones.
    \item ¡Nos vemos en la Semana 13 con más avances y consultas!
  \end{itemize}
\end{frame}

\end{document}

