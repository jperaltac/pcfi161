\documentclass[10pt]{beamer}

\usetheme[progressbar=frametitle]{metropolis}
\usepackage{appendixnumberbeamer}
\usepackage{fancyvrb}
\usepackage{booktabs}
\usepackage[scale=2]{ccicons}
\usepackage{pgfplots}
\usepgfplotslibrary{dateplot}
\usepackage{type1cm}
\usepackage{lettrine}
\usepackage{ragged2e}
\usepackage{xspace}
\newcommand{\themename}{\textbf{\textsc{metropolis}}\xspace}
\usepackage{graphicx} % Allows including images
\usepackage{booktabs} % Allows the use of \toprule, \midrule and \bottomrule in tables
\usepackage[utf8]{inputenc} %solucion del problema de los acentos.
\usepackage{xcolor}
\usepackage{verbatim}

\definecolor{LightGray}{gray}{0.9}
% Paleta de azules estilo Python/matplotlib
\definecolor{PythonBlue}{RGB}{31, 119, 180}      % Azul principal matplotlib
\definecolor{LightPythonBlue}{RGB}{174, 199, 232} % Azul claro para fondos
\definecolor{DarkPythonBlue}{RGB}{23, 90, 135}    % Azul oscuro para líneas

% Colores para tema oscuro tipo VS Code/Colab
\definecolor{DarkBackground}{RGB}{40, 44, 52}     % Fondo oscuro similar a VS Code
\definecolor{CodeText}{RGB}{171, 178, 191}        % Texto gris claro para código
\definecolor{DarkFrame}{RGB}{60, 63, 65}          % Color del marco/borde

% Colores para tema claro
\definecolor{LightBackground}{RGB}{248, 248, 242} % Fondo claro tipo GitHub
\definecolor{LightCodeText}{RGB}{51, 51, 51}      % Texto oscuro para tema claro
\definecolor{LightFrame}{RGB}{220, 220, 220}      % Marco gris claro

\usepackage{minted}

%%%%%%%%%%%%%%%%%%%%%%%%%%%%%%%%%%%%%%%%%%%%%%%%%%%%%%%%%%%%%%%%%%%%%%%%%%%%%%%%%%%%%%
% CONFIGURACIÓN DE TEMAS - CAMBIA AQUÍ PARA ALTERNAR
% ========================================================================
% Descomenta UNA de las siguientes líneas para elegir el tema:

% TEMA OSCURO (VS Code style)
%\newcommand{\mytheme}{oscuro}
%\usemintedstyle{monokai}
%\newcommand{\mybgcolor}{DarkBackground}

% TEMA CLARO (GitHub style) - comenta las 3 líneas de arriba y descomenta estas:
 \newcommand{\mytheme}{claro}
 \usemintedstyle{default}
 \newcommand{\mybgcolor}{LightBackground}

% ========================================================================
\newcommand{\mypyfile}[1]{\inputminted[linenos=true, fontsize=\footnotesize, frame=lines, framesep=5\fboxrule,framerule=1pt]{python}{#1}}

% Comando personalizado para código Python inline
\newcommand{\pycode}[1]{\begin{minted}{python}#1\end{minted}}

% ========================================================================
% COMANDOS ESPECÍFICOS PARA CADA TEMA
% ========================================================================
% Comando para tema oscuro específico
\newcommand{\pyoscuro}[1]{%
\begin{minted}[
  bgcolor=DarkBackground,
  style=monokai,
  breaklines,
  frame=lines,
  framesep=2mm,
  baselinestretch=1.2,
  linenos,
  fontsize=\footnotesize
]{python}
#1
\end{minted}
}

% Comando para tema claro específico
\newcommand{\pyclaro}[1]{%
\begin{minted}[
  bgcolor=LightBackground,
  style=default,
  breaklines,
  frame=lines,
  framesep=2mm,
  baselinestretch=1.2,
  linenos,
  fontsize=\footnotesize
]{python}
#1
\end{minted}
}
%%%%%%%%%%%%%%%%%%%%%%%%%%%%%%%%%%%%%%%%%%%%%%%%%%%%%%%%%%%%%%%%%%%%%%%%%%%%%%%%%%%%%%



%%%%%%%%%%%%%%%%%%%%%%%%%%%%%%%%%%%%%%%%%%%%%%%%%%%%%%%%%%%%%%%%%%%%%%%%%%%%%%%%%%%%%%
% Configuración global para todos los bloques minted de Python
% Usa automáticamente el tema seleccionado arriba
\setminted[python]{
  breaklines,
  frame=lines,
  framesep=2mm,
  baselinestretch=1.2,
  bgcolor=\mybgcolor,
  linenos, 
  fontsize=\footnotesize
} % Configuración dinámica según tema elegido
%%%%%%%%%%%%%%%%%%%%%%%%%%%%%%%%%%%%%%%%%%%%%%%%%%%%%%%%%%%%%%%%%%%%%%%%%%%%%%%%%%%%%%



%%%%%%%%%%%%%%%%%%%%%%%%%%%%%%%%%%%%%%%%%%%%%%%%%%%%%%%%%%%%%%%%%%%%%%%%%%%%%%%%%%%%%%
% Configuración de colores del tema con paleta azul Python
\setbeamercolor{progress bar}{fg=DarkPythonBlue,bg=LightPythonBlue}
\setbeamercolor{title separator}{fg=DarkPythonBlue,bg=white!50!black}
\setbeamercolor{frametitle}{fg=white,bg=PythonBlue}
\title[PCFI161]{Programaci\'on para F\'isica y Astronom\'ia}
\subtitle{Departamento de Física.}

\newcommand{\myfront}{
\author[PCFI161]{Corodinadora: C Loyola \\ Profesores C Femenías / F Bugini / D Basantes}
\institute[UNAB]{Universidad Andrés Bello \\ Departamento de Física y Astronomía}
\date{Primer Semestre 2025}
}

\titlegraphic{
  \includegraphics[width=.08\textwidth]{1) Logo/logo-tux.png}\hfill
  \includegraphics[width=.3\textwidth]{1) Logo/logo-unab.png}\hfill
  \includegraphics[width=.08\textwidth]{1) Logo/logo-python.png}
}

\makeatletter
\setbeamertemplate{title page}{
  \begin{minipage}[b][\paperheight]{\textwidth}
    \vfill%
    \ifx\inserttitle\@empty\else\usebeamertemplate*{title}\fi
    \ifx\insertsubtitle\@empty\else\usebeamertemplate*{subtitle}\fi
    \usebeamertemplate*{title separator}
    \ifx\beamer@shortauthor\@empty\else\usebeamertemplate*{author}\fi
    \ifx\insertdate\@empty\else\usebeamertemplate*{date}\fi
    \ifx\insertinstitute\@empty\else\usebeamertemplate*{institute}\fi
    \vfill
    \ifx\inserttitlegraphic\@empty\else\inserttitlegraphic\fi
    \vspace*{1cm}
  \end{minipage}
}
\makeatother

% Configuración personalizada del frametitle para incluir la sección
\makeatletter
\setbeamertemplate{frametitle}{%
  \nointerlineskip%
  \begin{beamercolorbox}[%
      wd=\paperwidth,%
      sep=0pt,%
      leftskip=\@ifundefined{metropolis@frametitle@padding}{12pt}{\metropolis@frametitle@padding},%
      rightskip=\@ifundefined{metropolis@frametitle@padding}{12pt}{\metropolis@frametitle@padding},%
    ]{frametitle}%
  \@ifundefined{metropolis@frametitlestrut@start}{}{\metropolis@frametitlestrut@start}%
  {\textcolor{LightPythonBlue}{\insertsectionhead}\ifx\insertsectionhead\@empty\else\ $\ni$  \fi}\insertframetitle%
  \nolinebreak%
  \@ifundefined{metropolis@frametitlestrut@end}{}{\metropolis@frametitlestrut@end}%
  \end{beamercolorbox}%
}
\makeatother

\makeatletter
% Configuración segura de longitudes del tema Metropolis
\@ifundefined{metropolis@titleseparator@linewidth}{}{\setlength{\metropolis@titleseparator@linewidth}{2pt}}
\@ifundefined{metropolis@progressonsectionpage@linewidth}{}{\setlength{\metropolis@progressonsectionpage@linewidth}{2pt}}
\@ifundefined{metropolis@progressinheadfoot@linewidth}{}{\setlength{\metropolis@progressinheadfoot@linewidth}{2pt}}
\makeatother



\title{Semana 12 – Sesión 2 (Sesión 24): Taller Evaluado y Lanzamiento de la Tarea V}
\author{PCFI161}
\date{21 mayo 2025}

\begin{document}

\myfront{}
\begin{frame}
  \titlepage
\end{frame}

\begin{frame}
  \frametitle{Resumen – Semana 12, Sesión 2 (Sesión 24)}
  \tableofcontents
\end{frame}

\metroset{block=fill}

% -----------------------------------------------------------------------------
\section{Introducción y Objetivos}
% -----------------------------------------------------------------------------
\begin{frame}{Objetivos del Taller}
  \begin{itemize}
    \item \textbf{Aplicar} estadística descriptiva sobre un dataset real.
    \item \textbf{Visualizar} patrones con histogramas, boxplots y violin plots.
    \item \textbf{Entregar} un notebook evaluado (\alert{Taller 12}) en 40 min.
    \item Presentar formalmente la \textbf{Tarea V} (trabajo domiciliario).
  \end{itemize}
\end{frame}

% -----------------------------------------------------------------------------
\section{Dataset del Taller}
% -----------------------------------------------------------------------------
\begin{frame}{Descripción del Dataset}
  \begin{itemize}
    \item Archivo: \texttt{stars\_brightness.csv}
    \item Columnas:
      \begin{enumerate}
        \item \texttt{magnitude\_app} – magnitud aparente.  
        \item \texttt{temperature\_K} – temperatura efectiva.
        \item \texttt{spectral\_class} – clase espectral (categórica).
      \end{enumerate}
    \item \emph{Tamaño}: 500 estrellas (fuente: catálogo HIPPARCOS reducido).
  \end{itemize}
\end{frame}

% -----------------------------------------------------------------------------
\section{Enunciado del Taller Evaluado}
% -----------------------------------------------------------------------------
\begin{frame}{Taller 12 – Instrucciones}
  \textbf{Tarea en grupos de 2–3 (40 min)}  
  \begin{enumerate}
    \item Leer el CSV con \texttt{pandas}.
    \item Calcular por columna: media, mediana, desvío estándar y rango intercuartil.
    \item Graficar:
      \begin{itemize}
        \item Histograma de \texttt{magnitude\_app} (30 bins).
        \item Boxplot de \texttt{temperature\_K} separado por \texttt{spectral\_class}.
        \item Violin plot opcional (\texttt{seaborn}) para comparar distribuciones.
      \end{itemize}
    \item Guardar el notebook como \texttt{Taller12\_Apellidos.ipynb} y subir a CANVAS.
  \end{enumerate}
\end{frame}

% -----------------------------------------------------------------------------
\section{Pautas de Evaluación}
% -----------------------------------------------------------------------------
\begin{frame}{Criterios de Corrección}
  \begin{itemize}
    \item \textbf{Exactitud estadística} (35 \%)
    \item \textbf{Calidad de las gráficas} (25 \%)
    \item \textbf{Claridad de código y comentarios} (20 \%)
    \item Buen uso de \textbf{pandas / NumPy} (20 \%)
  \end{itemize}
  \vspace{0.2cm}
  Trabajo individual \(\neq\) copiar/pegar: cada grupo debe entender su solución.
\end{frame}

% -----------------------------------------------------------------------------
\section{Lanzamiento de la Tarea V}
% -----------------------------------------------------------------------------
\begin{frame}{Tarea V – Análisis Estadístico de Datos Reales}
  \begin{block}{Resumen}
    \begin{itemize}
      \item Dataset a elección (proporcionamos 3 opciones en CANVAS).
      \item Entregar un informe Jupyter (\texttt{.ipynb}) con:
        \begin{enumerate}
          \item Limpieza y exploración (EDA).
          \item Estadística descriptiva completa.
          \item Al menos \textbf{tres} tipos de gráficas (histograma, boxplot, scatter, violin, etc.).
          \item Conclusiones físicas o astronómicas breves.
        \end{enumerate}
      \item Plazo: \textbf{2 junio 2025, 23:55}.
    \end{itemize}
  \end{block}
\end{frame}

% -----------------------------------------------------------------------------
\section{Tiempo de Trabajo}
% -----------------------------------------------------------------------------
\begin{frame}{Cuenta Regresiva}
  \Huge{\centering 40 minutos para el Taller 12}
  \vspace{0.4cm}
  \normalsize
  \begin{itemize}
    \item Dividan tareas: importación, estadística, gráficos.
    \item Revisen \texttt{df.describe()} para acelerar.
  \end{itemize}
\end{frame}

% -----------------------------------------------------------------------------
\section{Cierre}
% -----------------------------------------------------------------------------
\begin{frame}{Cierre y Próximos Pasos}
  \begin{itemize}
    \item Subir el notebook antes del fin de la clase.
    \item Revisión y feedback en 48 h.
    \item Consultas sobre la \textbf{Tarea V}: foro CANVAS.
  \end{itemize}
  \vspace{0.4cm}
  \Large{\centering ¡Buen trabajo!}
\end{frame}

\end{document}
