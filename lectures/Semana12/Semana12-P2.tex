\documentclass[10pt]{beamer}
\usetheme[progressbar=frametitle]{metropolis}
\usepackage{appendixnumberbeamer}
\usepackage{fancyvrb}
\usepackage{booktabs}
\usepackage[scale=2]{ccicons}
\usepackage{pgfplots}
\usepgfplotslibrary{dateplot}
\usepackage{type1cm}
\usepackage{lettrine}
\usepackage{ragged2e}
\usepackage{xspace}
\newcommand{\themename}{\textbf{\textsc{metropolis}}\xspace}
\usepackage{graphicx} % Allows including images
\usepackage{booktabs} % Allows the use of \toprule, \midrule and \bottomrule in tables
\usepackage[utf8]{inputenc} %solucion del problema de los acentos.
\usepackage{xcolor}
\usepackage{verbatim}

\definecolor{LightGray}{gray}{0.9}
% Paleta de azules estilo Python/matplotlib
\definecolor{PythonBlue}{RGB}{31, 119, 180}      % Azul principal matplotlib
\definecolor{LightPythonBlue}{RGB}{174, 199, 232} % Azul claro para fondos
\definecolor{DarkPythonBlue}{RGB}{23, 90, 135}    % Azul oscuro para líneas

% Colores para tema oscuro tipo VS Code/Colab
\definecolor{DarkBackground}{RGB}{40, 44, 52}     % Fondo oscuro similar a VS Code
\definecolor{CodeText}{RGB}{171, 178, 191}        % Texto gris claro para código
\definecolor{DarkFrame}{RGB}{60, 63, 65}          % Color del marco/borde

% Colores para tema claro
\definecolor{LightBackground}{RGB}{248, 248, 242} % Fondo claro tipo GitHub
\definecolor{LightCodeText}{RGB}{51, 51, 51}      % Texto oscuro para tema claro
\definecolor{LightFrame}{RGB}{220, 220, 220}      % Marco gris claro

\usepackage{minted}

%%%%%%%%%%%%%%%%%%%%%%%%%%%%%%%%%%%%%%%%%%%%%%%%%%%%%%%%%%%%%%%%%%%%%%%%%%%%%%%%%%%%%%
% CONFIGURACIÓN DE TEMAS - CAMBIA AQUÍ PARA ALTERNAR
% ========================================================================
% Descomenta UNA de las siguientes líneas para elegir el tema:

% TEMA OSCURO (VS Code style)
%\newcommand{\mytheme}{oscuro}
%\usemintedstyle{monokai}
%\newcommand{\mybgcolor}{DarkBackground}

% TEMA CLARO (GitHub style) - comenta las 3 líneas de arriba y descomenta estas:
 \newcommand{\mytheme}{claro}
 \usemintedstyle{default}
 \newcommand{\mybgcolor}{LightBackground}

% ========================================================================
\newcommand{\mypyfile}[1]{\inputminted[linenos=true, fontsize=\footnotesize, frame=lines, framesep=5\fboxrule,framerule=1pt]{python}{#1}}

% Comando personalizado para código Python inline
\newcommand{\pycode}[1]{\begin{minted}{python}#1\end{minted}}

% ========================================================================
% COMANDOS ESPECÍFICOS PARA CADA TEMA
% ========================================================================
% Comando para tema oscuro específico
\newcommand{\pyoscuro}[1]{%
\begin{minted}[
  bgcolor=DarkBackground,
  style=monokai,
  breaklines,
  frame=lines,
  framesep=2mm,
  baselinestretch=1.2,
  linenos,
  fontsize=\footnotesize
]{python}
#1
\end{minted}
}

% Comando para tema claro específico
\newcommand{\pyclaro}[1]{%
\begin{minted}[
  bgcolor=LightBackground,
  style=default,
  breaklines,
  frame=lines,
  framesep=2mm,
  baselinestretch=1.2,
  linenos,
  fontsize=\footnotesize
]{python}
#1
\end{minted}
}
%%%%%%%%%%%%%%%%%%%%%%%%%%%%%%%%%%%%%%%%%%%%%%%%%%%%%%%%%%%%%%%%%%%%%%%%%%%%%%%%%%%%%%



%%%%%%%%%%%%%%%%%%%%%%%%%%%%%%%%%%%%%%%%%%%%%%%%%%%%%%%%%%%%%%%%%%%%%%%%%%%%%%%%%%%%%%
% Configuración global para todos los bloques minted de Python
% Usa automáticamente el tema seleccionado arriba
\setminted[python]{
  breaklines,
  frame=lines,
  framesep=2mm,
  baselinestretch=1.2,
  bgcolor=\mybgcolor,
  linenos, 
  fontsize=\footnotesize
} % Configuración dinámica según tema elegido
%%%%%%%%%%%%%%%%%%%%%%%%%%%%%%%%%%%%%%%%%%%%%%%%%%%%%%%%%%%%%%%%%%%%%%%%%%%%%%%%%%%%%%



%%%%%%%%%%%%%%%%%%%%%%%%%%%%%%%%%%%%%%%%%%%%%%%%%%%%%%%%%%%%%%%%%%%%%%%%%%%%%%%%%%%%%%
% Configuración de colores del tema con paleta azul Python
\setbeamercolor{progress bar}{fg=DarkPythonBlue,bg=LightPythonBlue}
\setbeamercolor{title separator}{fg=DarkPythonBlue,bg=white!50!black}
\setbeamercolor{frametitle}{fg=white,bg=PythonBlue}
\title[PCFI161]{Programaci\'on para F\'isica y Astronom\'ia}
\subtitle{Departamento de Física.}

\newcommand{\myfront}{
\author[PCFI161]{Corodinadora: C Loyola \\ Profesores C Femenías / F Bugini / D Basantes}
\institute[UNAB]{Universidad Andrés Bello \\ Departamento de Física y Astronomía}
\date{Primer Semestre 2025}
}

\titlegraphic{
  \includegraphics[width=.08\textwidth]{1) Logo/logo-tux.png}\hfill
  \includegraphics[width=.3\textwidth]{1) Logo/logo-unab.png}\hfill
  \includegraphics[width=.08\textwidth]{1) Logo/logo-python.png}
}

\makeatletter
\setbeamertemplate{title page}{
  \begin{minipage}[b][\paperheight]{\textwidth}
    \vfill%
    \ifx\inserttitle\@empty\else\usebeamertemplate*{title}\fi
    \ifx\insertsubtitle\@empty\else\usebeamertemplate*{subtitle}\fi
    \usebeamertemplate*{title separator}
    \ifx\beamer@shortauthor\@empty\else\usebeamertemplate*{author}\fi
    \ifx\insertdate\@empty\else\usebeamertemplate*{date}\fi
    \ifx\insertinstitute\@empty\else\usebeamertemplate*{institute}\fi
    \vfill
    \ifx\inserttitlegraphic\@empty\else\inserttitlegraphic\fi
    \vspace*{1cm}
  \end{minipage}
}
\makeatother

% Configuración personalizada del frametitle para incluir la sección
\makeatletter
\setbeamertemplate{frametitle}{%
  \nointerlineskip%
  \begin{beamercolorbox}[%
      wd=\paperwidth,%
      sep=0pt,%
      leftskip=\@ifundefined{metropolis@frametitle@padding}{12pt}{\metropolis@frametitle@padding},%
      rightskip=\@ifundefined{metropolis@frametitle@padding}{12pt}{\metropolis@frametitle@padding},%
    ]{frametitle}%
  \@ifundefined{metropolis@frametitlestrut@start}{}{\metropolis@frametitlestrut@start}%
  {\textcolor{LightPythonBlue}{\insertsectionhead}\ifx\insertsectionhead\@empty\else\ $\ni$  \fi}\insertframetitle%
  \nolinebreak%
  \@ifundefined{metropolis@frametitlestrut@end}{}{\metropolis@frametitlestrut@end}%
  \end{beamercolorbox}%
}
\makeatother

\makeatletter
% Configuración segura de longitudes del tema Metropolis
\@ifundefined{metropolis@titleseparator@linewidth}{}{\setlength{\metropolis@titleseparator@linewidth}{2pt}}
\@ifundefined{metropolis@progressonsectionpage@linewidth}{}{\setlength{\metropolis@progressonsectionpage@linewidth}{2pt}}
\@ifundefined{metropolis@progressinheadfoot@linewidth}{}{\setlength{\metropolis@progressinheadfoot@linewidth}{2pt}}
\makeatother



% ----------  NUEVO: minted ----------
\usepackage{minted}        % Requiere compilar con -shell-escape
\setminted{
  fontsize=\scriptsize,
  linenos,
  baselinestretch=1,
  breaklines
}

\title{S12-P2 (Sesión 24)\\Repaso guiado}
\author{PCFI161}
\date{21 mayo 2025}

\begin{document}
\myfront{}

% ───────────────────────────────────────────────
\begin{frame}
  \titlepage
\end{frame}

\begin{frame}
  \frametitle{Ruta de la sesión (60 min)}
  \tableofcontents
\end{frame}

\metroset{block=fill}

% ============================================================
\section{Repaso \& Calentamiento (40 min)}
% ============================================================

\begin{frame}{Mini-quiz relámpago (3 min)}
\begin{block}{¿Recuerdas…?}
\begin{enumerate}
  \item ¿Para qué sirve \texttt{df.info()}?
  \item ¿Cómo obtengo la moda de una columna numérica?
  \item ¿Qué gráfico elegirías para detectar \textit{outliers}?~\footnote{En un conjunto de datos, un \textbf{outlier} (dato atípico) es un valor que se aleja de manera notable del patrón general. Respuesta : \textbf{boxplot}}
\end{enumerate}
\end{block}
{\scriptsize Respuestas rápidas en voz alta, sin nota.}
\end{frame}

\begin{frame}[fragile]{Creamos un dataset sintético de ¡pandas gigantes! (7 min)}
\begin{minted}{python}
import pandas as pd, numpy as np
np.random.seed(42)
n = 200
df_syn = pd.DataFrame({
    "height_m": np.random.normal(2.5, 0.3, n),     # altura
    "weight_kg": np.random.normal(150, 15, n),     # peso
    "bamboo_daily_kg": np.random.normal(25, 4, n), # bambú ingerido
    "continent": np.random.choice(
        ["Asia", "Europa", "América"], n, p=[.6,.25,.15])
})
\end{minted}

\begin{itemize}
  \item Datos inventados → sólo práctica.
  \item Recordamos \texttt{head()}, \texttt{describe()}, \texttt{value\_counts()}.
\end{itemize}
\end{frame}

\begin{frame}[fragile]{Trucos rápidos de inspección (5 min)}
\begin{minted}{python}
df_syn.isnull().sum()                       # ¿faltantes?
df_syn.groupby("continent")["weight_kg"].mean()
df_syn.quantile([.25, .75])                 # cuartiles
\end{minted}
\end{frame}

% ──────────────────────────────────────────────────────────
\begin{frame}[fragile]{¿Usemos \texttt{seaborn}?}
\begin{itemize} \footnotesize
  \item \textbf{Se apoya en \texttt{matplotlib}} todo lo que aprendes de la \textit{librería base} sigue siendo válido.
  \item \textbf{Ajustes estéticos por defecto}: paletas perceptualmente uniformes, estilos limpios y tipografías coherentes sin configuración extra.
  \item \textbf{API de alto nivel}: llamadas concisas (\texttt{sns.boxplot}, \texttt{sns.violinplot}, \texttt{sns.regplot}) que automatizan pasos de formateo y estadísticas.
  \item \textbf{Gráficos estadísticos listos} (distribuciones, relaciones, catplots) sin escribir cálculos preliminares.
  \item \textbf{Uso en investigación}: astrónomos y astrofísicos lo emplean para explorar catálogos masivos (Gaia, SDSS) y publicar figuras claras en revistas científicas.
  \item Conocemos \texttt{matplotlib} porque es \emph{el} motor de renderizado, pero mostramos \texttt{seaborn} para que veas que existen atajos y estilos listos para producción.
\end{itemize}

\small{
\begin{block}{Mensaje clave}
Aprende la base (\texttt{matplotlib}), pero no dudes en aprovechar atajos como \texttt{seaborn} cuando necesites  rototipar visualizaciones estadísticas rápidamente.
\end{block}
}

\end{frame}
% ──────────────────────────────────────────────────────────


\begin{frame}[fragile]{Gráficos de repaso}
\begin{minted}{python}
# Scatter altura vs peso
df_syn.plot.scatter("height_m", "weight_kg",
                    title="Pandas gigantes sintéticos")

# Heatmap de correlación exprés
import seaborn as sns, matplotlib.pyplot as plt
sns.heatmap(df_syn.corr(numeric_only=True), annot=True, square=True)
plt.title("Matriz de correlación")
\end{minted}

\begin{itemize}
  \item Repasamos scatter y heatmap.
  \item Discusión rápida: ¿correlación altura-peso?, ¿ingesta de bambú?
\end{itemize}
\end{frame}

\begin{frame}{Transición a Tarea}
\Large
Repaso completado. ¡Vamos a la Tarea!
\vspace{0.4cm}

\small
Recuerde archivar su notebook para entrega.
\end{frame}

% ============================================================
\section{Trabajo en equipo (reforzar)}
% ============================================================

\begin{frame}{Dataset real: \texttt{stars\_brightness.csv}}
\begin{itemize}
  \item 500 estrellas (muestra HIPPARCOS).
  \item Archivo csv : \url{https://gitarra.cl/lectures/gfiles/-/raw/main/pcfi161/S12/stars_brightness.csv}
  \item Columnas: \texttt{magnitude\_app}, \texttt{temperature\_K},
        \texttt{spectral\_class}, \texttt{radius\_solar}, y \texttt{metallicity\_Fe\_H}.
\end{itemize}
\end{frame}

\begin{frame}{Instrucciones}
Trabajo en grupos de 2-3  
\begin{enumerate}
  \item Leer el CSV con \texttt{pandas}.
  \item Calcular: media, mediana, desvío estándar, IQR.
  \item Graficar:
        \begin{itemize}
          \item Histograma de \texttt{magnitude\_app} (20 bins)
          \item Boxplot de \texttt{temperature\_K} por \texttt{spectral\_class}
          \item \emph{Opcional}: violin plot comparativo
        \end{itemize}
  \item Guarden su trabajo para estudiar.
\end{enumerate}
\end{frame}

%\begin{frame}{Criterios de corrección}
%\begin{itemize}
%  \item Exactitud estadística 35 %
%  \item Calidad de gráficas 25 %
%  \item Claridad de código/comentarios 20 %
%  \item Uso correcto de \texttt{pandas}/NumPy 20 %
%\end{itemize}
%\end{frame}

% ============================================================
%\section{Cuenta regresiva}
% ============================================================
%\begin{frame}{⏱ 20 min de Tarea}
%\huge\centering ¡Manos a la obra!
%\end{frame}

% ============================================================
\section{Cierre}
% ============================================================
\begin{frame}{Próximos pasos}
\begin{itemize}
  \item Guarden/descarguen su notebook antes de terminar la clase.
  \item Si tiene dudas: ¡consulte!.
  \item ¡Prepárense para la Solemne 2!
\end{itemize}
\end{frame}

\end{document}
